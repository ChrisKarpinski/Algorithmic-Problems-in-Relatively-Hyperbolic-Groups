\documentclass[12pt]{article}
\usepackage[utf8]{inputenc}
\usepackage[T1]{fontenc}
\usepackage{lmodern}
\usepackage{graphicx}
\newcommand{\vs}{\vskip10pt}
\newcommand{\bvs}{\vskip40pt}
\usepackage{amsmath}% http://ctan.org/pkg/amsmath
\usepackage{amsthm}
\usepackage{amssymb}
\usepackage[export]{adjustbox}% http://ctan.org/pkg/adjustbox
\usepackage[margin=0.5in]{geometry}
\setlength{\parindent}{0pt}
\usepackage{setspace}
\usepackage{enumitem}
\usepackage{float}
\usepackage{tabularx,ragged2e}


\title{Algorithmic Problems in Relatively Hyperbolic Groups}
\author{Chris Karpinski (101064314), Supervisor: Inna Bumagin}
\date{April 14, 2021}

\begin{document}
	
	\maketitle
	
	\begin{abstract}
		
		In this thesis, we examine the theory of hyperbolic groups and their generalizations: relatively hyperbolic groups. We examine some equivalent definitions of relatively hyperbolic groups, outline important algebraic and geometric properties and we construct algorithms to solve decision problems in relatively hyperbolic groups. 
		
	\end{abstract}
	
	\tableofcontents
	
	
	\newpage
	\section{Introduction and Preliminaries}
	
	Hyperbolic metric spaces and hyperbolic groups were first introduced in 1987 by Mikhail Gromov in his seminal paper [11]. This class of groups generalizes many familiar classes of groups such as finite groups and finitely generated free groups as well as the structure of certain groups encountered in algebraic topology (namely, fundamental groups of negatively curved manifolds). Hyperbolic groups have a number of interesting algebraic, geometric and algorithmic properties, which makes them a useful class of groups to study. Also introduced in Gromov's paper [11] are the more general relatively hyperbolic groups. The notion of relative hyperbolicity of a group is not only characterized by the group as a whole but depends on the relative geometry of subgroups within the group. Due to their rich algebraic, geometric and combinatorial properties, relatively hyperbolic groups provide a nice setting to discuss solvability of algorithmic problems, given knowledge of the solvability of certain problems in subgroups of the group. Studying these properties of relatively hyperbolic groups and their culmination in the solvability of an array of algorithmic problems will be the main focus of this paper. We begin by reviewing key concepts in group theory and metric space theory, from which we introduce the study of geometric group theory and develop the basic theory of relatively hyperbolic groups, examining these structures from a combinatorial, geometric, algebraic and finally, an algorithmic viewpoint. 
	
	\subsection{Free Groups and Group Presentations}
	
	\vs 
	
	We begin by discussing basic theory of free groups and group presentations that we will need in this paper. For a more elementary review of group theory, see the appendix. 
	
	\vs
	
	In the section on operations on subgroups and groups generated by subsets we saw how we can describe a group as being generated or built up algebraically by elements of some subset. Elements of this subset (which we call generators) can also satisfy some algebraic relations. Every group can be described in terms of generators and relations that these generators satisfy. Such a description of a group is called a \textit{group presentation} and is the subject of this subsection. Group presentations are one of the most convenient and concise ways to express a group. Group presentations are closely linked to free groups, which we construct first. 
	
	\vs 
	
	Let $X$ be any set. Let $X^{-1}$ be the set of formal inverses of elements of $X$. We refer to the set $X^{\pm} := X \cup X^{-1}$ as the \textit{alphabet} and we refer to elements of this set as \textit{letters} in the alphabet $X^{\pm}$. A \textit{word} $w$ in the alphabet $X^{\pm}$ is a finite string of letters from $X^{\pm}$, that is, $w$ has the form $w = x_1 x_2 ... x_n$ where $x_i \in X^{\pm}$ for each $i$. We now add a binary operation on the set of all words in $X^{\pm}$. We define multiplication of two words $w_1 = x_1 ... x_n$ and $w_2 = y_1 ... y_m$ simply by concatenating the two words: $w_1 w_2 := x_1 x_2  ... x_n y_1 y_2 ... y_m$. It is easy to see that the resulting multiplication is associative. We also see that the identity element of this mulitplication is the empty word (a string of length 0 in the alphabet). The set $X^{\pm}$ together with this operation has the structure of a \textit{monoid} (recall that a monoid is a set equipped with a binary operation that is associative and admits an identity element, but does not necessarily admit an inverse for every element of the set), called the \textit{free monoid on X}, which we will denote by $X^*$, and which will be useful to us later. To obtain a group structure, we will need to add in cancelation of a letter with its formal inverse. 
	
	\vs 
	
	The process of \textit{reduction} of a word $w$ consists of removing all pairs of letters $xx^{-1}$ or $x^{-1}x$ from $w$. We denote the reduction of $w$ by $\overline{w}$. A \textit{freely reduced word} $w$ is a word that has no instances of a letter and its formal inverse being adjacent to each other, which occurs if and only if $w = \overline{w}$. We then restrict to set of freely reduced words in $X^{\pm}$, which we denote by $F(X)$ and we update our binary operation on two freely reduced words to be concatenation of the words followed by reduction: that is, given freely reduced words $w_1 = x_1...x_n$ and $w_2 = y_1...y_m$ we define $w_1 w_2 := \overline{x_1...x_n y_1...y_m}$. In $F(X)$, we are therefore able to invert elements, where the inverse of a single letter $x$ is $x^{-1}$ and the inverse of a freely reduced word $w = x_1 ... x_n$ is the freely reduced word $w^{-1} = x_n^{-1}...x_1^{-1}$. Under this operation, $F(X)$ is therefore a group, called the \textit{free group on X} or \textit{the free group generated by X}. 
	
	\vs 
	
	\textbf{Definition 1.1: } If $n = \vert X \vert$ ($n$ could be any cardinal number), then we say that the \textit{rank} of the free group $F(X)$ is $n$. 
	
	\vs
	
	It turns out that free groups of the same rank are isomorphic (proof omitted), so there is, up to isomorphism, only one free group of each rank. We denote the free group of rank $n$ by $F_n$. Note that $F_0$ is the trivial group $\{e\}$ and $F_1$ is the infinite cyclic group $\mathbb{Z}$.
	
	\vs 
	
	One of the reasons why free groups are important is because every group is a quotient of a free group. Let $G$ be a group generated by some set $X$. We can then define a group homomorphism $\varphi: F(X) \rightarrow G$ by $\varphi(x) = x$ for all $x \in X$. This is a surjective group homomorphism and so by the first isomorphism theorem, we have that $F(X) / \ker(\varphi) \cong G$. This isomorphism allows us to construct a presentation for $G$. First, we must introduce one piece of terminology.
	
	\vs 
	
	\textbf{Definition 1.2: } Given a group $G$ and a subset $R \subseteq G$, the \textit{normal closure of R}, denoted $\langle \langle R \rangle \rangle$, is the intersection of all normal subgroups of $G$ containing $R$. Since the intersection of normal subgroups is a normal subgroup, $\langle \langle R \rangle \rangle$ is a normal subgroup of $G$ containing $R$ and is the smallest (with respect to inclusion) such normal subgroup. 
	
	\vs
	
	Returning to the above group homomorphism $\varphi$, if we take a set $R$ such that $\ker(\varphi) = \langle \langle R \rangle \rangle$, then the \textit{presentation} of $G$ is given as $G := \langle X \vert R \rangle$. The presentation of $G$ encodes the generators of $G$ ($X$) and the relations among these generators encoded in $R$ (i.e. $R$ contains all of the independent relations satisfied by the generators). Note that a presentation for a group is not unique as there can be many different generating sets of a group and there can be redundant relations. We list some examples of group presentations below. 
	
	\vs 
	
	\textbf{Example 1.3 (Examples of group presentations)}: 
	
	\begin{itemize}
		\item If $X$ is any set, a presentation for the free group $F(X)$ is $F(X) = \langle X \vert - \rangle$. This presentation encodes the fact a free group is "free" in the sense that it can be generated by a set in which the generators have no relations (hence they are "free", because they have no constraints). If $X = \{x_1,...,x_n\} $ we often write the presentation of $F(X) = F_n$ by $F_n = \langle x_1,...,x_n \vert - \rangle$. 
		\item A presentation for the finite cyclic group of order $n$ is $\mathbb{Z}_n = \langle a \vert a^n \rangle$
		\item A presentation for the symmetric group $S_n$ ($n \geq 2$) is $S_n = \langle \sigma_1,...,\sigma_{n-1} \vert \sigma_i^2 \text{ } \forall i, [\sigma_i, \sigma_j] \text{ } \forall \vert i - j \vert > 1, (\sigma_i \sigma_{i+1})^3 \text{ } \forall i \rangle$. 
		\item A presentation for the dihedral group of order $2n$ is $D_n$ is $D_n = \langle r,s \vert r^n, s^2, (sr)^2 \rangle$. 
	\end{itemize}

	\textbf{Definition 1.4: } A group $G$ is said to be \textit{finitely presented} if it has a presentation $G = \langle X \vert R \rangle$ where both $X$ and $R$ are finite sets. Most of the groups we encounter in this paper (and indeed, most of the groups studied in geometric group theory) are finitely presented. 
	
	\vs 
	
	\underline{Tietze Transformations}
	
	\vs
	
	An important concept in combinatorial group theory is that of a \textit{Tietze transformation}, which we will use in at least one instance in this paper. We briefly discuss Tietze transformations. 
	
	\vs 
	
	A Tietze transformation is a transformation between two presentations of a given group. 
	
	\vs
	
	\textbf{Definition 1.5: } Let $G$ be any group and let $\langle X \vert R \rangle $ be any presentation. Let $N$ be the normal closure of $R$ in the free group $F(X)$. We define the following modifications to $\langle X \vert R \rangle $. 
	
	\begin{enumerate}[label = (\roman*)]
		\item If $r \in N$, add $r$ to $R$ to obtain a new presentation $\langle X \vert R' \rangle$, where $R' = R \cup \{r\}$. 
		\item If $x \notin X$ and $w \in F(X)$, append $x$ to $X$ and append $r = x^{-1}w$ to $R$ to obtain a new presentation $\langle X' \vert R' \rangle$ where $X' = X \cup \{x\}$ and $R' = R \cup \{r\}$. 
	\end{enumerate}

	A \textit{Tietze transformation} is the passage from $\langle X \vert R \rangle$ to $\langle X' \vert R' \rangle$ by applying $(i)$ or $(ii)$, or the application of $(i)$ or $(ii)$ in the opposite direction, going from $\langle X' \vert R' \rangle$ to $\langle X \vert R \rangle$.
	
	\vs

	It is important to note that two presentations $\langle X \vert R \rangle$ and $\langle X' \vert R' \rangle$ related by a finite sequence of Tietze transformations define isomorphic groups. Indeed, in each of the above Tietze transformations, $\langle X \vert R' \rangle$ is a group isomorphic to $G$, as in $(i)$, $\langle \langle R' \rangle \rangle = \langle \langle R \rangle \rangle = N$ because $r \in N$, and in $(ii)$, this operation effectively adds $w$ to the generating set $X$ of $G$ because adding the relator $r = x^{-1}w$ gives $x = w$ in the group, however, as $w \in F(X)$, $w \in \langle X \rangle$ in $G$, so this transformation yields the same group as $G$. By induction, it is then evident that applying a finite sequence of Tietze transformations gives a presentation defining a group isomorphic to $G$. It turns out that the converse of the above is also true for finite presentations (i.e. two finite presentations defining isomorphic groups are connected by a finite sequence of Tietze transformations), as can be seen in Proposition 2.1 of [13]. However, this fact will not be of use to us, so we omit the proof. 
	
	\vs 
	
	\underline{Free Products}
	
	\vs
	
	An operation on groups that will have some relevance to us later and which has a nice expression in terms of group presentations is the free product. 
	
	\vs 
	
	\textbf{Definition 1.6: } Let $G = \langle X_1 \vert R_1 \rangle$ and $H = \langle X_2 \vert R_2 \rangle$ be groups. The \textit{free product} of $G$ with $H$, denoted $G*H$, is the group with presentation $\langle X_1 \cup X_2 \vert R_1 \cup R_2 \rangle$. 
	
	\vs 
	
	Given groups $A,B,G$, with $A = \langle a_1,... \vert r_1,... \rangle$ and $B = \langle b_1,... \vert s_1,... \rangle$ and group homomorphisms $\varphi: A \rightarrow G$ and $\psi: B \rightarrow G$, we define the free product of the homomorphisms $\varphi, \psi$ to be the homomorphism $\varphi * \psi: A * B \rightarrow G$ given on generators by $a_i \mapsto \varphi(a_i)$ and $b_i \mapsto \psi(b_i)$. 
	
	\vs 
	
	The following theorem from [13] gives us a convenient representation of each element in a free product and tells us precisely when a word is 1 in a free product. This will come in use in our later discussion on the word problem. 
	
	\vs 
	
	\textbf{Definition 1.7: } Let $G = H_1 * H_2$. A \textit{normal form} or \textit{reduced sequence} is a sequence $g_1,...,g_m, m \geq 0$ of elements of $G$ such that $g_i \neq 1$, for each $i$, $g_i \in H_j$ for some $j$, and for each $1 \leq i \leq n+1$, $g_i, g_{i+1}$ are not in the same free factor $H_j$. 
	
	\vs
	
	\textbf{Theorem 1.8 (Normal Form Theorem for Free Products: )} Let $G = H_1 * H_2$. If $w = g_1...g_m$ where $g_1,...,g_m$ is a non-empty reduced sequence, then $w \neq 1$ in $G$. 
	
	\begin{proof}
		
		We proceed by defining an action of $G$ on a suitable set and from this action, the above will easily follow. 
		
		\vs 
		
		Let $W$ be the set of all reduced sequences in $G$. We first define an action of $H_1$ on $W$ by defining a homomorphism $\phi: H_1 \rightarrow S_W, a \mapsto \bar{a}$ as follows: 
		
		\vs 
		
		If $a = 1$ we set $\bar{a} = \varepsilon$. If $a \neq 1$, we define $\bar{a}$ acting on a  reduced sequence $(g_1,...,g_n)$ as follows: 
		
		\vs 
		
		\[ \bar{a}(g_1,...,g_n) = \begin{cases} 
		(a, g_1,...,g_n) \text{ if } g_1 \in H_2, \\
		(ag_1,...,g_n) \text{ if } g_1 \in H_1 \text{ and } ag_1 \neq 1, \\
		(g_2,...,g_n) \text{ if } g_1 = a^{-1}
		\end{cases}
		\]
		
		We note that in either case $\bar{a}(g_1,...,g_n) $ is a reduced sequence. 
		
		\vs
		
		We claim that the map $\phi$ is a group homomorphism. First, we check that $\phi$ indeed maps into $S_W$ by showing that $\bar{a}$ has an inverse, namely $\overline{a^{-1}}$. First, if $a = 1$, then it is clear that $\overline{1^{-1}} = \overline{1}$ is the inverse of $a$. For $a \neq 1$, we will break into the cases on $g_1$ in the definition of $\bar{a}$:
		
		\begin{itemize}
			\item If $g_1 \in H_2$, then $\bar{a}(g_1,...,g_n) = (a, g_1,...,g_n)$, so $\overline{a^{-1}} \bar{a} (g_1,...,g_n) = \overline{a^{-1}}(a, g_1,...,g_n) = (g_1,...,g_n)$, as $a^{-1}a=1$. 
			\item If $g_1 \in H_1$ and $ag_1 \neq 1$, then $\bar{a}(g_1,...,g_n) = (ag_1,...,g_n)$. Next, $\overline{a^{-1}}(ag_1,...,g_n) = (a^{-1}ag_1,...,g_n) = (g_1,...,g_n)$, because $a g_1 \in H_1$ and $a^{-1} a g_1 = g_1 \neq 1$.
			\item If $g_1 = a^{-1}$, then $\bar{a} (g_1,...,g_n) = (g_2,...,g_n)$. Then $\overline{a^{-1}}(g_2,...,g_n) = (a^{-1}, g_2,...,g_n) = (g_1,...,g_n)$, because $g_2 \in H_2$. 
		\end{itemize}
	
	Thus, in all of the above cases, we see that $\overline{a^{-1}} \bar{a} (g_1,...,g_n) = (g_1,...,g_n)$ for any reduced sequence $(g_1,...,g_n)$. Thus, $\overline{a^{-1}} = \bar{a}^{-1}$. 
	
	\vs
	
	Next, we must prove that $\phi$ is a homomorphism. To do this, we consider similar cases as above. Let $a,b \in H_1$. We may assume that $a \neq b^{-1}$, as if $a=b^{-1}$, we have $\overline{a b} = \overline{1} = \varepsilon = \bar{a} \overline{a^{-1}} = \bar{a} \bar{b}$. We may also assume that $a,b \neq 1$, as if $a = 1$ say, then $\overline{ab} = \bar{b} = \varepsilon \bar{b} = \bar{a} \bar{b}$. 
	
	\begin{itemize}
		\item If $g_1 \in H_2$, then $\overline{a b} (g_1,...,g_n) = (ab, g_1,...,g_n) = \bar{a}(b, g_1,...,g_n) \text{ (because } b \in H_1 \text{ and } ab \neq 1) = \bar{a} \bar{b} (g_1,...,g_n)$.
		\item If $g_1 \in H_1$ and $abg_1 \neq 1$, then $\overline{a b} (g_1,...,g_n) = (abg_1,...g_n)$. Now if $b g_1 \neq 1$, then $(abg_1,...g_n) = \bar{a} (bg_1,...,g_n) = \bar{a} \bar{b} (g_1,...,g_n)$. If $bg_1 = 1$, then $(abg_1,...g_n) = (a, g_2,...,g_n) = \bar{a} (g_2,...,g_n) \text{ (as } g_2 \in H_2) = \bar{a} \bar{b} (g_1,...,g_n)$. 
		\item If $g_1 = (ab)^{-1}$, then $\overline{a b} (g_1,...,g_n) = (g_2,...,g_n)$. Note then that $g_1 \neq a^{-1}, b^{-1}$, since $ab \neq a,b$. We have $\bar{a} \bar{b} (g_1,...,g_n) = \bar{a} (bg_1,...,g_n) = (g_2,...,g_n)$ because $bg_1 \in H_1$ and $ab g_1 = 1$. 
	\end{itemize}

	Therefore, in either case we see that $\overline{a b} (g_1,...,g_n) = \bar{a} \bar{b} (g_1,...,g_n)$, so $\phi$ is a homomorphism. 
		
		We similarly define such a homomorphism $\psi: H_2 \rightarrow S_W$. We then take the free product of these homomorphisms to obtain a homomorphism $\phi * \psi: G \rightarrow S_W$. 
		
		\vs 
		
		Now let $w \in G$. Then $w = g_1...g_n$ for some reduced sequence $(g_1,...,g_n)$ Indeed, because $w \in H_1 * H_2$, we may write $w = w_1...w_m$ where, $w_i \in H_1$ or $w_i \in H_2$ for each $i$. We omit any $w_i = 1$ in the above decomposition. If adjacent factors $w_i, w_{i+1}$ are in the same free factor $H_j$, we may combine $w_i w_{i+1}$ into a single element of $H_j$. We keep doing this until no adjacent factors lie in the same free factor. We then end up with $w = g_1 ... g_n$ where $(g_1,...,g_n)$ is a reduced sequence. We see that $(\phi * \psi)(w)$ sends the empty sequence $()$ to $(g_1,...,g_n)$ (indeed, $\bar{w}(()) = \overline{g_1 ... g_n}(()) = \bar{g_1} \bar{g_2}...\bar{g_n} (()) = (g_1,...,g_n)$, as adjacent $g_i$ are in separate free factors, so populate the empty sequence with all of the $g_i$). Therefore, if $n > 0$, then $(g_1,...,g_n)$ is non-empty so $w \neq 1$. 
		
	\end{proof}

	
	\newpage
	
	\subsection{Metric spaces}
	
	Metric spaces are highly important objects in geometric group theory, as they are the main structures with which we model the geometry of groups. In geometric group theory, we can think of metric spaces as canvases on which groups draw. We outline here some metric space concepts that will be useful later on.
	
	\vs 
	
	Given a metric space $(X,d)$ we denote the closed ball of radius $r$ about a point $x \in X$ by $B_r(x)$. Given a subset $A$ of a metric space $X$ and a number $r \geq 0$, we denote the \textit{r-neighbourhood} of $A$ by $N_r(A)$ (that is, $N_r(A) := \cup_{x \in A} B_r(x)$). Given two closed subsets $A,B \subseteq X$, the \textit{Hausdorff distance} between $A$ and $B$ is defined by: $d_{\text{Haus}}(A,B) = \inf\{r \geq 0: A \subseteq N_r(B) \text{ and } B \subseteq N_r(A)\}$ 
	
	\vs 
	
	\textbf{Definition 1.9: } 
	
	\begin{itemize}
		\item Recall that a \textit{path} in a metric space is a continuous function $\alpha: I \rightarrow X$ where $I \subseteq \mathbb{R}$ is any non-empty interval. Furthermore, we define the \textit{length} of a path $\alpha: [0,1] \rightarrow X$ to be $\ell(\alpha) = \sup \{\sum_{i=0}^{n-1} d(\alpha(t_i), \alpha(t_{i+1}))\}$ where the supremum is over all partitions $\{0 = t_0 < t_1 < ... < t_n = 1\}$ of $[0,1]$. A path is called \textit{rectifiable} if it has finite length.  
		\item A \textit{geodesic path} $\alpha$ between two points $x,y$ is a path $\alpha$ such that $\ell(\alpha) = d(x,y)$. In other words, $\alpha$ is a minimal length path connecting its endpoints. Note that a geodesic path $\alpha: [0, \ell(\alpha)] \rightarrow X$ is an isometric embedding of the interval $[0, \ell(\alpha)]$ into $X$. For this reason, we often denote a geodesic path connecting points $x,y$ by $[x,y]$
		\item A \textit{geodesic ray} is an isometric embedding $\alpha: [0, \infty) \rightarrow X$ and a \textit{bi-infinite geodesic} is an isometric embedding $\alpha: \mathbb{R} \rightarrow X$. 
		\item For $k \geq 0$, a \textit{k-local geodesic} is a path in which every subpath of length at most $k$ is a geodesic. 
	\end{itemize}
	
	\vs 
	
	In geometric group theory, the metric spaces that we most often work with have the property that any pair of points can be joined with a geodesic path. We call such spaces \textit{geodesic metric spaces}. Note that in general, geodesic paths joining a pair of points are not unique (consider a sphere, for example). Spaces in which there is a unique geodesic path joining any pair of points are called \textit{uniquely geodesic spaces}. 
	
	\vs 
	
	One of the most common examples of metric spaces arising in geometric group theory are \textit{graphs}. We examine some graph-theoretic definitions and terminology.
	
	\vs 
	
	\textbf{Definition 1.10: } A \textit{graph} $\Gamma$ consists of a set $V(\Gamma)$ of vertices and a set $E(\Gamma)$ of edges connecting pairs of vertices. An edge connecting vertices $v_1, v_2$ is denoted by $e = (v_1, v_2)$. Given an edge $e = (e_{-}, e_{+})$, we call $e_{-}$ the \textit{origin} of $e$ and $e_{+}$ the \textit{terminus} of $e$. A \textit{path} in a graph is a sequence of adjacent edges. A \textit{connected} graph is a graph in which any two vertices can be joined by a path. A \textit{tree} is a graph with no closed loops. 
	
	\vs
	
	Connected graphs have a natural metric space structure. Let $\Gamma$ be a graph. We first assign each edge $e$ a length $\ell(e) > 0$. Then we homeomorphically identify each edge $e = (e_{-}, e_{+})$ in the graph with the interval $[0,\ell(e)]$, via a homeomorphism $\varphi_e: e \rightarrow [0,\ell(e)]$ such that $\varphi_e (\{e_{-}, e_{+}\}) = \{0,\ell(e)\}$. We can then define the distance between two vertices $x,y$ by $d(x,y) = \inf \{\sum_{i=1}^n \ell(e_i): p = e_1...e_n \text{ is a path from } x \text{ to } y\}$. Note that in general this will yield a pseudo-metric but not a metric unless $\inf \{\ell(e) : e \in E(\Gamma)\}> 0$, which will be the case for all of the graphs we work with in this paper. Then given two points $x,y$ on the same edge $e$, we define their distance to be $d(x,y) = \vert \varphi_e(x) - \varphi_e(y) \vert$. Now given a point $x$ on an edge $e$ and a point $y$ on another edge $f$, we define their distance to be $\min \{d(x,e_{-}) + d(y, f_{-}), d(x,e_{+}) + d(y, f_{-}), d(x,e_{-}) + d(y, f_{+}), d(x,e_{+}) + d(y, f_{+})\}$. 
	
	% add in a picture here
	
	\vs 
	
	\underline{Quasi-isometric embeddings, Quasi-isometries and Quasi-geodesics}
	
	\vs 
	
	In geometric group theory, we are often interested in a weakening of the notions of isometric embedding, isometry and geodesic. Instead of preserving metric space structure \textit{exactly}, it is often sufficient to preserve structure coarsely, or on a large-scale, up to some level of "fuzz" or "error". This is what the notion of \textit{quasi-isometry} is all about. Below, we let $(X,d_X)$ and $(Y, d_Y)$ be metric spaces.
	
	\vs 
	
	\textbf{Definition 1.11: } For constants $\lambda \geq 1$ and $C \geq 0$, a map $f: X \rightarrow Y$ is called a $(\lambda, C)$-\textit{quasi-isometric embedding} if $\forall x,y \in X$, we have $\frac{1}{\lambda} d_X(x,y) - C \leq d_Y(f(x), f(y)) \leq \lambda d_X(x,y) + C$. A map $f: X \rightarrow Y$ is a quasi-isometric embedding if it is a $(\lambda, C)$ quasi-isometric embedding for some $\lambda \geq 1$ and $C \geq 0$. In other words, $f$ is an isometric embedding up to some multiplicative error $\lambda$ and additive error $C$. Indeed, notice that $f$ is an isometric embedding (i.e. preserves distances) if and only if $f$ is a $(1, 0)$-quasi-isometric embedding. 
	
	\vs 
	
	A quasi-isometry is a "coarsely-invertible" (we will define precisely what this means shortly) quasi-isometric embedding. 
	
	\vs 
	
	\textbf{Definition 1.12: } A map $f: X \rightarrow Y$ is called \textit{coarsely surjective} if $\exists D \geq 0$ such that $\forall y \in Y, \exists x \in X$ such that $d_Y(f(x), y) \leq D$. In other words, $Y = N_D(f(X))$. Notice that $f$ is surjective if and only if $f$ is coarsely surjective with constant $D = 0$. 
	
	\vs 
	
	\textbf{Definition 1.13: } A $(\lambda, C)$\textit{-quasi-isometry} is a $(\lambda, C)$-quasi-isometric embedding $f: X \rightarrow Y$ that is coarsely sujective. A map $f: X \rightarrow Y$ is called a \textit{quasi-isometry} if it is a $(\lambda, C)$\textit{-quasi-isometry} for some $\lambda \geq 1$ and $C \geq 0$. Note that quasi-isometric embeddings and quasi-isometries need not be continuous. 
	
	\vs 
	
	\textbf{Definition 1.14: } A $(\lambda, 0)$-quasi-isometric embedding is called a \textit{bi-Lipschitz embedding} and a $(\lambda, 0)$-quasi-isometry is called a \textit{bi-Lipschitz equivalence}.
	
	\vs 
	
	It turns out that being a quasi-isometry is equivalent to being invertible in the coarse sense. 
	
	\vs 
	
	\textbf{Definition 1.15: } A map $g: Y \rightarrow X$ is called a \textit{quasi-inverse} of a map $f: X \rightarrow Y$ if $\exists D \geq 0$ such that $\forall x \in X, d_X((g \circ f)(x), x) \leq D$ and $\forall y \in Y, d_Y((f \circ g) (y), y) \leq D$. Quasi-inverses are not unique in general.
	
	\vs 
	
	\textbf{Proposition 1.16: } A quasi-isometric embedding $f: X \rightarrow Y$ is a quasi-isometry if and only if it has a quasi-inverse. Moreover, if $f: X \rightarrow Y$ is a quasi-isometry and if $g: Y \rightarrow X$ is a quasi-inverse of $f$ then $g$ is a quasi-isometry.
	
 	\begin{proof}
 		
 		"$\implies$": Suppose $f: X \rightarrow Y$ is a quasi-isometry. Then it is a $(\lambda, C)$-quasi-isometric embedding for some $\lambda \geq 1$ and $C \geq 0$ and is coarsely surjective for some constant $D \geq 0$. Given $y \in Y$, by coarse-surjectivity, we may choose an $x_y \in X$ such that $d_Y(f(x_y), y) \leq D$. Define $g: Y \rightarrow X$ by $g(y) = x_y$. We claim that $g$ is a quasi-inverse of $f$. Given $x \in X$, we have $f(x) \in Y$ so $d_Y(f(g(f(x)), f(x)) \leq D$. Since $f$ is a $(\lambda, C)$-quasi-isometric embedding, we have $\frac{1}{\lambda}d_X(g(f(x)), x) - C \leq d_Y(f(g(f(x)), f(x)) \implies d_X(g(f(x)), x) \leq \lambda(d_Y(f(g(f(x)), f(x)) + C) \leq \lambda(D + C)$. In addition, given $y \in Y$, by definition of $g$ we have $d_Y(f(g(y)), y) \leq D \leq \lambda(D + C)$. Therefore, $g$ is a quasi-inverse of $f$, with constant $\lambda(D + C)$. 
 		
 		\vs 
 		
 		"$\impliedby$": Suppose that $f$ has quasi-inverse $g: Y \rightarrow X$. We need to show that $f$ is coarsely surjective. We have that $\exists D \geq 0$ such that $d_Y((f \circ g)(y), y) \leq D$. Thus, given $y \in Y$, setting $x = g(y)$ gives $d_Y(f(x), y) \leq D$. Therefore, $f$ is coarsely surjective and thus a quasi-isometry. 
 		
 		\vs 
 		
 		For the "moreover" statement, it suffices to show that $g$ is a quasi-isometric embedding, because we know that it has quasi-inverse $f$, so applying the first part of the proposition would yield that $g$ is a quasi-isometry. 
 		\vs
 		Given $y_1, y_2 \in Y$, since $f$ is a $(\lambda, C)$-quasi-isometric embedding we have that $\frac{1}{\lambda} d_X(g(y_1), g(y_2)) - C \leq d_Y(f(g(y_1)), f(g(y_2))) \leq \lambda d_X(g(y_1), g(y_2)) + C$. Re-arranging this inequality yields $\frac{1}{\lambda} (d_Y(f(g(y_1)), f(g(y_2))) - C) \leq d_X(g(y_1), g(y_2)) \leq \lambda (d_Y(f(g(y_1)), f(g(y_2))) + C) $. Since $\lambda C \geq \frac{C}{\lambda}$, from above we have: 
 		
 		\begin{align*}
 		\frac{1}{\lambda} d_Y(f(g(y_1)), f(g(y_2))) - \lambda C &\leq d_X(g(y_1), g(y_2)) \\
 		&\leq \lambda d_Y(f(g(y_1)), f(g(y_2))) + \lambda C
 		\end{align*}
 		
 		Now using the triangle inequality,
 		
 		\begin{align*}
 		d_Y(f(g(y_1)), f(g(y_2))) &\leq d_Y(f(g(y_1)), y_1) + d_Y(y_1, y_2) + d_Y(y_2, f(g(y_2))) \\
 		&\leq 2D + d_Y(y_1,y_2)
 		\end{align*}
 		
 		 where $D$ is the quasi-inverse constant for $f,g$. Therefore, we have $\frac{1}{\lambda} (2D + d_Y(y_1, y_2) - C) \leq  d_X(g(y_1), g(y_2)) \leq \lambda (2D + d_Y(y_1, y_2) + C)$. Lastly since $-\lambda(2D + C) \leq \frac{1}{\lambda} (2D - C)$, we have: $\frac{1}{\lambda} d_Y(y_1, y_2) - \lambda (2D + C) \leq d_X(g(y_1), g(y_2)) \leq \lambda d_Y(y_1, y_2) + \lambda(2D + C)$. Thus, $g$ is a $(\lambda, \lambda(2D + C))$-quasi-isometric embedding. Therefore, $g$ is a quasi-isometry. 
 		
 	\end{proof}
 
 	Two metric spaces $(X, d_X), (Y, d_Y)$ are called \textit{quasi-isometric} if there exists a quasi isometry $f: X \rightarrow Y$. From the coarse-structure point of view commonly adopted in geometric group theory, quasi-isometries are the structure preserving maps we care about, so that if two spaces are quasi-isometric, we can say that they are essentially the same when looked at from "far away". 
 
 	\vs 
 	
 	\textbf{Example 1.17 (Examples of quasi-isometries and quasi-isometric spaces)}:
 	
 	\begin{itemize}
 		\item For any $n \in \mathbb{N}$, the inclusion $\mathbb{Z}^n \hookrightarrow \mathbb{R}^n$ is a quasi-isometry. Indeed, using the restriction of the Euclidean metric on $\mathbb{R}^n$ to $\mathbb{Z}^n$, we see that the inclusion is a (1,0) quasi isometric embedded. In addition, given $x = (x_1,...,x_n) \in \mathbb{R}^n$, for each $x_i, \exists z_i \in \mathbb{Z}$ such that $\vert x_i - z_i \vert \leq 1/2$, so $\vert(x_1,...,x_n) - (z_1,...,z_n) \vert = (\sum_{i=1}^n (x_i - z_i)^2)^{1/2} \leq (\sum_{i=1}^n (1/2)^2)^{1/2} = n^{1/2}/2$. Therefore, the inclusion is coarsely surjective for constant $n^{1/2}/2$. So, the inclusion is a quasi-isometry. 
 		\item Any bounded metric space $(M,d)$ is quasi-isometric to a singleton. Indeed, let $D < \infty$ be the diameter of $M$. Let $f: M \rightarrow \{*\}$. Then $f$ is a $(1, M)$-quasi-isometric embedding since for any $x,y \in M$, $d(x,y) - M \leq 0 = d_*(f(x), f(y)) \leq d(x,y) + M$ ($d_*$ denotes the metric on $\{*\}$) and furthermore $f$ is surjective. Therefore, $f$ is a quasi-isometry. 
 		\item The map $f: \mathbb{R} \rightarrow \mathbb{R}$ given by $f(x) = x^3$ is not a quasi-isometry. Indeed, we will show that $f$ is not a quasi-isometric embedding. Suppose that $f$ were a $(\lambda, C)$ quasi-isometric embedding for some $\lambda \geq 1, C \geq 0$. Then we would have $\frac{1}{\lambda} \vert x - y \vert - C \leq \vert x^3 - y^3 \vert \leq \lambda \vert x - y \vert + C$ for all $x,y \in \mathbb{R}$. But taking $y = 0$, this gives $\frac{1}{\lambda}x - C \leq x^3 \leq \lambda x + C$ for all $x \geq 0$. However, this is absurd as, for example, $\lim_{x \rightarrow \infty} \frac{x^3}{\lambda x + C} = \infty$, which contradicts the rightmost inequality. 
 		\item Let $\Gamma$ be a graph with combinatorial metric with the lengths of edges being bounded above by $B < \infty$. Then the vertex set $V(\Gamma)$ (as a metric subspace of $\Gamma$) is quasi-isometric to $\Gamma$. Indeed, as $V(\Gamma)$ is a metric subspace, the inclusion $\iota: V(\Gamma) \hookrightarrow \Gamma$ is a (1,0) quasi-isometric embedding. Furthermore, $\Gamma = N_{\frac{1}{2} B} (V(\Gamma))$ because every point in $\Gamma$ is a distance of at most $\frac{1}{2} B$ from a vertex. Therefore, $\iota$ is coarsely surjective. Hence, $\iota$ is a quasi-isometry.
 	\end{itemize}
 
 	\vs 
 	
 	We will also be interested in the weakening of the notion of geodesic. 
 	
 	\vs 
 	
 	\textbf{Definition 1.18: } A $(\lambda, \varepsilon)$\textit{-quasi-geodesic} in a metric space $(X,d)$ is a $(\lambda, \varepsilon)$-quasi-isometric embedding $\alpha: [0,T] \rightarrow X$. Equivalently, for any subpath $\beta$ of $\alpha$, we have $\ell(\alpha) \leq \lambda \ell(\beta) + \varepsilon$. We define a $(\lambda, \varepsilon)$-\textit{guasi-geodesic ray} to be a quasi-isometric embedding of $[0, \infty)$ into $X$ and a $(\lambda, \varepsilon)$-\textit{ bi-infinite quasi-geodesic} is a quasi-isometric embedding of $\mathbb{R}$ into $X$. 
 	
 	\vs 
 	
 	A simple observation that we will make use of later on is that the concatenation of a quasi geodesic with a path of certain length is again a quasi geodesic (with different constants). 
 	
 	\vs 
 	
 	\textbf{Lemma 1.19: } Let $p$ be a $(\lambda, c)$ quasi-geodesic in a metric space $(X,d)$ and let $q$ be a path of length at most $k$ such that $p_{+} = q_{-}$. Then the concatenation $pq$ is a $(\lambda, c + (\lambda + 1)k)$ quasi-geodesic. 
 	
 	\begin{proof}
 		
 		Let $r$ be a subpath of $pq$. We may then picture $r$ as in the figure below, being composed of paths $r_1, r_2$ (if $r$ is a subpath of $p$, then we set $r_2$ to be the empty path and if $r$ is a subpath of $q$ then we set $r_1$ to be the empty path). We then have $\ell(r) = \ell(r_1) + \ell(r_2) \leq \lambda d((r_1)_{-}, (r_1)_{+}) + c + k$ (as $p$ is a $(\lambda, c)$ quasi geodesic so $\ell(r_1) \leq \lambda d((r_1)_{-}, (r_1)_{+}) + c$ and $r_2$ is a subpath of $q$ so $\ell(r_2) \leq \ell(q) \leq k$). Now by the triangle inequality and the fact that $d((r_1)_+, r_+) \leq \ell(r_2)$, we have $d(r_-, (r_1)_+)  \leq d(r_-, r_+) + d((r_1)_+, r_+) \leq d(r_-, r_+) + \ell(r_2) \leq d(r_-, r_+) + k$. Therefore, $\ell(r) \leq \lambda (d(r_-, r_+) + k) + c + k$, as required. 
 		
 		
 		
\begin{figure} [H]
	\centering
	\includegraphics[width=0.7\linewidth]{"../Desktop/Honours Project/IMG_1443"}
	\caption{The subpath \textit{r} of \textit{pq} in the proof of Lemma 1.19}
	\label{fig:img1443}
\end{figure}
 	\end{proof}
 	
 	\vs
 	
	\underline{Hyperbolic Metric Spaces}
	
	\vs 
	
	We now introduce a class of metric spaces that will be integral to the study of relatively hyperbolic groups: hyperbolic metric spaces. There are several equivalent definitions of hyperbolicity of metric spaces. Here, we use the so-called \textit{slim triangle} definition. 
	
	\vs 
	
	\textbf{Definition 1.20: } A \textit{geodesic triangle} in a metric space is a triangle whose edges are geodesic paths. A geodesic triangle is called $\delta$-\textit{slim} for $\delta \geq 0$ if each edge of the triangle is contained the $\delta$-neighbourhood of the union of the other two sides. More precisely, a geodesic triangle with edges $[x,y], [x,z], [y,z]$, is $\delta$-slim if $[x,y] \subseteq N_{\delta} ([x,z] \cup [y,z]), [x,z] \subseteq N_{\delta} ([x,y] \cup [y,z]), \text{ and } [y,z] \subseteq N_{\delta} ([x,y] \cup [x,z])$.
	
	\vs 
	
	Hyperbolic metric spaces are geodesic metric spaces in which all geodesic triangles are \textit{slim} according to the above definition.
	
	\vs 
	
	
	\textbf{Definition 1.21: } A geodesic metric space $X$ is called $\delta$-\textit{hyperbolic} ($\delta \geq 0$) if every geodesic triangle in $X$ is $\delta$-slim. A geodesic metric space is called \textit{hyperbolic} if it is $\delta$-hyperbolic for some $\delta \geq 0$.
	
	\begin{figure} [H]
		\centering
		\includegraphics[width=0.7\linewidth]{"../Desktop/Honours Project/IMG_1452"}
		\caption{A $\delta$-slim triangle. Each edge is $\delta$-close to the union of the other two edges}
		\label{fig:img1452}
	\end{figure}
	
	Examples of hyperbolic metric spaces include any bounded metric space (just take $\delta$ to be the diameter of the metric space) as well as trees (indeed, in trees geodesic triangles have the shape of a tripod (see Figure 3) because a tripod shape is the only possible way to get a triangle without any closed loops, so each side of the triangle is contained in the union of the other two sides, so trees are 0-hyperbolic spaces). A non-example of a hyperbolic space is $\mathbb{R}^n$ for $n > 1$. To see why $\mathbb{R}^n$ is not hyperbolic for all $n > 1$, for each $\delta \geq 0$, consider a right isosceles triangle (with sides being straight lines, which are geodesics in $\mathbb{R}^n$) with legs of length $2(\delta + 1)$ (and therefore hypotenuse of length $2(\delta + 1)\sqrt2$), as shown in Figure 4. Letting $x$ be the midpoint of the hypotenuse, we have that for any point $y$ on the union of the legs, using the Pythagorean theorem, $\vert x - y \vert = \sqrt{(\delta + 1)^2 + (\vert b - y \vert - (\delta + 1))^2} \geq \delta + 1 > \delta$. Thus, the hypotenuse of this triangle is not in the $\delta$-neighbourhood of the union of the legs. 
	
\begin{figure} [H]
	\centering
	\includegraphics[width=0.4\linewidth]{"../Desktop/Honours Project/IMG_1541"}
	\caption{A geodesic triangle $xyz$ in a tree, which has the shape of a tripod. The three sides of the triangle are illustrated by the three colours}
	\label{fig:img1541}
\end{figure}
	
\begin{figure} [H]
	\centering
	\includegraphics[width=0.5\linewidth]{"../Desktop/Honours Project/IMG_1542"}
	\caption{A geodesic triangle in $\mathbb{R}^n$ which is not $\delta$-slim}
	\label{fig:img1542}
\end{figure}
	
	\vs 
	
	From this slim triangle condition, we can establish slimness of some other geometric figures in a hyperbolic space.
	
	\vs 
	
	\textbf{Lemma 1.22: } In a $\delta$ hyperbolic space $(X,d)$, geodesic quadrilaterals are $2\delta$-slim. 
	
	\begin{proof}
		
		We can decompose the quadrilateral into 2 geodesic triangles by drawing a diagonal, as shown in Figure 5. Since $X$ is $\delta$ hyperbolic, each triangle is $\delta$-slim. Thus, taking any point $x$ on any side $p_i$, there exists a point $y$ either on another side $p_j, j \neq i$ of part of the same triangle as $p_i$ or on the diagonal such that $d(x,y) \leq \delta$, by $\delta$-slimness of geodesic triangles. If $y$ is on $p_j$, then we are done, so we assume $y$ is on the diagonal. If $y$ is on the diagonal then again by $\delta$-slimness of geodesic triangles, there exists a point $z$ on another side of the triangle (now this side is an edge of the quadrilateral) such that $d(y,z) \leq \delta$. Then by the triangle inequality, we have $d(x,z) \leq d(x,y) + d(y,z) \leq 2\delta$.
		
		\begin{figure} [H]
			\centering
			\includegraphics[width=0.7\linewidth]{"../Desktop/Honours Project/IMG_1467"}
			\caption{The two cases in the proof of Lemma 1.22}
			\label{fig:img1467}
		\end{figure}
		
	\end{proof}
	
	\textbf{Lemma 1.23: } In a $\delta$-hyperbolic metric space, geodesic bigons (a figure consisting of two geodesics with the same endpoints) are $\delta$-slim. 
	
	\begin{proof}
		
		Let $\alpha, \beta$ be two geodesics with the same endpoints, $x = (\alpha)_- = \beta_-, y = \alpha_+ = \beta_+$. We show that $d_{\text{Haus}}(\alpha, \beta) \leq \delta$. We may assume $x \neq y$ (so that $\ell(\alpha), \ell(\beta) > 0$) because if $x = y$ then $\alpha, \beta$ are both constant paths at the same point and so $d_{\text{Haus}}(\alpha, \beta) = 0 \leq \delta$ in this case. 
		
		\vs 
		
		Choose a point $p \neq x,y$ on $\alpha$ and a point $q \neq x,y$ on $\beta$. We  obtain two geodesic triangles, one with sides $\beta, [x,p], [p,y]$ (geodesics $[x,p], [p,y]$ along $\alpha$) and the other with sides $\alpha, [x, q], [q, y]$ (geodesics $ [x, q], [q, y]$ along $\beta$). Each triangle is $\delta$-slim, so we have $\beta \subseteq N_{\delta}([x,p] \cup [p, y]) = N_{\delta}(\alpha)$ and $\alpha \subseteq N_{\delta}([x,q] \cup [q, y]) = N_{\delta}(\beta)$. Therefore, $d_{\text{Haus}}(\alpha, \beta) \leq \delta$, so the geodesic bigon formed by $\alpha, \beta$ is $\delta$-slim. 
		
\begin{figure} [H]
	\centering
	\includegraphics[width=0.7\linewidth]{"../Desktop/Honours Project/IMG_1519"}
	\caption{Partitioning $\alpha$ in the proof of Lemma 1.23}
	\label{fig:img1519}
\end{figure}

\begin{figure} [H]
	\centering
	\includegraphics[width=0.7\linewidth]{"../Desktop/Honours Project/IMG_1520"}
	\caption{Partitioning $\beta$ in the proof of Lemma 1.23}
	\label{fig:img1520}
\end{figure}

		
	\end{proof}
	
	
	From the above, we also obtain an important closeness fact about geodesics with close endpoints, which will be useful in our discussion of the geometry of relatively hyperbolic groups: 
	
	\vs 
	
	\textbf{Corollary 1.24: } Let $p,q$ be geodesics in a $\delta$ hyperbolic space $(X,d)$ such that $d(p_-, q_-), d(p_+, q_+) \leq k$. Then $d_{\text{Haus}}(p,q) \leq 2 \delta + k$. 
	
	\begin{proof}
		
		Let $a$ be a point on $p$. We construct a geodesic quadrilateral by joining $p_-$ to $q_-$ and $p_+$ to $q_+$ with geodesics $r,s$, respectively. By Lemma 1.22, this quadrilateral is $2 \delta$-slim. Therefore, there exists a point $b$ on $r,q$ or $s$. If $b$ is on $q$, then $d(a,b) \leq 2\delta \leq 2 \delta + k$ and if $b$ is on $r$ (resp. $b$ is on $s$) then $d(a,b) \leq 2\delta$ and also $d(b, p_-) \leq k$ (resp. $d(b, p_+) \leq k$). Thus, we obtain $d(a, p_-) \leq d(a,b) + d(b, p_-) \leq 2\delta + k$ (resp. $d(a, p_+) \leq d(a,b) + d(b, p_+) \leq 2\delta + k$). In any case therefore, we see that for any $a$ on $p$, there exists a point $c$ on $q$ such that $d(a,c) \leq 2 \delta + k$, so $p \subseteq N_{2 \delta + k} (q)$. Swapping the roles of $p,q$, we also see that $q \subseteq N_{2 \delta + k} (p)$. Therefore, $d_{\text{Haus}}(p,q) \leq 2 \delta + k$.
		
		\begin{figure} [H]
			\centering
			\includegraphics[width=0.7\linewidth]{"../Desktop/Honours Project/IMG_1468"}
			\caption{Two cases in the proof of Corollary 1.24}
			\label{fig:img1468}
		\end{figure}
		
	\end{proof}
	
	It turns out that hyperbolicity of geodesic metric spaces is an invariant under quasi-isometry. 
	
	\vs 
	
	\textbf{Theorem 1.25: } Suppose that $(X,d_X)$ and $(Y, d_Y)$ are geodesic metric spaces and suppose that $Y$ is hyperbolic. If there exists a quasi-isometric embedding $f : X \rightarrow Y$, then $X$ is hyperbolic. 
	
	\vs 
	
	This theorem will come as a corollary of another important theorem which states that in $\delta$-hyperbolic metric spaces, geodesics and quasi geodesics with the same endpoints are uniformly close. 
	
	\vs 
	
	\textbf{Theorem 1.26: } Let $(X,d)$ be a $\delta$-hyperbolic geodesic metric space. Then for any $\lambda \geq 1, c \geq 0$ $\exists R = R(\lambda, c, \delta)$ (i.e. $R$ only depends on $\lambda, c$ and $\delta$) such that for any $(\lambda, c)$-quasi-geodesic $p$ in $X$, and any geodesic $q$ from $p_-$ to $p_+$, we have $d_{\text{Haus}}(p,q) \leq R$. 
	
\begin{figure} [H]
	\centering
	\includegraphics[width=0.7\linewidth]{"../Desktop/Honours Project/IMG_1453"}
	\caption{An illustration of Theorem 1.26}
	\label{fig:img1453}
\end{figure}
	
	\vs 
	
	We sketch the proof of this theorem using the following Lemmas, which allows us to replace an arbitrary quasi-geodesic with a continuous quasi-geodesic: 
	
	\vs 
	
	\textbf{Lemma 1.27 (taming quasi geodesics): } Let $X$ be a geodesic metric space. Given any $(\lambda, \epsilon)$ quasi-geodesic $c: [a,b] \rightarrow X$ there exists a continuous $(\lambda, \epsilon')$ quasi geodesic $c': [a,b] \rightarrow X$ such that the following hold: 
	
	\begin{enumerate}[label = (\roman*)]
		\item $c'(a) = c(a)$, $c'(b) = c(b)$
		\item $\epsilon' = 2(\lambda + \epsilon)$
		\item For any $t \leq t' \in [a,b]$, we have $\ell(c_{\restriction [t, t']}) \leq k_1 d(c'(t), c'(t')) + k_2$ where $k_1 = \lambda (\lambda + \epsilon)$ and $k_2 = (\lambda \epsilon' + 3)(\lambda + \epsilon)$
		\item  $d_{\text{Haus}}(c, c') \leq \lambda + \epsilon$
	\end{enumerate} 

	\begin{proof}
		
		We define $c'$ to be $c$ on the set $\Sigma = \{a,b\} \cup (\mathbb{Z} \cap (a,b))$. We then complete our definition of $c'$ by joining the $c$-images of points of this set with geodesics and letting $c'$ be the resulting curve (see Figure 10). 
		
\begin{figure} [H]
	\centering
	\includegraphics[width=0.7\linewidth]{"../Desktop/Honours Project/IMG_1505"}
	\caption{The paths $c$ and $c'$ from Lemma 1.27}
	\label{fig:img1505}
\end{figure}
		
		\vs
		
		Note that for any geodesic segment $\gamma$ joining two adjacent points in the image of $\Sigma$, we have $\ell(\gamma) = d(\gamma_{-}, \gamma_{+}) \leq \lambda + \epsilon$ (because $d(\gamma_{-}, \gamma_{+}) = d(c(t_1), c(t_2))$ for some $t_1, t_2 \in \Sigma$ and $d(c(t_1), c(t_2)) \leq \lambda \vert t_1 - t_2 \vert + \epsilon$ because $c$ is a $(\lambda, \epsilon)$ quasi-geodesic, and we have $\vert t_1 - t_2 \vert \leq 1$, because $t_1, t_2$ are adjacent points in $\Sigma = \{a,b\} \cup (\mathbb{Z} \cap (a,b))$). As each point on $c'$ is a distance of at most $1/2 \ell(\gamma)$ from a point on $c(\Sigma)$, for $\gamma$ the geodesic segment which contains the given point, we have $c' \subseteq N_{1/2 (\lambda + \epsilon)} (c)$. Similarly, the length of any segment $\alpha$ of $c$ connecting two adjacent points of $c(\Sigma)$, is bounded above $\lambda + \epsilon$ because $c$ is a $(\lambda, \epsilon)$ quasi-geodesic and adjacent $t_1, t_2$ in $\Sigma$ are separated by a distance of at most 1. Therefore, the distance of any point on $\alpha$ to the endpoints of $\alpha$ is at most $1/2 \ell(\alpha) \leq 1/2 (\lambda + \epsilon)$, so $c \subseteq N_{1/2 (\lambda + \epsilon)} (c')$. Therefore, $d_{\text{Haus}}(c, c') \leq 1/2 (\lambda + \epsilon)$, proving (4). 
		
		\vs 
		
		Now for any $t \in [a,b]$ let $[t]$ denote the closest point of $\Sigma$ to $t$. We now show $c'$ is a $(\lambda, 2(\lambda + \epsilon))$ quasi-geodesic: 
		
		\vs 
		
		\begin{align*}
		d(c'(t'), c'(t)) &\leq d(c'([t]), c'([t'])) + \lambda + \epsilon, \hspace{0.5cm} \text{because each segment of } c' \text{ has length at most } \lambda + \epsilon \\
		&\leq \lambda \vert [t] - [t'] \vert + \epsilon + \lambda + \epsilon \hspace{0.5cm} \text{because } c_{\restriction \Sigma} = c'_{\restriction \Sigma} \text{ and } c \text{ is a } (\lambda, \epsilon) \text{ quasi-geodesic}. \\
		&\leq \lambda( \vert t - t' \vert + 1) + \lambda + 2 \epsilon \hspace{0.5cm} \text{ because points of } \Sigma \text{ are separated by a distance of at most } 1. 
 		\end{align*}
 		
 		And for the reverse inequality: 
 		
		\vs 
		
		\begin{align*}
		\frac{1}{\lambda} \vert t - t' \vert - 2(\lambda + \epsilon) &\leq \frac{1}{\lambda} (\vert t - t' \vert - 1) - (\lambda + 2 \epsilon) \hspace{0.5cm} \text{ because } \lambda \geq \frac{1}{\lambda}. \\
		&\leq \vert [t] - [t'] \vert - (\lambda + 2 \epsilon) \hspace{0.5cm} \text{ because } \vert t - [t] \vert, \vert t' - [t'] \vert \leq \frac{1}{2}. \\
		&\leq d(c'([t]), c'([t'])) - (\lambda + \epsilon) \hspace{0.5cm} \text{because } c_{\restriction \Sigma} = c'_{\restriction \Sigma} \text{ and } c \text{ is a } (\lambda, \epsilon) \text{ quasi-geodesic}. \\
		&\leq d(c'(t), c'(t')) \hspace{0.5cm} \text{because each segment of } c' \text{ has length at most } \lambda + \epsilon 
		\end{align*}
		
		Therefore, $c'$ is a $(\lambda, 2(\lambda + \epsilon))$ quasi-geodesic, proving (ii). It remains to prove (iii). 
		
		\vs 
		
		We note that for all $n,m \in [a,b] \cap \mathbb{Z}$, we have $\ell(c'_{\restriction [n,m]}) = \sum_{i=n}^{m-1} \leq (\lambda + \epsilon) \vert m - n \vert$ because $c'_{\restriction [n,m]}$ consists of geodesic segments of length at most $\lambda + \epsilon$ joining images of integers between $n,m$. 
		
		\vs 
		
		Similarly, taking into account the endpoints $a,b$, we have for any $n,m \in [a,b] \cap \mathbb{Z}$, $\ell(c'_{\restriction [a,m]}) \leq (\lambda + \epsilon)(m-a+1)$ and $\ell(c'_{\restriction [n,b]}) \leq (\lambda + \epsilon)(b- n +1)$ because we have at most $(m - a + 1)$ geodesic segments of $c'$ in $c'_{\restriction [a,m]}$ and each geodesic segment has length at most $\lambda + \epsilon$, and similarly for $c'_{\restriction [n,b]}$. Therefore, for any $t, t' \in [a,b], \ell(c'_{\restriction [t, t']}) \leq (\lambda + \epsilon)(\vert [t] - [t'] \vert + 2)$. Indeed, if both $t,t'$ are integers or endpoints of $[a,b]$, then this follows by the above inequalities. If one of $t,t'$ is in $\Sigma$ and the other is not (say, WLOG, $t \in \Sigma$, while $t' \notin \Sigma$), then $t = [t]$ while $t' \neq [t']$, and if $t'$ is in between an integer and $b$, then $\ell(c'_{\restriction [t, t']}) \leq \ell(c'_{\restriction [t, b]}) \leq (\lambda + \epsilon)(b - t + 1) \leq  (\lambda + \epsilon)(\vert [t] - [t'] \vert + 2)$, because $b \leq [t'] + 1$ and if $[t'] + 1 \in [a,b]$, then $t' \leq [t'] + 1$, so we obtain $\ell(c'_{\restriction [t, t']}) \leq \ell(c'_{\restriction [t, [t']+1]}) \leq (\lambda + \epsilon)(\vert t - ([t'] + 1) \vert) \leq (\lambda + \epsilon)(\vert t - [t'] \vert) + 1 \leq (\lambda + \epsilon)(\vert [t] - [t'] \vert + 2)$. Finally, we obtain the desired inequality by considering similar cases as above when neither $t$ nor $t'$ are in $\Sigma$. 
		
		\vs 
		
	In addition, as $c'$ is a $(\lambda, \epsilon')$ quasi-geodesic (established above), we have that $d(c'(t), c'(t')) \geq \frac{1}{\lambda} \vert t - t' \vert - \epsilon' \geq \frac{1}{\lambda} (\vert [t] - [t'] \vert -1) - \epsilon'$ (where the last inequality follows because $t,t'$ are a distance of at most $1/2$ from $[t], [t']$), and therefore, re-arranging, $\vert [t] - [t'] \vert \leq \lambda(d(c'(t), c'(t')) + \epsilon') + 1$. Then, plugging the previous inequality into our estimation of $\ell(c'_{\restriction [t, t']})$, we obtain: $\ell(c'_{\restriction [t, t']}) \leq (\lambda + \epsilon)(\lambda(d(c'(t), c'(t')) + \epsilon') + 3)$, as required. 
		
	\end{proof}
	
	\vs 
	
	\textbf{Lemma 1.28: } Let $(X,d)$ be a $\delta$-hyperbolic metric space. Let $c$ be a rectifiable path in $X$ with $\ell(c) \geq 1$. If $[p,q]$ is a geodesic connecting the endpoints of $c$, then for every $x \in [p,q]$, we have: $d(x,c) \leq \delta \vert \log_2( \ell(c)) \vert + 1$. 
	
	\begin{proof}
		
		%We proceed by induction on $\ell(c)$. 
		
		\vs 
		
		If $\ell(c) \leq 1$, then $d(p,q) \leq 1$, so if $x \in [p,q]$ then $d(x, \{p,q\}) \leq \frac{1}{2}$, hence $d(x,c ) \leq \frac{1}{2} < 1 = \delta \vert \log_2 (\ell(c)) \vert + 1$. Thus, we may assume $\ell(c) > 1$.
		
		\vs 
		
		Note that because $c$ is rectifiable, we may reparameterize $c$ to obtain $c(0) = p$ and $c(1) = q$. Find an $N \in \mathbb{N}$ such that $\ell(c)/2^{N+1} < 1 \leq \ell(c)/2^N$ (we can choose the largest $N \in \mathbb{Z}_{\geq 0}$ such that $\ell(c) \geq 2^N$). 
		
		\vs 
		
		We form a geodesic triangle $\Delta_1$ by connecting the points $c(0), c(\frac{1}{2})$ and $c(1)$. This triangle is $\delta$-slim, so for any $x \in [p,q]$, we have a point $y_1$ on either $[c(0), c(\frac{1}{2})]$ or $[c(\frac{1}{2}), c(1)]$ such that $d(x,y_1) \leq \delta$. If $y_1 \in [c(0), c(\frac{1}{2})]$, then we consider another geodesic triangle $\Delta_2$ connecting points $c(0), c(\frac{1}{4}), c(\frac{1}{2})$, as shown below. Analogously, if $y_1 \in [c(\frac{1}{2}), c(1)]$, then we set $\Delta_2$ to be a geodesic triangle connecting the points $c(\frac{1}{2}), c(\frac{3}{4}), c(1)$. Suppose, without loss of generality, that we have the former case. We then choose a point $y_2$ in $\Delta_2 \setminus \Delta_1$ such that $d(y_1, y_2) \leq \delta$, which we can do by the $\delta$-slimness of the triangle shown in green in the figure 11. We continue this inductively, where having constructed $n$ geodesic triangles, we construct the next triangle $\Delta_{n+1}$ which has common side $[c(f_n), c(f_n')]$ with $\Delta_n$, and suppose $[c(f_n), c(f_n')]$ contains the point $y_n$. We denote $f_{n+1} = (f_n + f_n')/2$ and we choose a point $y_{n+1} \in \Delta_{n+1} \setminus \Delta_n$ such that $d(y_n, y_{n+1}) \leq \delta$. 
		
		\vs 
		
		At step $N$ of the above inductive process (where $N$ is defined two paragraphs above), we obtain $y_N$ on a segment $[c(f_n), c(f_n')]$ of $\Delta_{N}$ of length $\ell(c)/2^N$ and (by repeated applications of the triangle inequality), $d(y_N, x) \leq N \delta$. Let $y$ be the closest endpoint of $\gamma_N$ to $y_N$, so that $y$ is on $c$. We then obtain $d(x,y) \leq d(x, y_N) + d(y_N, y) \leq \delta N + \ell(c)/2^{N+1} \leq \delta \log_2 (\ell(C)) +1$. 
		
\begin{figure} [H]
	\centering
	\includegraphics[width=0.7\linewidth]{"../Desktop/Honours Project/IMG_1504"}
	\caption{The inductive procedure in the proof of Lemma 1.28}
	\label{fig:img1504}
\end{figure}
		 		
	\end{proof}
	
	\vs 
	
	We now prove Theorem 1.26: 
	
	\begin{proof}
		
		First, we "tame" $c$ by replacing $c$ with a continuous $(\lambda, \epsilon')$ quasi-geodesic $c'$ as in Lemma 1.27. Let $[p,q]$ be a geodesic connecting the endpoints of $c$. Let $D = \sup \{d(x,c): x \in [p,q]\}$. Note that $D$ is finite by Lemma 1.28. Since the metric $d$ is continuous and since $[p,q]$ is compact (as it is an isometrically embedded compact interval) this supremum is attained, say by some $x_0$ on $[p,q]$. Note that the open ball $B(D, x_0)$ does not intersect $c$ (because if it did intersect $c$ then we would have a point on $c$ of distance less than $D$ to $x_0$, contradicting that $D = d(x_0,c)$). We choose a point $y$ on $[p,x_0]$ as follows: 
		
		\vs 
		
		If $d(x_0, p) < 2D$ then we let $y = p$. Otherwise, we let $y$ be a point on $[p, x_0]$ such that $d(x_0, y) = 2D$. Similarly choose a point $z$ on $[x_0, q]$ according to the same scheme. Let $y', z'$ be points on $c$ such that $d(y, y'), d(z, z') \leq D$ (exist by Lemma 1.28). Let $\gamma$ be a path going from $y$ to $y'$ via a geodesic, then along $c'$, as shown below. We note that by our choice of $y,z,x_0$, we have that $\gamma$ lies outside of the open ball $B(D, x_0)$. Indeed, by above we know that $c$ does not intersect the open ball of radius $D$ about $x_0$, so $B(D, x_0)$ can only intersect $\gamma$ along $[y, y']$ and $[z, z']$. First, if $d(x_0, p) \geq 2D$, so that $d(x_0, y) = 2D$, then for any $r$ on $[y, y']$, by the triangle inequality we have $d(r, x_0) \geq d(x_0, y) - d(r,y) > 2D - D = D$, so $[y, y']$ lies outside of $B(D, x_0)$. Similarly $[z,z']$ lies outside of $B(D, x_0)$ if $d(x_0, q) \geq 2D$. Now if $d(x_0, p) < 2D$ and $d(x_0, q) < 2D$ (so that $y =p$ and $z = q$), then we simply have $\gamma = c$, so $\gamma$ lies outside $B(D, x_0)$ in this case as well. 
		
		\vs
		
		Now, by the triangle inequality, we have: $d(y', z') \leq d(y', y) + d(y,z) + d(z, z') \leq 6 D$ and by Lemma 1.27, we have $\ell(\gamma) \leq 6Dk_1 + k_2 + 2D$. Because $\gamma$ lies outside of $B(D, x_0)$, we have $d(x, \gamma) \geq D$ and because $\gamma$ intersects $c'$ and by definition of $c'$, we have $d(x_0, \gamma) \leq D$. Therefore, $d(x_0, \gamma) = D$. By Lemma 1.28, we have $D \leq \delta \vert \log_2 (\ell(\gamma)) \vert + 1 \leq \delta \log_2 (6 D k_1 + k_2 + 2D) + 1$. Thus, we can implicitly derive an upper bound $D_0$ for $D$ which depends only on $\delta, \lambda, \epsilon$. 
		
		
		\begin{figure} [H]
			\centering
			\includegraphics[width=0.7\linewidth]{"../Desktop/Honours Project/IMG_1512"}
			\caption{The path $\gamma$ from the proof of Thereom 1.26}
			\label{fig:img1512}
		\end{figure}
	
	
		\vs 
		
		We now claim that $c' \subseteq N_{R'}([p,q])$ (open neighbourhood), where $R' = D_0 (1 + k_1) + k_2 / 2$. Indeed, let $[a', b']$ be a maximal subinterval of $[a,b]$ such that $c'([a', b'])$ does not intersect $N_{D_0}([p,q])$. Note that such a subinterval exists because we may find a point $r \in [a,b]$ such that $c(r) \notin N_{D_0}([p,q])$ (if such a point does not exist then $c \subseteq N_{D_0}([p,q]) \subseteq N_{R'}([p,q])$ and we are done), then because we are dealing with open neighbourhoods, we can expand an interval containing $r$ until along $c$ we hit the boundary of the open neighbourhood, and then this interval will yield our maximal subinterval $[a', b']$ of $[a,b]$. By Lemma 1.28, we have that $[p,q] \subseteq N_{D_0}(c')$. By connectedness of $[p,q]$, there exist $w \in [p,q], t \in [a, a']$, and $t' \in [b,b']$ such that $d(w,c'(t)), d(w, c'(t')) \leq D_0$. Indeed, if there did not exist $w \in [p,q], t \in [a, a']$, and $t' \in [b,b']$ such that $d(w,c'(t)), d(w, c'(t')) \leq D_0$, then letting $A = \{w \in [p,q]: d(w, c'(t)) > D_0 \text{ for all } t \in [a, a']\}$, $B = \{w \in [p,q]: d(w, c'(t)) > D_0 \text{ for all} t \in [b', b]\}$, we have that $A,B$ are open, $[p,q] = A \cup B$ (because we assume that there does not exist $w \in [p,q], t \in [a, a']$, and $t' \in [b,b']$ such that $d(w,c'(t)), d(w, c'(t')) \leq D_0$) and $A \cap B = \emptyset$ because if $w \in A \cap B$, then $w$ is further than $D_0$ from all points of $c'([a, a'])$ and $c'([b', b])$, and also $w$ is further than $D_0$ from any point of $c'([a', b'])$ by the choice of $[a', b']$, so we obtain that $w$ is further than $D_0$ from every point of $c'([a,b])$, contradicting that $[p,q] \subseteq N_{D_0}(c'([a,b]))$. Therefore, the sets $A,B$ form a disconnection of $[p,q]$, which contradicts the fact that $[p,q]$ is connected (as it is isometric to an interval in $\mathbb{R}$). Choosing such $w, t, t'$, we then have $d(c'(t'), c'(t)) \leq 2D_0$ by the triangle inequality. Therefore, by Lemma 1.27, we obtain $\ell(c'_{\restriction [t, t']}) \leq 2k_1 D_0 + k_2$. It follows that $c' \subseteq N_{R'}([p,q])$ (indeed, if a point $e$ is on $c'_{\restriction [t, t']}$ then $d(e, t) \leq 1/2 \ell(c'_{\restriction [t, t']}) = k_1 D_0 + 1/2 k_2$ or $d(e, t') \leq 1/2 \ell(c'_{\restriction [t, t']}) = k_1 D_0 + 1/2 k_2$, therefore $d(e,w) \leq D_0(1 + k_1) + 1/2 k_2$ in either case, and by maximality of $[a', b']$, we have $c'([a, a']), c'([b, b']) \subseteq N_{D_0}([p,q]) \subseteq N_{R'}([p,q])$, so that $c' \subseteq N_{R'}([p,q])$). By Lemma 1.27, we have $c \subseteq N_{\lambda+\epsilon} (c')$, therefore, $c \subseteq N_{R' + \lambda + \epsilon}([p,q])$. Conversely, we have $[p,q] \subseteq N_{D}(c') \subseteq N_{R'}(c')$ and $c' \subseteq N_{\lambda + \epsilon}(c)$, so $[p,q] \subseteq N_{R' + \lambda + \epsilon}(c)$. Therefore, $d_{\text{Haus}}([p,q], c) \leq R' + \lambda + \epsilon$. Thus, we set $R = R' + \lambda + \epsilon$. 
	
\begin{figure} [H]
	\centering
	\includegraphics[width=0.7\linewidth]{"../Desktop/Honours Project/IMG_1513"}
	\caption{An illustration of how $a'$ and $b'$ are defined in the proof of Theorem 1.26}
	\label{fig:img1513}
\end{figure}

		
	\end{proof}
	
	As a consequence of the above theorem, we have a characterization of hyperbolicity of geodesic metric spaces in terms of slimness of $(\lambda, c)$ quasi-geodesic triangles (that is, triangles whose edges are $(\lambda, c)$ quasi-geodesics). 
	
	\vs 
	
	\textbf{Corollary 1.29: } A geodesic metric space $(X,d)$ is hyperbolic if and only if for every $\lambda \geq 1, c \geq 0$ $\exists M$ such that every $(\lambda, c)$-quasi-geodesic triangle is $M$-slim. 
	
	\vs 
	
	\begin{proof}
		
		Suppose that $X$ is $\delta$-hyperbolic. Let $x,y,z$ be a $(\lambda, c)$ quasi-geodesic triangle in $X$ with vertices $x,y,z$ and edges $p_1$ between $x,y$, $p_2$ between $x,z$ and $p_3$ between $y,z$. We join adjacent vertices $x,y,z$ with geodesics $[x,y], [x,z], [y,z]$. By Theorem 1.26, $\exists R = R(\delta, \lambda, c)$ such that each of the quasi-geodesic edges are $R$-close (in Hausdorff distance) to their corresponding geodesic edges. Since $X$ is $\delta$-hyperbolic, the geodesic triangle with edges $[x,y], [x,z], [y,z]$ is $\delta$-slim. Therefore, the quasi-geodesic triangle is $2R + \delta$-slim (see figure 14). 
		
		
\begin{figure} [H]
	\centering
	\includegraphics[width=0.7\linewidth]{"../Desktop/Honours Project/IMG_1478"}
	\caption{Approximating a quasi-geodesic triangle with a "close" geodesic triangle in Corollary 1.29}
	\label{fig:img1478}
\end{figure}
		
		Conversely, suppose all $(\lambda, c)$-quasi-geodesic triangles are $M$ slim for some constant $M$ depending on $\lambda, c$. In particular, all geodesic (i.e. $(1,0)$-quasi-geodesic) triangles are slim. Hence $X$ is hyperbolic.  
	\end{proof}

	We also have, as a corollary of the above theorem and our previous geodesic closeness results the following: 
	
	\vs 
	
	\textbf{Lemma 1.30: } For any $\delta \geq 0, \lambda \geq 1, c \geq 0, k \geq 0$, $\exists K = K(\delta, \lambda, c, k)$ such that if $(X,d)$ is a $\delta$-hyperbolic space and if $p,q$ are $(\lambda, c)$-quasi geodesics in $X$ with $d(p_-, q_-), d(p_+, q_+) \leq k$ then $d_{\text{Haus}}(p,q) \leq K$. 
	
	\begin{proof}
		
		Join $p_-$ to $p_+$ and $q_-$ to $q_+$ with geodesics $[p_-, p_+], [q_-, q_+]$, respectively. By Theorem 1.26, $d_{\text{Haus}}(p, [p_-, p_+]), d_{\text{Haus}}(q, [q_-, q_+]) \leq M$, for some $M = M(\delta, \lambda, c)$.
		
		\vs
		 By Corollary 1.24, we have $d_{\text{Haus}}([p_-, p_+], [q_-, q_+]) \leq 2\delta + k$. Therefore, $d_{\text{Haus}}(p,q) \leq d_{\text{Haus}}(p, [p_-, p_+]) + d_{\text{Haus}}([p_-, p_+], [q_-, q_+]) + d_{\text{Haus}}([q_-, q_+], q) \leq 2M + 2\delta + k$. Thus, we set $K = 2M + 2\delta + k$.
		
		\begin{figure} [H]
			\centering
			\includegraphics[width=0.7\linewidth]{"../Desktop/Honours Project/IMG_1469"}
			\caption{An illustration of proof of Lemma 1.30}
			\label{fig:img1469}
		\end{figure}
		
	\end{proof}

	As a result of Corollary 1.29, the proof of our main theorem follows: 
	
	\begin{proof}
		
		Consider a geodesic triangle $\Delta$ in $X$ with edges $p_1, p_2, p_3$. Apply the quasi-isometric embedding $f$ to this triangle to obtain a $(\lambda, c)$ quasi-geodesic triangle $f(\Delta)$ in $Y$ with edges $f(p_1), f(p_2), f(p_3)$ (note that the edges $f(p_i)$ are $(\lambda, c)$-quasi-geodesics because the $p_1$ is an isometric embedding (and therefore a $(\lambda, c)$ quasi-isometric embedding) of an interval into $X$ and so the composition $f \circ p_i$ is a $(\lambda, c)$ quasi-isometric embedding of an interval into $Y$). By the above corollary, $f(\Delta)$ is $M$-slim for some $M = M(\delta, \lambda, c)$. Thus, for any $i = 1,2,3$, taking any point $x$ on $p_i$, $\exists y$ on $p_j, i \neq j$ such that $d_Y(f(x), f(y)) \leq M$. But then since $f$ is a $(\lambda, c)$ quasi-isometric embedding we have $d_X(x,y) \leq \lambda d_Y(f(x), f(y)) + c \lambda \leq \lambda M + \lambda c$. Therefore, $\Delta$ is $\lambda (M + c)$-slim. So, $X$ is hyperbolic. 
		
	\end{proof}

	In hyperbolic spaces, $k$-local geodesics turn out to be quasi-geodesics. This will prove useful to us later on when we discuss algorithmic problems. 
	
	\vs 
	
	\textbf{Lemma 1.31: } Let $p: [0, T] \rightarrow X$ be a $k$-local geodesic in a $\delta$-hyperbolic space $X$ for $k > 8\delta$. Then: 
	
	\begin{enumerate} [label = (\roman*)]
		\item For any geodesic $[p(0), p(T)]$ connecting the endpoints of $p$ we have $p \subseteq N_{2 \delta}([p(0), p(T)])$ and $[p(0), p(T)] \subseteq N_{3 \delta}(p)$, so that $d_{\text{Haus}}(p, [p(0), p(T)]) \leq 3 \delta$ 
		\item 	$p$ is a $(\lambda, c)$-quasi-geodesic for $\lambda = \frac{k + 4 \delta}{k - 4\delta}$ and $c = 2\delta$. Note that $\frac{k + 4 \delta}{k - 4\delta} \leq 3$ in this case. 
	\end{enumerate}
	

	
	\begin{proof}
		
		We begin by proving the first assertion in (i), that for any geodesic $[p(0), p(T)]$ connecting the endpoints of $p$ we have $p \subseteq N_{2 \delta}([p(0), p(T)])$. Let $x = p(t), t \in [0, T]$ be a point on $p$ at maximal distance from $[p(0), p(T)]$ (note that such a point $x$ exists by the extreme value theorem, as the metric $d$ is continuous and the image of $p$ is compact because it is the image of the compact interval $[0,T]$ under a continuous function). We consider the following cases: 
		
		\vs 
		
		(a): $t > 4 \delta$ and $T - t > 4 \delta$. Then there exists a subpath of $p$ centered about $x$ of length greater than $8 \delta$ but less than $k$. Indeed, from $t > 4 \delta$ and $T - t > 4 \delta$ we have $T > 8 \delta$. But $\ell(p) = T$, because we may partition $p$ into some number $n$ subsegments of length at most $k$, so projecting the endpoints $(p_i)_{i=1}^n$ of these subsegments onto $[0,T]$, we obtain a partition $(t_i)_{i=1}^n$ of $[0,T]$ (the $t_i$ are such that $p_i = p(t_i)$ for all $i$). But each subsegment is a geodesic, being a subpath of length at most $k$, therefore, $d(p_i, p_{i+1}) = d(p(t_i), p(t_{i+1})) = t_{i+1} - t_i$ for each $i  = 1,...,n-1$. Therefore, $\ell(p) = \sum_{i = 1}^{n-1} \ell(p_i) = \sum_{i = 1}^{n-1} d(p(t_i), p(t_{i+1})) = \sum_{i = 1}^{n-1} (t_{i+1} - t_i) = T$. The same reasoning as above yields $\ell(p_{\vert [s, s']}) = s' - s$ for any $s,s' \in [0,T]$. Therefore, $\ell(p_{\vert [0, t]}) = t > 4 \delta$ and $\ell(p_{\vert [t, T]}) = T - t > 4 \delta$, so we may find the desired subpath by traveling far enough away from $x$ along both directions of $p$. Note that the subpath is a geodesic, having length less than $k$. Let $y, z$ be the endpoints of this subpath. Choose points $y', z'$ on $[p(0), p(T)]$ that are closest possible to $y,z$, respectively. We form a geodesic quadrilateral consisting of the edges $[y, y']$, $[y', z']$ (along the geodesic $[p(0), p(T)]$), $[z,z']$ and our above subpath between $z$ and $y$. By the $2 \delta$-slimness of geodesic quadrilaterals (c.f. Lemma 1.22), there is a point $w$ on a side other than our above subpath such that $d(x, w) \leq 2 \delta$. We argue that $w$ must be on $[y', z']$. If $w$ is on $[y, y']$, then we obtain: 
		
		\begin{align*}
		d(x, y') - d(y, y') &\leq d(x,w) + d(w,y') - (d(y, w) + d(w, y')), \hspace{0.25cm} \text{triangle inequality and that } [y,y'] \text{ is a geodesic}. \\
		&= d(x,w) - d(y, w) \\
		&\leq d(x,w) - (d(y,x) - d(x,w)), \hspace{0.25cm} \text{triangle inequality} \\
		&= 2d(x,w) - d(y,x) \\
		&< 4 \delta - 4 \delta, \hspace{0.25cm} \text{as } d(x,w) \leq 2\delta \text{ and } d(y,x) > 4 \delta \\
		&= 0
		\end{align*} 
		
		Therefore, $d(x, y') < d(y,y')$, which contradicts our choice of $x$ being at a maximal distance from points on $[p(0), p(T)]$. Thus, $w$ cannot be on $[y, y']$. An analogous argument shows $w$ cannot be on $[z,z']$. Therefore, $w$ must be on the subpath of $[y', z']$. Therefore, we conclude that $p \subseteq N_{2 \delta}([p(0), p(T)])$. 
		
\begin{figure} [H]
	\centering
	\includegraphics[width=0.7\linewidth]{"../Desktop/Honours Project/IMG_1516"}
	\caption{The first case in the proof of Lemma 1.31 (a)}
	\label{fig:img1516}
\end{figure}
		
		\vs 
		
		(b) Next, we suppose that $t \leq 4\delta$ or $T - t \leq 4 \delta$. Suppose first that $t \leq 4 \delta$. If also $T - t \leq 4 \delta$ then $\ell(p) = T \leq 8 \delta < k$, so $p$ is a geodesic and by Lemma 1.23 ($\delta$-slimness of geodesic bigons) we conclude that $p \subseteq N_{2 \delta} ([p(0), p(T)])$. Therefore, we assume that $T - t > 4 \delta$. We may thus choose a point $z$ on $p$ such that $\ell([x, z]_p) = 4 \delta$ (i.e. the length along $p$). Note then that $\ell([p(0), z]_p) = \ell([p(0), x]_p) + \ell([x, z]_p) \leq 4 \delta + 4 \delta < k$, so $[p(0), z]_p$ is a geodesic. Let $z'$ be a point on $[p(0), p(T)]$ at minimal distance to $z$ and connect $z$ to $z'$ with a geodesic $[z,z']$. We then have a geodesic triangle consisting of the sides $[p(0), z]_p, [p(0), z']$ and $[z, z']$. By $\delta$-slimness of geodesic triangles, we have a point $w$ either on $[z,z']$ or $[p(0), z']$ such that $d(x, w) \leq \delta$. We argue similarly as in (a) that $w$ must be on $[p(0), z']$. If $w$ is on $[z,z']$, we then obtain: 
		
		\begin{align*}
		d(x, z') - d(z, z') &\leq d(x,w) + d(w,z') - (d(z, w) + d(w, z')), \hspace{0.25cm} \text{triangle inequality and that } [z,z'] \text{ is a geodesic}. \\
		&= d(x,w) - d(z, w) \\
		&\leq d(x,w) - (d(z,x) - d(x,w)), \hspace{0.25cm} \text{triangle inequality} \\
		&= 2d(x,w) - d(z,x) \\
		&\leq 2 \delta - 4 \delta, \hspace{0.25cm} \text{as } d(x,w) \leq \delta \text{ and } d(z,x) = 4 \delta \\
		&< 0
		\end{align*} 
		
		Therefore, we obtain $d(x, z') < d(z,z')$, contradicting our choice of $x$ being at the maximal distance among points of $p$ and points of $[p(0), p(T)]$. Therefore, $p \subseteq N_{\delta}([p(0), p(T)]) \subseteq N_{2 \delta}([p(0), p(T)]) $ in this case. Note that we obtain the same conclusion if we assume $T - t \leq 4 \delta$ rather than $t \leq 4 \delta$. 
	
\begin{figure} [H]
	\centering
	\includegraphics[width=0.7\linewidth]{"../Desktop/Honours Project/IMG_1517"}
	\caption{The second case in the proof of Lemma 1.31 (a)}
	\label{fig:img1517}
\end{figure}
		
		\vs 
		
		We now move on to prove the second part of (i). Let $y$ be a point on $[p(0), p(T)]$. By above, we have $p \subseteq N_{2 \delta}([p(0), p(T)]) = N_{2 \delta} ([p(0), y] \cup [y, p(T)]) = N_{2 \delta} ([p(0), y]) \cup N_{2 \delta} ([y, p(T)])$ and $p$ intersects both sets non-trivially. Therefore, by connectedness, we must have $p \subseteq N_{2 \delta} ([p(0), y]) \cap N_{2 \delta} ([y, p(T)])$. Thus, taking an arbitrary $x \in p$, there exists $q \in [p(0), y]$ such that $d(x, q) \leq 2 \delta$ and there exists $r \in [y, p(T)]$ such that $d(x, r) \leq 2 \delta$. We then obtain a geodesic triangle with edges $[x, q], [x, r], [q,r]$. By $\delta$-slimness of the triangle, there exists $w$ on $[x, q]$ or on $[x,r]$ such that $d(y, w) \leq \delta$. In either case, since $d(x, q) \leq 2 \delta$ and $d(x, r) \leq 2 \delta$, by the triangle inequality we obtain $d(x, y) \leq 3 \delta$. Thus, $[p(0), p(T)] \subseteq N_{3 \delta}(p)$. 
		
\begin{figure} [H]
	\centering
	\includegraphics[width=0.7\linewidth]{"../Desktop/Honours Project/IMG_1518"}
	\caption{An illustration of the proof of the second assertion in Lemma 1.31 (i)}
	\label{fig:img1518}
\end{figure}
		
		\vs
		
		We give a sketch of the proof of (ii).
		
		\vs 
		
		We write $p: [0, T] \rightarrow X$ for the path. First, we note that for any $t < t' \in [0,T]$ we have $d(p(t), p(t')) \leq  t' - t $. Indeed, we construct a partition $\{t = t_0 < t_1 < ... < t_N = t'\}$ such that $\ell(p_{\vert [t_i, t_{i+1}]}) = k$ for each $0 \leq i  \leq n-1$ and $\ell(p_{\vert [t_{n-1}, t_{n}]}) < k$, as follows. If $\ell(p) < k$ then $p$ is a geodesic and the result is proven. If $\ell(p) \geq k$, we may choose $t_1 \in [t,t']$ such that $\ell(p_{\vert [t, t_1]}) = k$. If $\ell(p_{\vert [t_1, t']}) \geq k$, then we choose another point $t_2 \in [t_1, t']$ such that $\ell(p_{\vert [t_1, t_2]}) = k$. We continue this process until we get some $n$ such that $\ell(p_{\vert [t_{n-2}, t_{n-1}]}) < k$. We then set $t_{n} = t'$ and $t_0 = t$. Since $p$ is a $k$-local geodesic, we have that $p_{\vert [t_i, t_{i+1}]}$ is a geodesic, so $\ell(p_{\vert [t_i, t_{i+1}]}) = d(p(t_i), p(t_{i+1})) = \vert t_i - t_{i+1} \vert$. Thus, $d(p(t), p(t')) \leq \sum_{i = 0}^{n-1} d(p(t_i), p(t_{i+1})) = \sum_{i=0}^{n-1} (t_{i+1} - t_i) = t' - t$ (the sum telescopes). Therefore, we need only show that $\frac{1}{\lambda}  \vert t - t' \vert - c \leq d(p(t), p(t'))$ for every $t,t' \in [0, T]$. 
		
		\vs 
		
		To do this, using (i), we may join the endpoints of $p$ with a geodesic $[p(0), p(T)]$ such that $p \subseteq([p(0), p(T)])$. Then we let $k' = k/2 + 2 \delta$. Similar to above, we may divide $p$ into some number $M$ pieces of length $k'$ and a piece of length $\eta < k'$ at the end. We denote $p_i$ the resulting subpaths/pieces. We therefore have that $Mk' + \eta = \ell(p) = \sum_{i=1}^n \ell(p_i) = \sum_{i=1}^n d(p(s_{i-1}), p(s_i)) = \sum_{i=1}^n (s_i - s_{i-1}) = T$, where $(s_i)_{i=1}^n$ is the partition of $[0, T]$ obtained from the endpoints of the $p_i$. Note that from $Mk' + \eta = T$, we have $M \leq \frac{T}{k'}$. For each endpoint $x$ of the subpaths $p_i$, we project onto $[p(0), p(T)]$ by choosing a point $x'$ on $[p(0), p(T)]$ with $d(x, x') \leq 2 \delta$ (such a point $x'$ on $[p(0), p(T)]$ exists because $p$ and $[p(0), p(T)]$ are $2 \delta$ close). Using the choice of $k'$ and the $\delta$-hyperbolicity of $X$, we can show that the sequence of projected points on $[p(0), p(T)]$ is monotone (that is, if $x,y$ are points on $p$ with $d(p(0), y) > d(p(0), x)$, then $d(p(0), y') > d(p(0), x')$). 
		
		\vs
		
		
\begin{figure} [H]
	\centering
	\includegraphics[width=0.7\linewidth]{"../Desktop/Honours Project/IMG_1480"}
	\caption{A subpath $p_i$ from the proof of 1.31 (ii)}
	\label{fig:img1480}
\end{figure}
		
		Using the triangle inequality, we obtain that the distance between successive projected points on $[p(0), p(T)]$ is at least $k' - 4 \delta$, and the distance between the last projected endpoint and $p(T)$ is at least $\eta - 2 \delta$. 
		
		
\begin{figure} [H]
	\centering
	\includegraphics[width=0.7\linewidth]{"../Desktop/Honours Project/IMG_1481"}
	\caption{The last subpath from the proof of Lemma 1.31 (ii)}
	\label{fig:img1481}
\end{figure}
		
		 We then obtain $d(p(0), p(T)) \geq M(k' - 4 \delta) + \eta - 2 \delta = T - 4 \delta M - 2 \delta$, using the fact that $T = Mk' + \eta$. Then, using $M \leq \frac{T}{k'}$, we obtain $d(p(0), p(T)) \geq T - 4 \delta \frac{T}{k'} - 2 \delta = T \frac{k' - 4 \delta}{k'} - 2 \delta \geq T \frac{k' - 4 \delta}{k' + 4 \delta} - 2 \delta = \frac{1}{\lambda} T - c$. To derive this lower bound, we only used the fact that $p$ is a $k$-local geodesic. As any subpath $p_{\vert [t, t']}$ of $p$ is also a $k$-local geodesic (for if not, then this would yield a subpath of $p$ of length at most $k$ which is not geodesic, contradicting the fact that $p$ is a $k$-local geodesic), this lower bound holds for $p_{\vert [t, t']}$, that is for any $t,t' \in [0, T]$, we have $d(p(t), p(t') \geq \frac{1}{\lambda} \vert t - t' \vert - c$. Combining this with $d(p(t), p(t')) \leq \vert t - t' \vert \leq \lambda \vert t - t' \vert + c$, we obtain $ \frac{1}{\lambda} \vert t - t' \vert - c \leq d(p(t), p(t')) \leq \lambda \vert t - t' \vert + c$. Therefore, $p$ is a $(\lambda, c)$ quasi-geodesic. 
				
	\end{proof}

	\vs 

	\underline{Gromov boundary of a hyperbolic space}
	
	\vs 
	
	A useful construction that comes up often in the theory of hyperbolic groups and spaces is the \textit{Gromov boundary} of a space. We can think of this as representing the "points at infinity" of a hyperbolic space. 
	
	\vs 
	
	To define the Gromov boundary, we first need to define an equivalence relation on geodesic rays. Let $(X,d)$ be a hyperbolic metric space. We define a relation on geodesic rays as follows: given two geodesic rays $\alpha, \beta$ in $X$, $\alpha \sim \beta$ if and only if $d_{\text{Haus}}(\alpha, \beta) < \infty$ (that is, points on $\alpha$ and $\beta$ remain within a bounded distance of each other). The \textit{Gromov boundary}, denoted $\partial X$,  is then defined as the set of all $\sim$ equivalence classes of geodesic rays. We show next that there is a natural topology on $\bar{X} := X \cup \partial X$. It will be convenient first to introduce the following notation. 
	
	\vs 
	
	\textbf{Definition 1.32: } A \textit{generalized ray} is a geodesic $c: I \rightarrow X$, where $I$ is either a finite length interval $[0, R]$ or $I = [0, \infty)$. If $I = [0, R]$, then we can extend $c$ to $[0, \infty]$ by letting $c(t) = c(R)$ for all $t \geq R$ (which includes $c(\infty) = c(R)$). If $I = [0, \infty)$, then we can also extend $c$ to $[0, \infty]$ by letting $c(\infty)$ be the equivalence class of $c$ on $\partial X$. 
	
	\vs 
	
	We are now ready to define a topology on $\bar{X}$. 
	
	\vs 
	
	Fix a point $p \in X$. We define a notion of convergence in $\bar{X}$ as follows. Given a sequence of points $(x_n)_{n}$ in $\bar{X}$ and a point $x \in \bar{X}$, we write $x_n \rightarrow x$ if there exists a sequence $(c_n)_n$ of generalized rays with $c_n(0) = p$ and $c_n(\infty) = x_n$ for all $n$, such that for every subsequence of $(c_n)_n$, there exists exists a subsequence of this subsequence which converges on compact sets to a generalized geodesic $c$ such that $c(\infty) = x$. We then define a topology on $\bar{X}$ as follows: a subset $B$ of $\bar{X}$ is closed if and only if for every sequence $(x_n)_n$ of elements of $B$ such that $x_n \rightarrow x$, we have $x \in B$. 
	
	\vs
	
\begin{figure} [H]
	\centering
	\includegraphics[width=0.7\linewidth]{"../Desktop/Honours Project/IMG_1540"}
	\caption{A generalized ray $c$ in a space $X$ with its right endpoint at infinity on the Gromov boundary $\partial X$}
	\label{fig:img1540}
\end{figure}
		
		
		\newpage
	\subsection{Geometric Group Theory}
	
	In geometric group theory, there are two main ways to understand a group geometrically: (1) by associating to the group a certain geometric object called the \textit{Cayley graph of a group} and (2) by observing the geometry of the action of a group on metric spaces. We begin by examining the Cayley graph of a group. 
	
	\vs 
	
	\underline{The Cayley Graph of a Group and Groups as Metric Spaces}
	
	\vs 
	
	We begin by showing how every group can naturally be made into a metric space. Fix a generating set $S$ for a group $G$. Define the \textit{word length} of an element $g \in G$ with respect to $S$, denoted $\vert g \vert_S$, by $\vert g \vert_S = \min \{n \geq 0: g = \Pi_{i=1}^n s_i, \text{ for } s_i \in S^{\pm}\}$ (i.e. $\vert g \vert_S$ is the minimum length of a product of generators from $S$ or their inverses which represents $g$ in $G$). Defining the empty product to be $e$, we have that $\vert g \vert_S = 0 \iff g = e$. We now examine a couple of simple yet useful facts about word length $\vert \cdot \vert_S$. 
	
		\vs 
	
	\textbf{Proposition 1.33: Properties of word length}: Fix a generating set $S$ for a group $G$. We denote $\vert \cdot \vert$ the word length with respect to $S$. Then: 
	
	\begin{enumerate}[label = (\alph*)]
		\item For any $g \in G$, $\vert g^{-1} \vert = \vert g \vert$. 
		\item For any $g,h \in G$, $\vert gh \vert \leq \vert g \vert + \vert h \vert$.
	\end{enumerate}
	
	\begin{proof}
		
		(a): Denote $n = \vert g \vert$. Then $\exists s_1,...,s_n \in S^{\pm}$ such that $g = \Pi_{i=1}^n s_i$. Then $g^{-1} = \Pi_{i=n}^{1} s_i^{-1}$, so $\vert g^{-1} \vert \leq n$. But if $\vert g^{-1} \vert < n$, then we could write $g^{-1} = \Pi_{i = 1}^m t_i$ for $m < n$ and $t_i \in S^{\pm}$, so then $g = \Pi_{i = m}^1 t_i^{-1}$, which would yield $\vert g \vert \leq m < n$, a contradiction. Therefore, $\vert g^{-1} \vert = n$. 
		\vs 
		(b): Denote $\vert g \vert = n$ and $\vert h \vert = m$. Then there exist $s_1,...,s_n,t_1,...,t_m \in S^{\pm}$ such that $g = \Pi_{i=1}^n s_i$ and $h = \Pi_{i=1}^m t_i$. Then $gh = (\Pi_{i=1}^n s_i)(\Pi_{j=1}^m t_j) = s_1...s_nt_1,...,t_m$. This product has $n+m$ elements of $S^{\pm}$, so $\vert gh \vert \leq n + m = \vert g \vert + \vert h \vert$.
		
	\end{proof}
	
	These properties of word length allow us to define a metric on $G$. Define $d_S$ on $G$ by $d_S(g,h) = \vert g^{-1}h \vert_S$. Then $d_S$ is a metric since for $g,h,k \in G$: $d_S(g,h) = 0 \iff \vert g^{-1} h \vert_S = 0 \iff g^{-1} h = e \iff g = h$, $d_S(g,h) = \vert g^{-1}h \vert_S = \vert (g^{-1}h)^{-1} \vert_S = \vert h^{-1} g \vert_S = d_S(h,g)$ and $d_S(g,h) = \vert g^{-1} h \vert_S = \vert g^{-1} k k^{-1} h \vert_S \leq \vert g^{-1} k \vert_S + \vert k^{-1} h \vert_S = d_S(g,k) + d_S(k,h)$. This metric is called the \textit{word metric} on $G$ with respect to the generating set $S$. As a slight aside, note that given any set $X$, we can define word length on the free monoid on $X$, $X^*$, as the number of letters in a word $W \in X^*$ (i.e. if $W = x_1...x_n$, where $x_i \in X$, $n \geq 0$, then the word length of $W$ is $n$). We will denote the word length on $X^*$ by $\vert \vert \cdot \vert \vert_X$ or simply $\vert \vert \cdot \vert \vert$ when the monoid generating set $X$ is clear. The free monoid word length will come in use to us in several places throughout this thesis.
	
	\vs 
	
	In addition to the group itself being a metric space, we can associate another geometric object to the group: its Cayley graph.
	
	\vs 
	
	The \textit{Cayley graph} of $G$ with respect to $S$, denoted $\Gamma(G, S)$, is constructed as follows. The vertex set is $G$. For the edges, for any pair of vertices $g, h$ such that $h = gs$ for some $s \in S$, we draw an edge $e$ labeled by $s$ connecting $g$ and $h$. Given a path $p$ in a Cayley graph $\Gamma(G,S)$ consisting of edges $e_1,...,e_n$ with each $e_i$ labeled by $s_i$, the \textit{label} of $p$, denoted $\phi(p)$, is $\phi(p) = s_1...s_n$ (as an element of the free monoid $(S^{\pm})^*$). We make the Cayley graph into a metric space by equipping it with the combinatorial metric. Below are some examples of Cayley graphs of various groups. Notice that the word metric is just the combinatorial metric on the Cayley graph $\Gamma(G, S)$ restricted to the vertex set $G$. Note also that the Cayley graph with respect to any generating set is a geodesic metric space. 
	
	
\begin{figure} [H]
	\centering
	\includegraphics[width=0.7\linewidth]{"../Desktop/Honours Project/IMG_1444"}
	\caption{The Cayley graph of $\mathbb{Z}_5$ with the given generating set}
	\label{fig:img1444}
\end{figure}

\begin{figure} [H]
	\centering
	\includegraphics[width=0.7\linewidth]{"../Desktop/Honours Project/IMG_1445"}
	\caption{The Cayley graph of a free group with respect to its free basis}
	\label{fig:img1445}
\end{figure}
	\vs 
\begin{figure} [H]
	\centering
	\includegraphics[width=0.7\linewidth]{"../Desktop/Honours Project/IMG_1446"}
	\caption{The Cayley graph of $S_3$ with respect to the given generating set}
	\label{fig:img1446}
\end{figure}
	
\begin{figure} [H]
	\centering
	\includegraphics[width=0.7\linewidth]{"../Desktop/Honours Project/IMG_1447"}
	\caption{A Cayley graph of the rank 2 free abelian group with respect to its standard basis}
	\label{fig:img1447}
\end{figure}
	Cayley graphs allow us to define what it means for a group to be hyperbolic. 
	
	\vs 
	
	\textbf{Definition 1.34: } A group $G$ is called \textit{hyperbolic} if there is a finite generating set $S$ of $G$ such that $\Gamma(G, S)$ is a hyperbolic metric space. 
	
	\vs
	
	\underline{Groups acting on metric spaces}
	
	\vs 
	
	When we consider a group $G$ acting on the metric space $(X,d)$, this action will always be an \textit{isometric} action (i.e. $\forall g \in G, x,y \in X$, we have $d(x, y) = d(g \cdot x, g \cdot y)$), which makes sense because isometries are the isomorphisms in the category of metric spaces. 
	
	\vs 
	
	\textbf{Definition 1.35: } An action of a group $G$ on a metric space $(X,d)$ is called \textit{properly discontinuous} if for every $x \in X$, $\vert \{g \in G: g \cdot x \in B_r(x_0)\}\vert < \infty$ for every $x_0 \in X, r \geq 0$. Intuitively, a properly discontinuous action has only finitely many group elements that take $x$ close to any other point. 
	
	\vs 
	
	\textbf{Definition 1.36: } An action of a group $G$ on a metric space $(X,d)$ is called \textit{cobounded} if there exists some ball $B$ of finite radius such that $G\cdot B (:= \{g \cdot b: g \in G, b \in B\}) = X$. Intuitively, a cobounded action can reach every point in $X$ by acting on a smaller subset. 
	
	\vs 
	
	\textbf{Definition 1.37: } A group action on a metric space is called \textit{geometric} if it is properly discontinuous and cobounded.
	
	\vs 
	
	The action of a group on itself by left multiplication induces an action on its Cayley graph. First, we must define what it means for a group to act on a graph. A group $G$ acts on a graph $\Gamma(V;E)$ if it acts on the vertex set $V$ and if $g \cdot (x,y)$ is the edge $(g\cdot x, g \cdot y)$ in $E$ for any $g \in G$ and any $(x,y) \in E$. Now defining the action of $G$ on the Cayley graph $\Gamma(G, S)$, as: $\forall g \in G$, $g \cdot h = gh$ $\forall h \in G$ and for any edge $(x,y)$, $g \cdot (x,y) := (g \cdot x, g \cdot y) = (gx, gy)$. Note that if $(x,y)$ is an edge in $\Gamma(G;S)$ then so is $(g \cdot x, g \cdot y)$ because $gy = gxs$ for some $s \in S$ as $y = xs$, so this is indeed an action. It turns out that this action has some nice properties. 
	
	\vs 
	
	\textbf{Theorem 1.38: } The natural action of a finitely generated group $G$ on its Cayley graph $\Gamma(G;S)$ ($\vert S \vert < \infty$) is geometric. 
	
	\begin{proof}
		
		We must show the action is properly discontinuous and cobounded. 
		
		\vs 
		
		Properly discontinuous: We note that, as $S$ is finite, $\Gamma(G;S)$ is locally finite, meaning that $B_r(x)$ is finite for any $x \in \Gamma(G;S)$ and any $r\geq 0$. In addition, because the action by left multiplication of $G$ on itself is a free action, it follows that the action on vertices is an injective map for every $g \in G$, which implies that the action is properly discontinous on vertices. Now if $x \in \Gamma(G;S)$ is not a vertex, then $x \in (a,b) \implies g \cdot x \in (g \cdot a, g \cdot b)$. But $g \cdot a$ and $g \cdot b$ are in any $B_r(x_0)$ for only finitely many $g \in G$, so $g \cdot x$ is also in $B_r(x_0)$ for only finitely many $g \in G$, as $d(g \cdot x, g \cdot a), d(g \cdot x, g \cdot b) \leq 1$.
		
		\vs 
		
		Cobounded: We note that $V(\Gamma(G;S)) = G \cdot \{e\}$ and every point in $\Gamma(G;S)$ is within a distance of $\frac{1}{2}$ from a vertex, so $\Gamma(G;S) = G \cdot B_{\frac{1}{2}}(e)$
		
	\end{proof}
	
	\newpage
	\section{Relatively hyperbolic groups}
	\subsection{Definitions and basic notions}
	
	We now have all of the tools needed to define and discuss relatively hyperbolic groups. Let $G$ be a finitely generated group with finite generating set $X$ and let $(H_{\lambda})_{\lambda \in \Lambda}$ be a collection of subgroups of $G$. We call the subgroups $H_{\lambda}$ in the collection \textit{parabolic subgroups}. In the Cayley graph $\Gamma(G;X)$, for each left coset $gH_{\lambda}$, we add a vertex $v(gH_{\lambda})$ and we add edges of length $\frac{1}{2}$ connecting each vertex in $gH_{\lambda}$ to $v(gH_{\lambda})$. The resulting graph that we obtain from this operation is called the \textit{coned-off Cayley graph} and is denoted $\hat{\Gamma}(G;X)$.
	
	
	\vs 
\begin{figure} [H]
	\centering
	\includegraphics[width=0.7\linewidth]{"../Desktop/Honours Project/IMG_1448"}
	\caption{The coned-off Cayley graph}
	\label{fig:img1448}
\end{figure}
	
	\textbf{Definition 2.1: } A group $G$ as above is called \textit{weakly hyperbolic} relative to the collection of subgroups $\{H_{\lambda}\}_{\lambda \in \Lambda}$ if $\hat{\Gamma}(G;X)$ is a hyperbolic metric space. 
	
	\vs 
	
	We need one more concept in order to define hyperbolicity relative to a collection of subgroups. We let $\mathcal{H} = \sqcup_{\lambda \in \Lambda} H_{\lambda} \setminus \{1\}$. We consider the Cayley graph $\Gamma(G; X \sqcup \mathcal{H})$, called the \textit{relative Cayley graph}. We take disjoint unions rather than ordinary unions because we wish to distinguish $X$ and each $H_{\lambda}$ as separate alphabets.
	
	\vs 
	
	\textbf{Definition 2.2: } Let $\alpha$ be a path in $\Gamma(G; X \sqcup \mathcal{H})$. An $H_{\lambda}$-\textit{component} of $\alpha$ is a subpath of $\alpha$ labeled by elements of $H_{\lambda}$. A \textit{parabolic component} of $\alpha$ is an $H_{\lambda}$ compoenent for some $\lambda$. Two $H_{\lambda}$ components are called \textit{connected} if there is a path in $\Gamma(G; X \sqcup \mathcal{H})$ labeled by elements of $H_{\lambda}$ joining vertices of the two components. An $H_{\lambda}$ component is called \textit{isolated} in $\alpha$ if it is not connected to any other $H_{\lambda}$ component in $\alpha$. A path is said to have no \textit{backtracking} if all of its parabolic components are isolated. A path is called \textit{irreducible} if it contains no subpaths of the form $ee^{-1}$ (here, $e^{-1}$ is the path $e$ traveled in reverse).
	
\begin{figure} [H]
	\centering
	\includegraphics[width=0.7\linewidth]{"../Desktop/Honours Project/IMG_1449"}
	\caption{A parabolic component $\eta$ in the path $\alpha$}
	\label{fig:img1449}
\end{figure}

\begin{figure} [H]
	\centering
	\includegraphics[width=0.7\linewidth]{"../Desktop/Honours Project/IMG_1450"}
	\caption{Connected parabolic components}
	\label{fig:img1450}
\end{figure}

	
	\vs 
	
	\textbf{Definition 2.3: } The pair $(G, \{H_{\lambda}\}_{\lambda \in \Lambda})$ is said to satisfy the \textit{Bounded Coset Penetration Property (BCP)} if for every $\lambda \geq 1, \varepsilon \geq 0$, $\exists C = C(\lambda, \varepsilon)$ such that given two $(\lambda, \varepsilon)$-quasi-geodesic paths $\alpha, \beta$ in $\Gamma(G; X \cup \mathcal{H})$ with the same endpoints and given any isolated component $\eta$ of the closed path $\alpha\beta^{-1}$, we have $d_X(\eta_{-}, \eta_{+}) \leq C$ ($d_X$ denotes the metric on $\Gamma(G;X)$, as we explain below Definition 2.4). Intuitively, the BCP says that isolated components of closed paths made up of two quasi-geodesics (i.e. quasi-geodesic bigons) have uniformly bounded $X$-distance between its endpoints. 
	
\begin{figure} [H]
	\centering
	\includegraphics[width=0.7\linewidth]{"../Desktop/Honours Project/IMG_1451"}
	\caption{The bounded coset penetration property (BCP)}
	\label{fig:img1451}
\end{figure}
	
	\vs 
	
	\textbf{Definition 2.4: } A group $G$ is said to be hyperbolic relative to $\{H_{\lambda}\}_{\lambda \in \Lambda}$ if it is weakly hyperbolic relative to this collection and $(G, \{H_{\lambda}\}_{\lambda \in \Lambda})$ satisfies BCP. 
	
	\vs 
	
	Relatively hyperbolic groups have two metrics associated with them, one for each Cayley graph. On the non-relative Cayley graph $\Gamma(G;X)$ we have the \textit{non-relative metric} $d_X$ and on the relative Cayley graph $\Gamma(G; X \sqcup \mathcal{H})$ we have the \textit{relative metric} $d_{X \cup \mathcal{H}}$. Observe that since $X \subseteq X \cup \mathcal{H}$, we have $d_{X \cup \mathcal{H}} \leq d_X$.
	
	\vs 
	
	Instead of the coned-off Cayley graph in the definition of a relatively hyperbolic group, we can use the relative Cayley graph in its place, as the coned-off Cayley graph and the relative Cayley graph are quasi-isometric. Indeed, since the vertex sets of each $\Gamma(G; X \sqcup \mathcal{H})$ and of $\hat{\Gamma}(G;X)$ are quasi-isometric to their respective graphs (see the fourth example in Example 1.17), it suffices to show that $V(\Gamma(G; X \sqcup \mathcal{H}))$ and $V(\hat{\Gamma}(G;X))$ are quasi-isometric. We claim that the inclusion $\iota: V(\Gamma(G; X \sqcup \mathcal{H})) \rightarrow V(\hat{\Gamma}(G;X))$ is a quasi-isometry. Indeed, the inclusion is an isometric embedding and every vertex in $V(\hat{\Gamma}(G;X))$ is a distance of at most 1/2 from a vertex in $V(\Gamma(G; X \sqcup \mathcal{H}))$, so $V(\hat{\Gamma}(G;X)) = N_{1/2} (V(\Gamma(G; X \sqcup \mathcal{H})))$. Therefore, $\iota$ is a quasi-isometry. 
	
	\vs
	
	Before studying the properties of relatively hyperbolic groups, we first give some examples of these objects. 
	
	\vs 
	
	\textbf{Example 2.5 (Examples and non-examples of relatively hyperbolic groups): } 
	
	\begin{itemize}
		\item Consider a finitely generated group $G$ which is hyperbolic relative the collection consisting of only the trivial subgroup. Then $G$ is a hyperbolic group. Indeed, the left cosets of the trivial subgroup are just singletons consisting of each group element. So, the coned-off Cayley graph simply consists of an additional vertex attached to each vertex of the original Cayley graph via an edge of length $\frac{1}{2}$. Therefore, we see that the coned-off Cayley graph $\hat{\Gamma}(G; X)$ (where $X$ is a finite generating set) is quasi-isometric to $\Gamma(G;X)$. As the cone off $\hat{\Gamma}(G; X)$ is a hyperbolic metric space and hyperbolicity of metric spaces is preserved by quasi-isometry by Theorem 1.26, it follows that $\Gamma(G;X)$ is hyperbolic and hence $G$ is a hyperbolic group. 
		\item Finite groups and finitely generated free groups are relatively hyperbolic groups. Indeed, both of these types of groups are hyperbolic (for finite groups, their Cayley graph with respect to any generating set is a bounded metric space (as there are only finitely many vertices), which is hyperbolic as discussed below Definition 1.20 and for free groups their Cayley graph with respect to their free basis is a tree, which is also hyperbolic as discussed below Definition 1.20). Therefore, by the poiny above, both of these types of groups are relatively hyperbolic (with respect to the trivial subgroup). 
		\item Another example of a hyperbolic group (and hence, of a relatively hyperbolic group) arises from free products. It turns out that the free product $G * H$ of two groups $G,H$ is hyperbolic if and only if both $G, H$ are hyperbolic. We will prove this when we relate relative hyperbolicity to Dehn functions later on. 
		\item Consider the free abelian group $G = \langle a,b \vert [a,b] \rangle$ and subgroup $H = \langle a \rangle$. Then $G$ is weakly hyperbolic relative to $H$. Indeed, the map $f: \mathbb{Z} \rightarrow V(\Gamma(G; \{a,b\} \sqcup \langle a \rangle))$ given by $f(n) = b^n$ is a quasi isometry since it is a (1,0)-quasi-isometric embedding as $d_{\{a,b\} \cup H} (f(n),f(m)) = d_{\{a,b\} \cup H} (b^n,b^m) = \vert n - m \vert$ for all $n,m \in \mathbb{Z}$. In addition, $G \subseteq N_1(f(\mathbb{Z}))$ since for any $n,m \in \mathbb{Z}$, $d_{\{a,b\} \cup H}(a^nb^m, b^m) = 1$. Therefore, $f$ is a quasi-isometry. However, the pair $(G,H)$ with the generating set $\{a,b\}$ does not satisfy the BCP because we may produce a geodesic bigon as in the picture below and the isolated $H$-component $\eta$ in the lower geodesic portion of the bigon has $d_{\{a,b\}} (\eta_{-}, \eta_{+}) = n \rightarrow \infty$. This implies that $G$ is not hyperbolic relative to $H$ (as relative hyperbolicity is an invariant of finite generating set, see Lemma 2.8 for more details). 
		\item If $H$ is a hyperbolic group, then $\mathbb{Z}^2 * H$ is hyperbolic relative to $\mathbb{Z}^2$, but is not hyperbolic (which shows that groups hyperbolic relative to proper, non-trivial subgroups form a strictly larger class than the class of hyperbolic groups). Indeed, note that $\mathbb{Z}^2$ is not a hyperbolic group because its Cayley graph using its free basis is the lattice $\mathbb{Z}^2$ in $\mathbb{R}^2$ (see Figure 25), which is quasi-isometric to $\mathbb{R}^2$ by Example 1.17 and $\mathbb{R}^2$ is not hyperbolic as discussed below Definition 1.21. Therefore, $\mathbb{Z}^2 * H$ is not hyperbolic by the above example. To show that $H$ is hyperbolic relative to $\mathbb{Z}^2$, we shall also use relative Dehn functions (see the next section). 
		
\begin{figure} [H]
	\centering
	\includegraphics[width=0.7\linewidth]{"../Desktop/Honours Project/IMG_1454"}
	\caption{The parabolic component $\eta$ from the second example in Example 2.5}
	\label{fig:img1454}
\end{figure}
		
		\item Every finitely generated group $G$ is weakly hyperbolic relative to $\{G\}$. Indeed, let $X$ be a finite generating set for $G$. We see that the relative Cayley graph $\Gamma(G; G \sqcup X)$ is a bounded metric space (it has diameter 1 as we may join any two vertices labeled by elements $g,h \in G$ by an edge labeled $g^{-1}h \in X \sqcup G$), so it is hyperbolic. The BCP property also holds. Suppose we have a $(\lambda, \varepsilon)$-quasi-geodesic bigon $c = \alpha \beta^{-1}$ in $\Gamma(G; X \sqcup G)$. Then a component $\eta$ of $c$ is isolated iff it is the only component in $c$ (because if there were another component of $c$, then this would be connected to the first component because any two vertices in $\Gamma(G; X \sqcup G)$ are connected by an edge labeled by an element of $G$). However, note that we have $\ell(\alpha) \leq \lambda d_{X \cup G}(\alpha_{-}, \alpha_{+}) + \varepsilon \leq \lambda + \varepsilon $ because $d_{X \cup G}(\alpha_{-}, \alpha_{+}) \leq 1$ and similarly $\ell(\beta) \leq \lambda + \varepsilon$. Therefore, by the triangle inequality, we have $d_X(\eta_-, \eta_+) \leq d_X(\alpha_-, \eta_-) + d_X(\alpha_-, \alpha_+) + d_X(\eta_+, \alpha_+) \leq \ell([\alpha_-, \eta_-]_{\beta}) + \ell([\alpha_-, \alpha_+]) + \ell([\eta_+, \alpha_+]_{\beta}) \leq 3(\lambda + \varepsilon)$. Therefore, the BCP property holds for $C(\lambda, \varepsilon) = 3(\lambda + \varepsilon)$. Thus, we conclude that any group $G$ is hyperbolic relative to itself. 
	\end{itemize}

	Our next task will be to define the notion of a \textit{relative presentations} and show how relatively hyperbolic groups can be described by relative presentations.
	
	\vs
	
	Let $G$ be a group and let $\{H_{\lambda}\}_{\lambda \in \Lambda}$ be a collection of subgroups. We say that $G$ is generated by $X \subseteq G$ with respect to $\{H_{\lambda}\}_{\lambda \in \Lambda}$ if $G$ is generated by $X \cup (\cup_{\lambda \in \Lambda} H_{\lambda})$. We call such $X$ a \textit{relative generating set}. Note that every generating set is a relative generating set. In this case we can write $G \cong ((*_{\lambda \in \Lambda} H_{\lambda}) * F(X))/\langle \langle R \rangle \rangle$ where $R$ is some subset of the free product $F := (*_{\lambda \in \Lambda} H_{\lambda}) * F(X))$. We let $\mathcal{H} = \sqcup_{\lambda \in \Lambda} H_{\lambda} \setminus \{1\}$, so that $F$ is generated by $X \cup \mathcal{H}$.
	
	\vs 
	
	If $S_{\lambda}$ denotes a set of defining relators in $H_{\lambda}$ then a set of defining relators of $F$ is $\cup_{\lambda} S_{\lambda}$. Hence, a set of defining relators for $G$ is $R \cup (\cup_{\lambda} S_{\lambda})$ and so a presentation for $G$ is $G = \langle X \cup \mathcal{H} \vert R \cup S \rangle$, where $S = \cup_{\lambda} S_{\lambda}$. 
	
	\vs 
	
	\textbf{Definition 2.6: } With the notation as above, a relative presentation for the group $G$ is written as $G = \langle X, \mathcal{H} \vert R, S \rangle$. For brevity, we will often surpress this and write $G = \langle X, H_{\lambda}, \lambda \in \Lambda \vert R \rangle$. 
	
	\vs 
	
	\textbf{Definition 2.7: } We say that a relative presentation is \textit{finite} if $X,R$ are both finite. If there exists a finite relative presentation of a group $G$ relative to the collection $\{H_{\lambda}\}_{\lambda \in \Lambda}$ then we say that $G$ is \textit{finitely presented} relative to $\{H_{\lambda}\}_{\lambda \in \Lambda}$. 
	
	\vs 
	
	For each relative presentation $\langle X, H_{\lambda}, \lambda \in \Lambda \vert R \rangle$, we have a metric $d_{X \cup \mathcal{H}}$ on the relative Cayley graph $\Gamma(G; X \sqcup \mathcal{H})$, as we discussed below Definition 2.4. It turns out that for two finite relative generating sets $X_1, X_2$, the relative Cayley graphs $\Gamma(G; X_1 \sqcup \mathcal{H})$ and $\Gamma(G; X_2 \sqcup \mathcal{H})$ are quasi-isometric. To show this, it suffices to show that the group $G$ equipped with these two metrics are quasi-isometric spaces, since the $G$ as a metric space is quasi-isometric to its Cayley graph (see the fourth example in Example 1.17). 
	
	\vs 
	
	\textbf{Lemma 2.8: } Let $X_1, X_2$ be two finite relative generating sets (relative to a collection of subgroups $\{H_{\lambda}\}_{\lambda \in \Lambda}$) of a group $G$. Then $(G, d_{X_1 \cup \mathcal{H}})$ is bi-Lipschitz equivalent to $(G, d_{X_2 \cup \mathcal{H}})$. 
	
	\begin{proof}
		
		First, we may assume that $X_1, X_2 \neq \emptyset$, as if both $X_1, X_2 = \emptyset$, then $d_{X_1 \cup \mathcal{H}} = d_{\mathcal{H}} = d_{X_2 \cup \mathcal{H}}$. If $X_1 = \emptyset$ and $X_2 \neq \emptyset$ then letting $M = \max \{\vert x \vert_{\mathcal{H}}: x \in X_2\} + 1$, we have $d_{X_2 \cup \mathcal{H}} \leq d_{\mathcal{H}} \leq M d_{\mathcal{H}}   = M d_{X_1 \cup \mathcal{H}}$ and $d_{\mathcal{H}} \leq M d_{X_2 \cup \mathcal{H}}$ (writing a word over $X_2 \cup \mathcal{H}$ and replacing each letter by a word over $\mathcal{H}$). An analogous argument holds when $X_2 = \emptyset$ and $X_1 \neq \emptyset$. 
		
		\vs
		
		Now suppose $X_1, X_2 \neq \emptyset$. Let $M_1 = \max \{\vert x \vert_{X_2 \cup \mathcal{H}} : x \in X_1\} + 1$ and $M_2  = \max \{\vert x \vert_{X_1 \cup \mathcal{H}}: x \in X_2\} + 1$. Let $M = \max \{M_1, M_2\}$. Then for any $g \in G$, writing $g$ as a word in $X_1 \cup \mathcal{H}$ and replacing each letter from $X_1$ by a word in $X_2 \cup \mathcal{H}$, we obtain a word over $X_2 \cup \mathcal{H}$ of length at most $M_1$, so $\vert g \vert_{X_2 \cup \mathcal{H}} \leq M_1 \vert g \vert_{X_1 \cup \mathcal{H}} \leq M \vert g \vert_{X_1 \cup \mathcal{H}}$. Symmetrically, we obtain $\vert g \vert_{X_1 \cup \mathcal{H}} \leq M \vert g \vert_{X_2 \cup \mathcal{H}}$.
		
		\vs
		
		 Thus, for any $g, h \in G$, we have $\frac{1}{M} d_{X_1 \cup \mathcal{H}}(g,h) = \frac{1}{M} \vert g^{-1}h \vert_{X_1 \cup \mathcal{H}} \leq \vert g^{-1}h \vert_{X_2 \cup \mathcal{H}} = d_{X_2 \cup \mathcal{H}}(g,h) \leq M \vert g^{-1}h \vert_{X_1 \cup \mathcal{H}} = M d_{X_1 \cup \mathcal{H}}$. Therefore, $id_G$ is a bi-Lipschitz equivalence between $(G, d_{X_1 \cup \mathcal{H}})$ and $(G, d_{X_2 \cup \mathcal{H}})$. 
		
	\end{proof}

	\vs 
	
	The above lemma implies that relative hyperbolicity is independent of finite generating set, as relative Cayley graphs for two different finite generating sets are quasi-isometric and if the BCP property holds with respect to one finite generating set, then it also holds with respect to any other, since (non-relative) metrics with respect to finite generating sets are quasi-isometric (taking the collection of subgroups to be empty in Lemma 2.8).
	
	\vs 
	
	Following Osin [16], we now show that a finitely generated group $G$ generated by a finite set $X$ and hyperbolic relative to a finite collection of subgroups $\{H_1,...,H_m\}$ admits a finite relative presentation. First, we will require the following piece of terminology. Below we fix a hyperbolicity constant $\delta$ for $\Gamma(G; X \sqcup \mathcal{H})$. 
	
	\vs
	
	\textbf{Definition 2.9: } We say that a cycle $p$ in $\Gamma(G; X \sqcup \mathcal{H})$ is \textit{atomic} if every subpath $q$ of $p$ with $\ell(q) \leq \frac{1}{2} \ell(p)$ is a geodesic. 
	
	\vs
	
	Atomic cycles have bounded length and have all of their $H_i$ components isolated, as the following Lemma shows: 
	
	\vs 
	
	\textbf{Lemma 2.10: } Let $p$ be an atomic cycle in $\Gamma(G; X \sqcup \mathcal{H})$. We then have: 
	
	\begin{enumerate}[label = (\roman*)]
		\item $\ell(p) \leq 4 \delta + 9$
		\item Every $H_i$ component of $p$ is isolated in $p$. 
	\end{enumerate}

	\begin{proof}
		
		(i): Suppose for contradiction that $\ell(p) > 4 \delta + 9$. We decompose $p$ into 3 subpaths: $p = p_1 p_2 p_3$, where $\ell(p_1) = \ell(p_2)$ and $\ell(p_3) \leq 1$. We then see that $\ell(p_1),\ell(p_2) \leq \frac{1}{2}\ell(p)$, and so $p_1, p_2$ are geodesics, since $p$ is atomic. Note also that $p_3$ is a geodesic as it is either an edge or a point. Therefore, $p_1,p_2,p_3$ form a geodesic triangle. Since $2 \ell(p_1) + \ell(p_3) = \ell(p) > 4 \delta + 9$ and $\ell(p_3) \leq 1$, we have $\ell(p_1) > 2 \delta + 4$. Let $w$ be the midpoint of $p_1$ and let $z$ be a point on $p_2 \cup p_3$ such that $d_{X \cup \mathcal{H}}(z, w) \leq \delta$ (such a $z$ exists since $\Gamma(G; X \sqcup \mathcal{H})$ is $\delta$-hyperbolic). Note that $w$ may not be a vertex, so we choose a vertex $v$ on $p_1$ which is closest to $w$ and similarly we choose a vertex $u$ on $p_2 \cup p_3$ which is closest to $z$. By the triangle inequality, we then have $d_{X \cup \mathcal{H}}(u,v) \leq d_{X \cup \mathcal{H}}(u,w) + d_{X \cup \mathcal{H}}(w, v) \leq 1 + \delta$. Note that one of the subsegments $[u,v]_p$ and $[v,u]_p$ (i.e. $[v,u]_p$ goes from $v$ to $u$ along $p$, while $[u,v]_p$ goes from $u$ to $v$ along $p$, so that $p = [u,v]_p \cup [v,u]_p$) is not a geodesic, as $\ell([v,u]_p), \ell([u,v]_p) > \frac{1}{2} \ell(p_1) - \frac{1}{2} > \delta + \frac{3}{2} > \delta + 1 = d_{X \cup \mathcal{H}}(u,v)$. But also one of $[u,v]_p$, $[v,u]_p$ has length at most $\frac{1}{2} \ell(p)$ because $p = [u,v]_p \cup [v,u]_p$, so that one of $[u,v]_p, [v,u]_p$ is a geodesic, a contradiction. Therefore, we conclude that $\ell(p) \leq 4 \delta + 9$. 
		
\begin{figure} [H]
	\centering
	\includegraphics[width=0.7\linewidth]{"../Desktop/Honours Project/IMG_1527"}
	\caption{The proof of Lemma 2.10 (i)}
	\label{fig:img1527}
\end{figure}
		
		\vs 
		
		(ii): Suppose $p$ had two connected $H_i$ components $a,b$. Then we can decompose $p$ as $p = a r b s$ for subpaths $r,s$ (see picture below). But then neither $ar$ nor $bs$ is a geodesic because $d_{X \cup \mathcal{H}}(a_-, r_+) = 1$ (as $a_-$ and $r_+$ are connected by an element of $H_i$), but $\ell(ar) \geq 2$, as neither $a$ nor $s$ are trivial, therefore $ar$ is not a geodesic, and similarly, $d_{X \cup \mathcal{H}}(b_-, s_+) = 1$ while $\ell(bs) \geq 2$, so $bs$ is not a geodesic. However, this is a contradiction because $(ar)(bs) = p$, so one of $ar$ or $bs$ must have length at most $\frac{1}{2} \ell(p)$, and hence at least one of $ar, bs$ must be a geodesic since $p$ is atomic. We therefore conclude that $p$ has no connected components.
		
\begin{figure} [H]
	\centering
	\includegraphics[width=0.7\linewidth]{"../Desktop/Honours Project/IMG_1528"}
	\caption{Connected components in the proof of Lemma 2.10 (ii)}
	\label{fig:img1528}
\end{figure}
		
	\end{proof}

	We are going to use atomic cycles to define a finite relative presentation for $G$. 
	
	\vs 
	
	\textbf{Corollary 2.11: } Let $\mathcal{A}$ denote the set of labels of atomic cycles in $\Gamma(G; X \sqcup \mathcal{H})$. Then $\mathcal{A}$ is finite. 
	
	\begin{proof}
		
		Let $p$ be an atomic cycle in $\Gamma(G; X \sqcup \mathcal{H})$. We decompose $p$ into two subpaths as $p = cd$ where $\ell(c) \leq \ell(d) \leq \ell(c) + 1$. We then have $\ell(c) \leq \frac{1}{2} \ell(p)$, so $c$ is a geodesic, as $p$ is atomic. We also have $\ell(d) \leq \frac{1}{2} \ell (p) + 1$, so either $d$ has length $\frac{1}{2} \ell (p)$ in which case $d$ is a geodesic or $d$ is the concatenation of a path of length $\frac{1}{2} \ell(p)$ (hence a geodesic) and an edge, in which case $d$ is a $(1,2)$-quasi geodesic by Lemma 1.19. Therefore, in either case $d$ is a $(1,2)$ quasi-geodesic. By Lemma 2.10(ii), all $H_i$ components of $p$ are isolated, so $c,d$ are both paths without backtracking. Therefore, by BCP, every component of $p$ has $X$-length bounded above by $C(1,2)$. Therefore, writing the label of each component of $p$ as a word in $X$ to obtain the label of $p$, $\phi(p)$ as a word over $X$, since $\ell(p) \leq 4 \delta + 9$, we have $\vert \vert \phi(p) \vert \vert_X \leq (4 \delta + 9) C(1,2)$. Therefore, $p$ can only be labeled by words over $X$ of length at most $(4 \delta + 9) C(1,2)$. Since $X$ is finite, this implies that there are only finitely many choices for labels of $p$. Hence $\mathcal{A}$ is finite. 
	\end{proof}

	From the above set $\mathcal{A}$, we obtain that: 
	
	\vs 
	
	\textbf{Theorem 2.12: } $\langle X, H_i, i \in \{1,...,m\} \vert \mathcal{A} \rangle$ is a relative presentation for $G$. 
	
	\vs 
	
	The proof of the above requires an excursion into van Kampen diagrams which we will cover in the next section. Therefore, we postpone the proof of this theorem until the next section, when we will cover van Kampen diagrams. 
	
	\vs 
	
	We introduce some additional notation and terminology before proceeding to the next section. 
	
	\vs 
	
	Given a word $w \in (X \cup \mathcal{H})^*$, we denote $\pi(w)$ to be the element in $G$ represented by $w$. An $H_{\lambda}$ \textit{subword} of $w$ is a subword of $w$ consisting only of letters from $H_{\lambda}$. An $H_{\lambda}$-\textit{syllable} is a maximal $H_{\lambda}$ subword. A relative presentation $\langle X, H_{\lambda}, \lambda \in \Lambda \vert R \rangle $ is called \textit{reduced} if every relator $r \in R$ has the shortest possible length of any word in $(X \cup \mathcal{H})^*$ representing the same element in $F$. This is equivalent to every $H_{\lambda}$-syllable consisting of a single letter. 
	
	\vs 
	
	We define special subsets of $G$ that will capture to what extend elements of parabolic subgroups appear as syllables in relators.
	
	\vs 
	
	\textbf{Definition 2.13: } For each $\lambda \in \Lambda$, define $\Omega_{\lambda}$ to be the subset of $H_{\lambda}$ containing those elements $g \in H_{\lambda}$ such that $\exists r \in R$ and an $H_{\lambda}$-syllable $v$ of $r$ such that $\pi(v) = g$. In addition, put $\Omega = \cup_{\lambda} \Omega_{\lambda}$. 
	
	\vs
	
	We note that if $R$ is finite, then so is each $\Omega_{\lambda}$. Indeed,
	
	\vs
		
		\begin{align*}
		\Omega_{\lambda} &= \{g \in H_{\lambda}: g = \pi(v) \text{ for some } v \text{ } H_{\lambda} \text{ syllable of some } r \in R \} \\
		&= \{\pi(v): v \text{ is an } H_{\lambda} \text{ syllable of some } r \in R\} \cap H_{\lambda}
		\end{align*}
		
		but if $R$ is finite, then there are only finitely many $H_{\lambda}$-syllables of words in $R$ since each $H_{\lambda}$ syllable of a word $r \in R$ is a subword $r$ and there are only finitely many subwords of words from $r$ since each word has finitely many letters and there are only finitely many words in $R$; thus, $\{\pi(v): v \text{ is an } H_{\lambda} \text{ syllable of some } r \in R\}$ is a finite set and so $\Omega_{\lambda}$ is finite.
	

	
	\vs 

\newpage
	\subsection{Combinatorics of Relatively Hyperbolic Groups}
	
	One of the key things that makes relatively hyperbolic groups particularly well-suited for the study of algorithmic problems is their rich geometric structure. We study some of their geometry here and later on apply our results to algorithmic problems. 
	
	\vs 
	
	\underline{van Kampen diagrams and Dehn functions}
	
	\vs 
	
	A geometric-combinatorial construction that will be useful to us when we discuss algorithmic problems in relatively hyperbolic groups are \textit{van Kampen} diagrams over a given presentation. 
	
	\vs 
	
	First, we introduce some terminology from algebraic topology. We assume the reader to be familiar with basic notions of point set topology. An \textit{n-cell} is a topological space which is homeomorphic to the $n$-unit disk $D^n = \{x \in \mathbb{R}^{n+1}: \vert x \vert \leq 1\}$. An \textit{n-complex} consists of a disjoint union of $k$-cells for $k \leq n$ (this disjoint union is called the $k$-skeleton of the complex) and an attaching map that glues the boundary of every $k$ cell to the $k-1$ skeleton ($k \geq 1$). An $n$-complex is \textit{finite} if the set of all cells is finite. 
	
\begin{figure} [H]
	\centering
	\includegraphics[width=0.5\linewidth]{"../Desktop/Honours Project/IMG_1455"}
	\caption{An example of a cell complex}
	\label{fig:img1455}
\end{figure}
	
	\vs 
	
	Let $G$ be a group with finite presentation $\langle X \vert R \rangle$. We assume that $X$ is symmetrized, that is, $X = X^{-1}$. A \textit{planar map} $\Delta$ is a finite, oriented, connected, simply connected (meaning that every loop in $\Delta$ can be continuously shrunk to a point) 2-complex in which each edge (1-complex) $e$ has label $\phi(e) \in X$ and $\phi(e^{-1}) = \phi(e)^{-1}$. 
	
	\vs
	
	A \textit{van Kampen diagram} over the presentation $\langle X \vert R \rangle$ is a planar map $\Delta$ such that for every 2-cell $\Pi$ of $\Delta$ has $\phi(\partial \Pi) = r^{\pm}$ where $r \in R$ (that is, the boundary of every 2-cell is labeled by a relator or its inverse). 
	
	\vs 
	
	van Kampen diagrams are particularly useful when studying words representing 1 in $G$, as the following lemma illustrates: 
	
	\vs 
	
	\textbf{Lemma 2.14 (van Kampen Lemma):} Let a group $G$ have finite presentation $\langle X \vert R \rangle$. Then $w = 1$ in $G$ if and only if there exists a van Kampen diagram $\Delta$ over $\langle X \vert R \rangle$ such that $\phi(\partial \Delta) = w$.
	
	\begin{proof}
		
		We provide a sketch of the proof, using pictures to aid our arguments. 
		
		\vs
		
		Suppose that there exists a van Kampen diagram $\Delta$ over $\langle X \vert R \rangle$ with $\phi(\partial \Delta) = w$. We begin by deleting one of the boundary regions of $\Delta$ to obtain a new diagram $\Delta'$ with boundary label $w'$. We observe that the deleted region has boundary label $frf^{-1}$ for some $f \in F(X)$, $r \in R$. Therefore, $w = frf^{-1} w'$. We repeat this procedure until all boundary regions of $\Delta$ are removed, so that the boundary label at the end of this process is the empty word. We then obtain $w = f_1 r_1 f_1^{-1} f_2 r_2 f_2^{-1}...f_n r_n f_n^{-1}$ for some $f_i \in F(X)$, $r_i \in R$. We therefore see that $w \in \langle \langle R \rangle \rangle$ in $F(X)$, so $w = 1$ in $G$. 
		
\begin{figure} [H]
	\centering
	\includegraphics[width=0.7\linewidth]{"../Desktop/Honours Project/IMG_1456"}
	\caption{Deleting boundary regions in the proof of van Kampen Lemma}
	\label{fig:img1456}
\end{figure}
		
		\vs 
		
		Conversely, suppose that $w$ is a freely reduced word over $X$ such that $w = 1$ in $G$. Then $w = \Pi_{i=1}^n f_i r_i f_i^{-1}$ for some $f_i \in F(X)$, $r_i \in R$ (i.e. $w \in \langle \langle R \rangle \rangle$). We form a van Kampen diagram $\Delta$ over $\langle X \vert R \rangle$ having boundary label $\Pi_{i=1}^n f_i r_i f_i^{-1}$ (with the $f_i$, $r_i$ written as words over $X$) as shown below. Though $w = \Pi_{i=1}^n f_i r_i f_i^{-1}$ in $F(X)$, $\Pi_{i=1}^n f_i r_i f_i^{-1}$ may not be freely reduced when we write the $f_i$ and $r_i$ as words over $X$. We then perform a sequence of folding moves to $\Delta$ to obtain new diagrams whose labels correspond to reduction operations that we apply to $\Pi_{i=1}^n f_i r_i f_i^{-1}$. As it only takes finitely many reduction operations to freely reduce $\Pi_{i=1}^n f_i r_i f_i^{-1}$ to $w$, there is a resulting finite sequence of folding operations that we apply to $\Delta$ to obtain a new diagram $\Delta'$ with $\phi(\partial \Delta') = w$. 
		
\begin{figure} [H]
	\centering
	\includegraphics[width=0.5\linewidth]{"../Desktop/Honours Project/IMG_1457"}
	\caption{Folding process in the proof of van Kampen Lemma}
	\label{fig:img1457}
\end{figure}
		
	\end{proof}
	
	When we include relative presentations into the discussion on van Kampen diagrams, we have two types of relators: those "intrinsic" relators coming from relations in the parabolic subgroups and "extrinsic" relators coming from the generating set of the normal closure. Therefore, we can distinguish between two types of cells in a van Kampen diagram. 
	
	\vs 
	
	\textbf{Definition 2.15: } In a van Kampen diagram $\Delta$ over a relative presentation $\langle X, \mathcal{H} \vert R, S \rangle$, a 2-cell $\Pi$ is called an \textit{S-cell} if $\phi(\partial \Pi) \in S$ and an \textit{R-cell} if $\phi(\partial \Pi) \in R$.

	\vs
	
	In a van Kampen diagram, we have an algebro-combinatorial notion of area.
	
	\vs 
	
	\textbf{Definition 2.16: } Given a van Kampen diagram $\Delta$, let $\mathcal{N}_R(\Delta)$ denote the number of $R$-cells in $\Delta$ and let $\mathcal{N}_S(\Delta)$ denote the number of $S$-cells in $\Delta$. Then the \textit{area} of $\Delta$ is defined to be $\text{Area}(\Delta) = \mathcal{N}_R(\Delta) + \mathcal{N}_S(\Delta)$.
	
	\vs 
	
	We may also classify van Kampen diagrams according to their \textit{type}. 
	
	\vs 
	
	\textbf{Definition 2.17: } Given a van Kampen diagram $\Delta$ over a relative presentation $\langle X, \mathcal{H}_{\lambda}, \lambda \in \Lambda \vert R \rangle $, the \textit{type} of $\Delta$ is the triple $(\mathcal{N}_{R}(\Delta), \mathcal{N}_S(\Delta), \vert E(\Delta) \vert)$, where $E(\Delta)$ denotes the set of edges of $\Delta$. 
	
	\vs 
	
	We fix a lexicographic order on the set of all types (recall that the lexicographic (or dictionary) order $\preccurlyeq$ on triples of natural numbers is defined as: $(a,b,c) \preccurlyeq (d,e,f)$ if and only if $a < d$ or ($a = d$ and $b < e$) or ($a = d$ and $b = e$ and $c \leq f$)). A van Kampen diagram is called \textit{minimal} if it has minimal type among all van Kampen diagrams over the same relative presentation. 
	
	\vs 
	
	In a van Kampen diagram, we may distinguish between two types of edges: \textit{internal} and \textit{external}. 
	
	\vs 
	
	\textbf{Definition 2.18: } Let $\Delta$ be a van Kampen diagram over $\langle X, \mathcal{H} \vert R \rangle$. An edge $e$ of $\Delta$ is called \textit{internal} if it is a common edge of two 2-cells in $\Delta$, otherwise $e$ is called \textit{external}. We further differentiate external edges into two types: an external edge $e$ is called external of the \textit{first type} if it is part of the boundary of some 2-cell in $\Delta$, otherwise it is called an external edge of the \textit{second type}. 
	
	\begin{figure} [H]
		\centering
		\includegraphics[width=0.7\linewidth]{"../Desktop/Honours Project/IMG_1458"}
		\caption{Edge types in van Kampen diagrams. $d$ is internal, while the other edges are all external. $e_5$ is the only external edge of second type}
		\label{fig:img1458}
	\end{figure}
	
	\vs 
	
	A fact that will prove useful in our discussion of the word problem is that every internal edge is the edge of some $R$-cell. 
	
	\vs 
	
	\textbf{Lemma 2.19: } Suppose that $\Delta$ is a van Kampen diagram over $\langle X, \mathcal{H} \vert R \rangle$ and is of minimal type. Then every internal edge of $\Delta$ is an edge of some $R$-cell of $\Delta$. 
	
	\begin{proof}
		
		We proceed by contradiction. Let $e \in E(\Delta)$ be an internal edge and suppose that $e$ is not an edge forming the boundary of some $R$-cell. Then there are two possibilities: (i) either $e$ is a common edge of two $S$-cells $\Pi_1$ and $\Pi_2$ or (ii) $e$ connects two portions of a single $S$-cells $\Pi$ (see the figure 37). 
		
		
\begin{figure} [H]
	\centering
	\includegraphics[width=0.7\linewidth]{"../Desktop/Honours Project/IMG_1459"}
	\caption{The two possibilities in the proof of Lemma 2.19}
	\label{fig:img1459}
\end{figure}

		
		\vs 
		
		In the first case, denoting the remainder of $\partial \Pi_1$ by $c_1$ and the remainder of $\partial \Pi_2$ by $c_2$ as in the diagram, we note that $c_1 c_2^{-1}$ is a closed path (i.e. has the same starting and end point) and therefore $\phi(c_1c_2^{-1}) = 1$ in $F$. Hence, we may replace the two cells $\Pi_1, \Pi_2$ by a single $S$-cell with boundary $c_1 c_2^{-1}$. This decreases the type of the diagram (because the number of $R$-cells remains the same when we make this replacement, yet the number of $S$-cells decreases by 1), contradicting the assumption of minimal type of the diagram. 
		
		\begin{figure} [H]
			\centering
			\includegraphics[width=0.7\linewidth]{"../Desktop/Honours Project/IMG_1460"}
			\caption{Replacing the two cells $\Pi_1, \Pi_2$ by a single cell $\pi$}
			\label{fig:img1460}
		\end{figure}
		
		\vs 
		
		In the second case, we note that $d_1$ is a closed loop and therefore we can replace $\Pi$ with a single $S$-cell with boundary $d_1$. This reduces the number of edges in $\Delta$ (while leaving the number of $R$ and $S$-cells the same) and hence reduces the type of the diagram, contradicting the minimality of $\Delta$.
		
\begin{figure} [H]
	\centering
	\includegraphics[width=0.7\linewidth]{"../Desktop/Honours Project/IMG_1461"}
	\caption{Replacing the cell $\Pi$ with the single cell $S$}
	\label{fig:img1461}
\end{figure}
		
	\end{proof}

	From the above, we can deduce a bound on the number of edges in $\Delta$.
	
	\vs
	
	\textbf{Corollary 2.20: } Let $\Delta$ be a van Kampen diagram of minimal type over a finite relative presentation $\langle X, \mathcal{H} \vert R \rangle$. Then $\vert E(\Delta) \vert\leq M \mathcal{N}_R(\Delta) + \ell(\partial \Delta)$, with $M := \max_{r \in R} \vert \vert r \vert \vert$. 
	
	\begin{proof}
		
		Denote $\text{int}(E(\Delta))$ and $\text{ext}(E(\Delta))$ as the set of all internal and external edges of $\Delta$, respectively. Since every edge in $\Delta$ is either an internal or an external edge, we have: 
		
		\vs
		
		$\vert E(\Delta) \vert = \vert \text{int}(E(\Delta)) \vert + \vert \text{ext}(E(\Delta)) \vert$. Now by Lemma 2.19, every internal edge is part of the boundary of an $R$-cell and also every external edge is part of the $\partial \Delta$ so $\vert \text{int}(E(\Delta)) \vert \leq \sum_{A \in \mathcal{N}_R(\Delta)} \ell(\partial A)$ and $\vert \text{ext}(E(\Delta)) \vert \leq \ell(\partial \Delta)$, so $\vert E(\Delta) \vert \leq \sum_{A \in \mathcal{N}_R(\Delta)} \ell(\partial A) + \ell(\partial \Delta)$. In addition, as the boundary of each $R$-cell is labeled by a word from $R$ and $M$ is the maximum length of words from $R$, we have $\ell(\partial A) \leq M$ for each $A \in \mathcal{N}_R(\Delta)$, therefore $\vert E(\Delta) \vert \leq M \mathcal{N}_R(\Delta) + \ell(\partial \Delta)$, as required. 
		
	\end{proof}
	
	\vs
	
	Given a cycle $\alpha$ in a relative Cayley graph $\Gamma(G; X \cup \mathcal{H})$, we define the relative area of $\alpha$, denoted $\text{Area}^{\text{rel}} (\alpha)$ as the minimum of $\mathcal{N}_R(\Delta)$ where $\Delta$ ranges over all van Kampen diagrams over $\langle X \cup \mathcal{H} \vert R \cup S \rangle$ such that $\phi(\partial \Delta) = \phi(q)$. 
	
	\vs 
	
	We now prove an important lemma and proposition that will be of great use to use when we discuss how to solve the word problem. 
	
	\vs 
	
	\textbf{Lemma 2.21: } Let $G$ be a group given by a finite, reduced relative presentation $\langle X, H_{\lambda}, \lambda \in \Lambda \vert R \rangle$ and let $q$ be a cycle in $\Gamma(G, X \sqcup \mathcal{H})$. Let $p_1,...,p_k$ be all of the isolated $H_{\lambda}$ components of $q$. Then $\pi(\phi(p_i)) \in \langle \Omega_{\lambda} \rangle$ $\forall i =1,...,k$ and $\sum_{i=1}^k \vert \pi(\phi(p_i)) \vert_{\Omega_{\lambda}} \leq M \cdot \text{Area}^{\text{rel}}(q)$, where $M = \max_{r \in R} \vert \vert r \vert \vert$.
	
	\begin{proof}
		
		By relabeling the $p_i$ as necessary, we may assume that $q$ decomposes as $q = p_1 q_1 ... p_k q_k$ for paths $q_i$ in $\Gamma(G; X \sqcup \mathcal{H})$. Denote $Q$ as the set of all irreducible cycles $q'$ with the same end points as $q$. We can therefore write $q' = p_1'q_1...p_k' q_k$ for $p_i'$ being $H_{\lambda}$ components of $q'$ such that $(p_i')_{\pm} = (p_i)_{\pm}$ (so that therefore $\pi(\phi(p_i')) = \pi(\phi(p_i))$). We also note that the $p_i'$ are isolated in $q'$, as follows from the $p_i$ being isolated in $q$. We denote $\mathcal{D}$ as the set of all van Kampen diagrams $\Delta$ such that $\phi(\partial \Delta) = \phi(q')$ for some $q' \in Q$. 
		
		\vs 
		
		Now let $\Delta \in \mathcal{D}$ be a van Kampen diagram of minimal type. We will show that every edge of every $p_i'$ contained in $\partial \Delta$ is part of the boundary of some $R$-cell. Suppose that this were not the case. Then there are only 2 possibilities: (i) either some $p_i'$ contains an external edge of second type or (ii) some $S$-cell $\Pi$ contains a common edge with some $p_i'$. These cases are illustrated below. In either case, we see that we can reduce the type of $\Delta$, contradicting the minimality of its type. Therefore, we conclude that every edge of each $p_i'$ is contained in the boundary of some $R$-cell. Since $\langle X, \mathcal{H} \vert R \rangle$ is reduced, every $H_{\lambda}$ component of every $r \in R$ has length 1, so every edge of each $p_i'$ is labeled by an element of $\Omega_{\lambda}$, hence $\pi(\phi(p_i')) \in \langle \Omega_{\lambda} \rangle$. 
		
		\vs 
		
		Lastly, since $\Delta$ is minimal, we have $\mathcal{N}_R(\Delta) \leq \text{Area}^{\text{rel}}(q)$. Thus, $\sum_{i=1}^k \vert \pi(\phi(p_i)) \vert_{\Omega_{\lambda}} \leq \sum_{i=1}^k \ell(p_i') \leq M \text{Area}^{\text{rel}}(q)$.
		
\begin{figure} [H]
	\centering
	\includegraphics[width=0.7\linewidth]{"../Desktop/Honours Project/IMG_1462"}
	\caption{Case (i) in the proof of Lemma 2.21}
	\label{fig:img1462}
\end{figure}

\begin{figure} [H]
	\centering
	\includegraphics[width=0.7\linewidth]{"../Desktop/Honours Project/IMG_1463"}
	\caption{Case (ii) in the proof of Lemma 2.21}
	\label{fig:img1463}
\end{figure}

		
	\end{proof}  

	\textbf{Proposition 2.22: } Suppose that $G = \langle X \rangle$ for some finite $X$ and that $G$ is finitely presented with respect to $\{H_{\lambda}\}_{\lambda \in \Lambda}$. Then for each $\lambda$, $H_{\lambda} = \langle \Omega_{\lambda} \rangle$. 
	
	\begin{proof}
		
		Let us write a finite, reduced relative presentation for $G$: $G = \langle X, \mathcal{H} \vert S_{\lambda}, R \rangle$. 
		
		\vs 
		
		Let $\lambda \in \Lambda$ and let $h \in H_{\lambda} \setminus \{1\}$. Since $G$ is generated by $X$, there is a word $w \in X^*$ such that $\pi_X(w) = h$ (where $\pi_X: X^* \rightarrow G$ is the surjective map giving the element in $G$ represented by $w$). Consider the cycle $q = pr$ in $\Gamma(G; X \sqcup \mathcal{H})$ where $\pi(\phi(p)) = h$ and $\phi(r) = w^{-1}$ (with equality in the free monoid $X^*$). Then $p$ is an isolated $H_{\lambda}$ component of $q$ (because $r$ has no $H_{\lambda}$ components at all, being a word in $X$, so there is no $H_{\lambda}$ component of $r$ for $p$ to connect to). Applying Lemma 2.21 we obtain $h \in \langle \Omega_{\lambda} \rangle$. Therefore $H_{\lambda} = \langle \Omega_{\lambda} \rangle$. We note therefore that $H_{\lambda}$ is finitely generated, since $\Omega_{\lambda}$ is finite as follows from the finiteness of $R$ (see above). 
		
	\end{proof}

	We next introduce the concept of a \textit{relative Dehn function} associated with a relative presentation, which along with van Kampen diagrams will be useful to the description of Osin's solution to the word and membership problem. We begin by discussing relative isoperimetric functions. 
	
	\vs 
	
	\textbf{Definition 2.23: } A \textit{relative isoperimetric function} for the relative presentation $\langle X, \mathcal{H} \vert R \rangle$ is a non-decreasing function $f: \mathbb{N} \rightarrow \mathbb{N}$ satisfying the following: For every $n \in \mathbb{N}$ and every word $w \in (X \cup \mathcal{H})^*$ representing 1 in $G$ with $\vert \vert w \vert \vert \leq n$, $\exists k \in \mathbb{N}$ such that $k \leq f(n)$ and $w = \Pi_{i=1}^k f_i r_i f_i^{-1}$ (equality in $F$) for some $r_i \in R$ and $f_i \in F$. 
	
	\vs 
	
	Intuitively, an isoperimetric function is an upper bound for the length of a product of conjugates representing a word equal to 1 in $G$. 
	
	\vs 
	
	\textbf{Definition 2.24: } A \textit{relative Dehn function} of the relative presentation  $\langle X, \mathcal{H} \vert R \rangle$ is the minimum over all relative isoperimetric functions. If the relative presentation $\langle X, \mathcal{H} \vert R \rangle$ does not admit any finite relative isoperimetric function, then we say that the relative Dehn function is not well-defined. The relative Dehn function is not well defined when there are words $w$ over $X \cup \mathcal{H}$ of some bounded length $n$ which require an arbitrarily large number of terms to be written as a product of conjugates of elements of $R$ (i.e. the minimum value of $k$ above can get arbitrarily large). We denote the relative Dehn function by $\delta^{\text{rel}}(n)$.
	
	\vs 
	
	Before proceeding, we stop to complete a proof of the following theorem from last section: 
	
	\vs 
	
	\textbf{Theorem 2.25: } $\langle X, H_i, i \in \{1,...,m\} \vert \mathcal{A} \rangle$ is a relative presentation for $G$. 
	
	\begin{proof}
		
		Given a word $W$ of length at most $n$, it suffices to show that there exists a van Kampen diagram with boundary label $W$ and whose $R$-cells are labeled by words in $\mathcal{A}$, as this will give us that every word representing 1 in $G$ can be written as the product of cyclic conjugates of words in $\mathcal{A}$, and combining with the fact that each word in $\mathcal{A}$ represents 1 in $G$ gives us that $\mathcal{A}$ generates the kernel of the natural epimorphism $F(X) * (*_{i=1}^m \mathcal{H}_i) \rightarrow G$ as a normal subgroup and hence $\langle X, H_i, i \in \{1,...,m\} \vert \mathcal{A} \rangle$ is a relative presentation for $G$. We will also show that the number of such $R$-cells is at most $2^n$ which will also give us that $\delta_G^{\text{rel}}(n) \leq 2^n$ for all $n$. We proceed by induction. In the base case, when $n = 0$, $W$ is the empty word and so we take our van Kampen diagram to just be a point. This has $0 \leq 2^0$ $R$-cells, as required. Now consider a word $W$ of length at most $n$. Let $p$ be a cycle in the relative Cayley graph with label $W$. If $p$ is atomic, then we consider the diagram with a single cell labeled by $W \in \mathcal{A}$, so that the number of $R$-cells is $1 \leq 2^n$. If $p$ is not atomic, then we can write $p = q_1 q_2$, where $\ell(q_1), \ell(q_2) \leq \ell(p) - 1$ and one of the $q_i$ (say, $q_1$) has $\ell(q_1) \leq \frac{1}{2} \ell(p)$ and is not a geodesic. We therefore have a geodesic path $q_1'$ between its endpoints which will be shorter than $q_1$. We can then decompose $p$ into the product of two cycles, as shown in the figure 42 below, both of which are shorter than $p$. Let $V_1$ denote the label of the cycle $q_1 (q_1')^{-1}$ and $V_2$ the label of $q_1' q_2$. Then $W = V_1 V_2$ as elements of the free product $F(X) * (*_{i = 1}^m \mathcal{H}_i)$, and $\vert \vert V_i \vert \vert \leq \vert \vert W \vert \vert - 1 \leq n - 1$ for each $i$. Employing the inductive assumption, we then have van Kampen diagrams with boundaries labeled by $V_1$ and $V_2$ consisting of at most $2^{n-1}$ $R$-cells labeled by words from $\mathcal{A}$. Gluing these diagrams together along a point (so that the boundary of the new diagram is the concatenation of the boundaries of the two smaller diagrams, resembing the concatenation of the two cycles in figure 42 below), we then obtain a diagram with boundary label $W = V_1 V_2$ and with at most $2^{n-1} + 2^{n-1} = 2^n$ $R$-cells labeled by elements from $\mathcal{A}$, as required. Therefore, $\langle X, H_i, i \in \{1,...,m\} \vert \mathcal{A} \rangle$ is indeed a finite relative presentation for $G$ and the relative Dehn function of $G$ for this presentation is well defined.
		
\begin{figure} [H]
	\centering
	\includegraphics[width=0.7\linewidth]{"../Desktop/Honours Project/IMG_1533"}
	\caption{The decomposition of $p$ in the proof of theorem 2.24 when $p$ is not atomic}
	\label{fig:img1533}
\end{figure}
		
	\end{proof}
	
	\vs 
	
	We define an important equivalence relation for functions $f: \mathbb{N} \rightarrow \mathbb{N}$. 
	
	\vs 
	
	\textbf{Definition 2.26: } Given two functions $f,g: \mathbb{N} \rightarrow \mathbb{N}$, we say that $f$ is asymptotically less than $g$, written $f \preccurlyeq g$ if $\exists C,K,L \in \mathbb{N}$ such that $f(n) \leq Cg(Kn) + L$ for all $n \in \mathbb{N}$. We say that $f,g$ are asymptotically equal, written $f \sim g$ if $f \preccurlyeq g$ and $g \preccurlyeq f$.
	
	\vs 
	
	Next, we will prove a theorem demonstrating that weakly relatively hyperbolic groups have linear relative Dehn functions. 
	
	\vs 
	
	\textbf{Theorem 2.27: } For a group $G$ finitely presented relative to a collection of subgroups $\{H_{\lambda}\}_{\lambda \in \Lambda}$, if the relative Cayley graph $\Gamma(G; X \cup \mathcal{H})$ is hyperbolic then the relative Dehn function of $G$ with respect to the relative presentation $\langle X, \mathcal{H}_{\lambda}, \lambda \in \Lambda \vert R \rangle $, $\delta^{\text{rel}}$, is equivalent to a linear function. 
	
	\vs 
	
	To prove this, we must first develop the concept of a Dehn function of a Cayley graph. We begin with some definitions from algebraic topology: 
	
	\vs
	
	\begin{itemize}
		\item A \textit{combinatorial map} between CW complexes is a map such that when restricted to each open cell is a homeomorphism onto an open cell of the codomain CW complex. 
		\item A \textit{combinatorial complex} is a CW complex $K$ such that the attaching map $\phi_e: \partial D^{n+1} \rightarrow Sk^{(n)}K$ ($Sk^{(n)}K$ is the $n$-skeleton of $K$, the union of all $n$-cells of $K$) is a combinatorial map for each $(n+1)$-cell $e$ of $K$. 
		\item A \textit{singular combinatorial map} $f: L \rightarrow K$ between CW complexes $L,K$ is a continuous map such that for every open $n$-cell $e$ in $L$, either $f \vert_e$ is a homeomorphism onto an open cell of $K$ or $f(e) \subseteq Sk^{(n+1)} L$. 
		\item A \textit{k-partition} of $D^2$ for $k \geq 3$ is a homeomorphism $P: D^2 \rightarrow K$ where $K$ is a combinatorial 2-complex such that every 2-cell is the image of a regular $\ell$-gon in $D^2$ for $3 \leq \ell \leq k$. From this $k$-partition we get an induced cell structure on $D^2$: vertices are preimages under $P$ of 0-cells of $K$, edges are preimages of 1-cells and faces are preimages of 2-cells. 
		
		
\begin{figure} [H]
	\centering
	\includegraphics[width=0.7\linewidth]{"../Desktop/Honours Project/IMG_1464"}
	\caption{A $k$-partition of $D^2$ and an induced cell structure on $D^2$}
	\label{fig:img1464}
\end{figure}
		
	\end{itemize} 

	We will mostly be concerned with combinatorial maps, rather than singular combinatorial maps, because it is only the image of combinatorial maps that contribute to areas of diagrams, because singular combinatorial maps may send 2-cells to 1-cells, which do not contribute to the area. However, in some instances it will be necessary to map 2-cells to 1-cells (such as in the 0-bordering process discussed below), so we still consider singular combinatorial maps. 
	
	\vs

	Now let $\Sigma$ be a graph equipped with the combinatorial metric and let $c$ be a cycle in $\Sigma$. A \textit{k-filling} of $c$ consists of a $k$-partition of $D^2$ along with a singular combinatorial map $\Phi: Sk^{(1)} D^2 \rightarrow \Sigma$ such that $\Phi(\partial D^2) = c$. We denote $\vert \Phi \vert$ to be the number of faces of $D^2$ induced by the $k$-partition. 
	
	\vs 
	
	We define the \textit{k-area} of $c$ to be $A^{(k)}(c) := \min \{ \vert \Phi \vert : \Phi \text{ is a k-filling of } c\}$. We then define the \textit{Dehn function of } $\Sigma$ to be the function $f^{(k)}_ {\Sigma} (n) = \sup \{A^{(k)}(c) : c \text{ is a cycle in } \Sigma \text{ with }\ell(c) \leq n\}$. 
	
	\vs 
	
	The next lemma tells us that for some $k$, every cycle in a relative Cayley graph admits a $k$-filling. 
	
	\vs 
	
	\textbf{Lemma 2.28: } Let $G$ be a group with reduced, finite relative presentation $\langle X, \mathcal{H} \vert R \rangle$. Then for $k = \max \{\max \{\vert \vert r \vert \vert : r \in R\}, 4\}$, every cycle $q$ in $\Gamma(G; X \sqcup \mathcal{H})$ admits a $k$-filling with $A^{(k)}(q) \leq (k+1) \text{Area}^{\text{rel}}(q) + 2 \ell (q)$. 
	
	\begin{proof}
		
		We sketch a proof of this result. 
		
		\vs 
		
		First, we fix a van Kampen diagram $\Delta$ of minimal type such that $\phi(\partial \Delta) = \phi(q)$. By Corollary 2.20 we obtain $\sum_{\Pi \in S(\Delta)} \ell(\partial \Pi) \leq k \text{Area}^{\text{rel}}(q) + \ell(q)$. 
		
		\vs 
		
		We next obtain a new diagram $\Psi$ by triangulating every $S$-cell of length greater than 3 (see picture). Denote the triangulation of each such $S$-cell $\Pi$ by $Y(\Pi)$ and replace each such $\Pi$ by $Y(\Pi)$. We obtain $\text{Area} (\Psi) \leq (k+1) \text{Area}^{\text{rel}}(q) + \ell(q)$. 
		
		\vs 
		
		Next, we define a $k$-filling of $q$ using $\Psi$. We note that every 2-cell is an $\ell$-gon. However, in general, $\Psi$ may not be homeomorphic to a disk. To assure that $\Psi$ is homeomorphic to a disk, we perform a topological procedure called a 0-bordering of the contour of $\Psi$ by adding $\ell(q)$ new cells. The result is a diagram $\Psi'$ which is a $k$-filling of $q$. In this case, we require the added 2-cells $\Pi_i$ to be sent to an edge in order to ensure $\Phi(\partial D^2) = c$. 
		
		\vs 
		
		We then have $A^{(k)}(q) \leq \text{Area}(\Psi') = \text{Area}(\Psi) + \ell(q) \leq (k+1) \text{Area}(q) + 2 \ell(q)$, as required. 
		
		\vs
		
\begin{figure} [H]
	\centering
	\includegraphics[width=0.7\linewidth]{"../Desktop/Honours Project/IMG_1535"}
	\caption{Triangulation of a 2-cell $\Pi$}
	\label{fig:img1535}
\end{figure}

\begin{figure} [H]
	\centering
	\includegraphics[width=0.7\linewidth]{"../Desktop/Honours Project/IMG_1536"}
	\caption{0-bordering of a 2-cell $\Pi$ by the 0-faces $\Pi_i$}
	\label{fig:img1536}
\end{figure}


	\end{proof}

	The above lemma allows us to deduce the following: 
	
	\vs
	
	\textbf{Theorem 2.29: } Suppose $G$ is finitely presented relative to $\{H_{\lambda}\}_{\lambda \in \Lambda}$ and suppose $\delta^{\text{rel}}$ (wrt $\langle X, \mathcal{H} \vert R \rangle$) is well-defined. Then $f_{\Gamma(G; X \sqcup \mathcal{H})} \sim \delta^{\text{rel}}$. 
	
	\begin{proof}
		
		We denote $\Gamma = \Gamma(G; X \sqcup \mathcal{H})$. By Lemma 2.28, for any $n \in \mathbb{N}$ and any cycle $q$ with $\ell(q) \leq n$ we have $A^{k}(q) \leq (k+1) \text{Area}^{\text{rel}}(q) + 2 \ell(q) \leq (k+1) \delta_G^{\text{rel}}(n) + 2n$, for $k$ as in Lemma 2.28. Then, taking the supremum over all such cycles we obtain $f_{\Gamma} (n) \leq (k+1) \delta_G^{\text{rel}}(n) + 2n$ for all $n \in \mathbb{N}$ and therefore $f_{\Gamma} \preccurlyeq \delta_G^{\text{rel}}$. 
		
		\vs 
		
		For the reverse inequality, let $k \in \mathbb{N}$ be as in the above Lemma (such that any cycle in $\Gamma$ admits a $k$-filling). Let $q$ be a cycle in $\Gamma$ such that $\ell(q) \leq n$. Let $\Phi: Sk^{(1)} D^2 \rightarrow \Gamma$ be a $k$-filling with the minimum number of faces induced on $D^2$. We then obtain $\vert \Phi \vert \leq f_{\Gamma}(n)$ for all $n \in \mathbb{N}$. We shall use $\Phi$ to define labels and orientation on edges of $D^2$. We define orientation on edges of $D^2$ according to orientation on edges on the image of $\Phi(e)$, as shown in the picture below. 
		
		\begin{figure} [H]
			\centering
			\includegraphics[width=0.7\linewidth]{"../Desktop/Honours Project/IMG_1534"}
			\caption{Definition of edge orientation in the proof of Theorem 2.28}
			\label{fig:img1534}
		\end{figure}
		
		\vs 
		
		Now let $\Pi$ be a 2-cell of $D^2$. Then because we have a $k$-partition of $D^2$, every 2-cell has length at most $k$, thus $\ell(\partial \Pi) \leq k$. Therefore, as $\phi(\partial \Pi)$ is 1 in $G$, we have a van Kampen diagram $\Xi(\Pi)$ over $\langle X, \mathcal{H} \vert R \rangle$ such that $\phi(\partial \Xi(\Pi)) = \phi(\partial \Pi)$ and, by definition of the Dehn function, $N_{R} (\Xi(\Pi)) \leq \delta^{\text{rel}}(k) < \infty$ (as the relative Dehn function is well-defined). Replacing every 2-cell $\Pi$ of $D^2$ with $\Xi(\Pi)$, we obtain a van Kampen diagram $\Delta$ over $\langle X, \mathcal{H} \vert R \rangle$ such that $\phi(\Delta) = \phi(q)$. From this diagram we obtain: 
		
		\vs 
		
		$\text{Area}^{\text{rel}}(q) \leq N_R (\Delta) \text{ (by definition of Area of a cycle)} \leq \delta^{\text{rel}}(k) \vert \Phi \vert \text{ (because there are } \vert \Phi \vert \text{ 2-cells and we split each 2-cell into }  N_{R} (\Xi(\Pi)) \leq \delta^{\text{rel}}(k) \text{ smaller 2-cells)} \leq f_{\Gamma} (n) \delta^{\text{rel}}(k) \text{ (because  } \vert \Phi \vert \leq f_{\Gamma}(n) )$.
		
	\end{proof}

	To prove our desired theorem, we need one more result:
	
	\vs 
	
	\textbf{Theorem 2.30: } Let $G$ be a group and $\{H_{\lambda}\}_{\lambda \in \Lambda}$ a collection of subgroups. If $\Gamma(G; X \sqcup \mathcal{H})$ is a hyperbolic space and $\delta^{\text{rel}}_G$ is well-defined, then $\Gamma(G; X \sqcup \mathcal{H})$ has a linear Dehn function. 
	
	\begin{proof}
		
		We sketch a proof of this result, following a proof given in [3]. The converse of the above is also true, though we shall omit its proof (a proof of the converse can also be found in [3] - it is Theorem 2.9 in [3]). 
		
		\vs 
		
		By increasing $\delta$ if necessary, we may assume that $\delta$ is a positive integer. Let $c$ be a loop in $\Gamma(G; X \sqcup \mathcal{H})$. We first introduce some notation and terminology. Let $\ell_0(c)$ denote the number of maximal non-trivial loops about vertices of $c$ where $c$ is constant (that is, vertices which are the images under $c$ of some maximal non-zero subinterval). We note that $\ell_0(c) \leq \ell(c) + 1$, because $\ell_0(c)$ is at most the number of vertices of $c$, which is at most $\ell(c) + 1$. By a standard $k$-filling $(P, \Phi)$ of $c$, we mean a $k$-filling of $c$ in which we triangulate $D^2$ and the singular combinatorial map $\Phi$ sends vertices on the boundary of $D^2$ to vertices on $c$ and $\Phi$ sends each edge of $D^2$ to either a concatenation of edges of $c$ or to a vertex of $c$. 
		
	
\begin{figure} [H]
	\centering
	\includegraphics[width=0.7\linewidth]{"../Desktop/Honours Project/IMG_1506"}
	\caption{ The combinatorial map $\Phi$ in the proof of Theorem 2.29}
	\label{fig:img1506}
\end{figure}
		
		\vs 
		
		We proceed by induction on $\ell(c) + \ell(c_0)$ to show that every loop $c$ in $\Gamma(G; X \sqcup \mathcal{H})$ admits a standard $16\delta$-filling of area $(8 \delta + 2)(\ell(c) + \ell_0(c))$, which will show that $\Gamma(G; X \sqcup \mathcal{H})$ has a linear Dehn function, by Remark 2.3 (5) in [3]. 
		
		\vs 
		
		In the base case, when $\ell(c) + \ell_0(c) = 1$, we have either $\ell(c) = 1$ and $\ell_0(c) = 0$ or $\ell(c) = 0$ and $\ell_0(c) = 1$. These cases are illustrated below. In the latter case, the image of $c$ consists of just a vertex, so we may take any 2-complex $K$ homeomorphic to $D^2$ consisting of exactly $8 \delta + 2$ triangles to get an induced cell structure on $D^2$ consisting of $8 \delta + 2$ 2-cells, and we take our singular combinatorial map $\Phi$ to be the constant map at the unique vertex of $c$. We therefore have a standard $16 \delta$-filling of $c$ of area $8 \delta + 2$. In the former case, the image of $c$ consists of a unique vertex and a unique edge, so we again form any 2-complex homeomorphic to $D^2$ with $8 \delta + 2$ triangles and we then define our singular combinatorial map $\Phi$ to map every edge along with the boundary endpoints of the edge homeomorphically onto the loop $c$. This again gives us a $16 \delta$-filling of $c$ of area $8 \delta + 2$. Therefore, in either case, we achieve a $16 \delta$-filling of $c$ of area $8 \delta + 2$. 
		
\begin{figure} [H]
	\centering
	\includegraphics[width=0.7\linewidth]{"../Desktop/Honours Project/IMG_1507"}
	\caption{Base case of Theorem 2.30}
	\label{fig:img1507}
\end{figure}
		\vs 
		
		Now for the inductive step, we assume $\ell(c) \geq 2$. We consider two cases: 
		
		\begin{itemize}
			\item $\ell_0(c) = 0$. Then $c$ is either locally injective or $c$ contains a loop which consists of traversing an edge and then immediately traversing that same edge backwards to return to the original starting vertex. In the former case, by Lemma 2.6 of Bridson and Haefliger [3], there exist points $s,t \in [0,1]$ such that $c(s), c(t)$ are vertices of $X$ and $d(c(s), c(t)) \leq \min \{\ell(c_{\restriction [s,t]}) - 1, 16 \delta - \ell(c_{\restriction [s,t]})\}$. Identifying $\partial (D^2) = S^1$ with $[0,1]$ with its endpoints identified, we take $s,t \in \partial(D^2)$ and map the shortest path on $\partial(D^2)$ between $s,t$ to a geodesic segment in $X$ connecting $c(s), c(t)$. In the latter case, we take points $s,t$ on $\partial(D^2)$ which map to the endpoints of the edge $e$ along which we travel back and forth and we connect $s,t$ with the straight line path in $D^2$ between them. We map this straight line path homeomorphically onto $e$. 
			
\begin{figure} [H]
	\centering
	\includegraphics[width=0.7\linewidth]{"../Desktop/Honours Project/IMG_1508"}
	\caption{$\ell_0(c) = 0$ when $c$ is locally injective}
	\label{fig:img1508}
\end{figure}
			\item $\ell_0(c) \geq 1$, then we take a subpath $c_{\restriction [s,t]}$ which consists of the concatenation of a vertex and an edge. In $D^2$, we then join $s,t$ by a straight line and homeomorphically map this straight line onto $c_{\restriction [s,t]}$. 
			
			\begin{figure} [H]
				\centering
				\includegraphics[width=0.7\linewidth]{"../Desktop/Honours Project/IMG_1514"}
				\caption{$\ell_0(c) = 0$ when $c$ is not locally injective}
				\label{fig:img1514}
			\end{figure}
			
\begin{figure} [H]
	\centering
	\includegraphics[width=0.7\linewidth]{"../Desktop/Honours Project/IMG_1509"}
	\caption{$\ell_0(c) \geq 1$ case}
	\label{fig:img1509}
\end{figure}
		\end{itemize}
	
		In each of these cases, we divide $D^2$ into two regions, one region which is mapped to an loop containing $c([s,t])$ of length at most $16 \delta$ and the other region being the remainder of $D^2$ which has boundary mapped to a loop $c'$ with $\ell(c') + \ell_0(c') < \ell(c) + \ell_0(c)$. By the induction assumption, there exists a $16 \delta$-filling $(P, \Phi)$ of $c'$ with area $(8 \delta + 2)(\ell(c') + \ell_0(c')) < (8 \delta + 2)(\ell(c) + \ell_0(c))$. We complete this filling of $c'$ to the desired filling of $c$ by joining the vertices of the triangulation of the portion of $D^2$ associated to $c'$ to a designated vertex on the straight line path joining $s,t$ in $D^2$ (see picture below). 

\begin{figure} [H]
	\centering
	\includegraphics[width=0.7\linewidth]{"../Desktop/Honours Project/IMG_1515"}
	\caption{Completing the filling of $c'$ to a filling of $c$}
	\label{fig:img1515}
\end{figure}
		
	\end{proof}

	\vs
	
	We are now ready to prove the desired theorem:
	
	\begin{proof}
		
		Since $\Gamma(G; X \sqcup \mathcal{H})$ is hyperbolic, by Theorem 2.30 $f_{\Gamma(G; X \sqcup \mathcal{H})}$ is linear. Furthermore, by Theorem 2.29, $\delta^{\text{rel}} \sim f_{\Gamma(G; X \sqcup \mathcal{H})}$. Therefore, $\delta^{\text{rel}}$ is equivalent to a linear function.
		
	\end{proof} 


\newpage
	\subsection{Geometry in Relative Cayley Graphs}
	
	In this subsection, we assume that we have a finitely generated group $G$ hyperbolic relative to a finite collection of subgroups $\{H_1,...,H_m\}$. We choose a finite generating set $X$ of $G$ such that the following conditions hold: (i) we have a finite relative presentation $\langle X, H_1,...,H_m \vert R \rangle$, and we let $L \in \mathbb{N}$ be such that $\delta^{rel}(n) \leq Ln$ for all $n \in \mathbb{N}$ and such that $ML > 1$, where $M = \max \{\vert \vert r \vert \vert : r \in R\}$, (ii) for any cycle $q$ in $\Gamma(G; X \sqcup \mathcal{H})$, if $p_1,...,p_k$ are isolated components $H_i$ components of $q$, then $\sum_{i=1}^k d_X((p_i)_-, (p_i)_+) \leq ML \ell(q)$. Since $G$ is hyperbolic relative to $\{H_1,...,H_m\}$, we have that $\Gamma(G; X \sqcup \mathcal{H})$ is a hyperbolic space (since the coned off and relative Cayley graphs are quasi-isometric, as discussed in Definition 2.4, and the coned off Cayley graph is hyperbolic). We now show that there exists a finite generating set $X$ of $G$ which satisfies the above requirements and also satisfies the Lemma below: 
	
	\vs 
	
	\textbf{Lemma 2.31: } Let $G$ be a finitely generated group hyperbolic relative to subgroups $\{H_1,...,H_m\}$. Then there exists a finite generating set $X$ such that the following condition holds: 
	
	\vs 
	
	For any cycle $q$ in $\Gamma(G; X \sqcup \mathcal{H})$ and any isolated $H_i$ components $p_1,...,p_k$ of $q$, we have $\sum_{i=1}^k d_X((p_i)_{-}, (p_i)_{+}) \leq ML \ell(q)$. 
	
	\begin{proof}
		
		We begin with a finite, reduced relative presentation $\langle X', H_1,...,H_m \vert R' \rangle$ for $G$ (as seen in the previous section, such a relative presentation always exists). Now define $X = X' \cup \Omega$, where $\Omega$ is as defined in section 2.2. Then $X$ is a finite set ($X'$ is finite from the definition of a finite relative presentation and that $\Omega$ is finite follows from $R'$ being finite (see section 2.2)). Then $X \supseteq X' \cup \mathcal{H}$, so $X$ is a generating set for $G$. Define $R = R' \cup \{w \mathcal{W}^{-1}: w \in \Omega\}$, with $\mathcal{W}$ being a word in $F(X')$ representing $w$ in $G$. We see that $\langle X, H_1,...,H_m \vert R \rangle$ arises by applying a finite sequence of Tietze transformations to $\langle X', H_1,...,H_m \vert R' \rangle$ (finite because we have a Tietze transformation for each element of $\Omega$, and $\Omega$ is finite), therefore $\langle X, H_1,...,H_m \vert R \rangle$ defines a group isomorphic to $G$, as required. Note that we may increase $L$ in the inequality $\delta^{\text{rel}}_G(n) \leq Ln$ such that $ML > 1$ without any problems. It remains to show that the set $X$ we have defined above satisfies the above Lemma. 
		
		\vs 
		
		For the isolated components $p_1,...,p_k$ in the Lemma, denote the endpoints $u_i = (p_i)_-$ and $v_i = (p_i)_+$. By Lemma 2.21, we have $\sum_{i = 1}^k \vert u_i^{-1} i_i \vert \leq M' \text{Area}(q) \leq M' L' \ell(q)$, where $M' = \max_{r \in R'} \vert \vert r \vert \vert$ and $L'$ is the bound on the relative Dehn function coming from the relative presentation $\langle X', H_1,...,H_m \vert R' \rangle$. We increase $L$ such that $L' \leq L$. Since $R \supseteq R'$, we note that $M:= \max_{r \in R} \vert \vert r \vert \vert \geq M'$. Therefore, we have $\sum_{i=1}^k d_X((p_i)_+, (p_i)_-) = \sum_{i=1}^k \vert u_i^{-1} v_i \vert_X \leq \sum_{i=1}^k \vert u_i^{-1}v_i \vert_{\Omega} \leq M' L' \ell(q) \leq ML \ell(q)$, as required. 
		
	\end{proof}
	
	\vs
	
	\underline{Quasi-geodesics in relative Cayley graphs}
	
	\vs
	
	We examine some results about quasi-geodesics in relative Cayley graph that will be useful to us when we study such problems as the conjugacy problem in relatively hyperbolic groups. 
	
	\vs 
	
	Recall that for relative Cayley graphs we have two notions of distances: relative $d_{X \cup \mathcal{H}}$ and non-relative $d_X$. We will find that we can often deduce bounds on non-relative distances using facts about the relative Cayley graph. We define an important measure of closeness of quasi-geodesics in the relative Cayley graph. 
	
	\vs
	
	\textbf{Definition 2.32: } We say that two $(\lambda, c)$ quasi-geodesics $p, q$ in $\Gamma(G; X \cup \mathcal{H})$ are \textit{k-similar} if $d_X(q_{-}, p_{-}), d_X(q_{+}, p_{+}) \leq k$. 
	
	\vs 
	
	\textbf{Definition 2.33: } A vertex $a$ on a path $p$  in $\Gamma(G; X \sqcup \mathcal{H})$ is called a \textit{phase vertex} if it is not in between edges labeled by letters from a parabolic subgroup. A path $p$ is called \textit{locally minimum} if every vertex of $p$ is a phase vertex. Equivalently, every parabolic component of $p$ has length 1. 
	
	\vs 
	
	The next lemma tells us that we may always assume that our path is locally minimum. 
	
	\vs 
	
	\textbf{Lemma 2.34: } If $p$ is a path in $\Gamma(G;X \cup \mathcal{H})$, then there exists a locally minimum path $\hat{p}$ with the same set of phase vertices as $p$. In addition, if $p$ is a $(\lambda, c)$ quasi geodesic then so is $\hat{p}$ and if $p$ has no backtracking then neither does $\hat{p}$.
	
\begin{figure} [H]
	\centering
	\includegraphics[width=0.7\linewidth]{"../Desktop/Honours Project/IMG_1465"}
	\caption{Collapsing components to obtain a locally minimum path}
	\label{fig:img1465}
\end{figure}
	
	\begin{proof}
		
		Let $\hat{p}$ be the path obtained by replacing all parabolic components of $p$ by a single edge (labeled by an element of the corresponding parabolic subgroup). Then $p$ and $\hat{p}$ have the same set of phase vertices, because collapsing all of the parabolic components of $p$ does not affect the set of phase vertices. 
		
		\vs 
		
		Next, suppose that $p$ is a $(\lambda, c)$-quasi-geodesic. Let $q$ be a subpath of $\hat{p}$. We obtain a subpath $\tilde{q}$ of $p$ by replacing every parabolic edge of $q$ with the corresponding parabolic syllable in $p$. Since $q$ is a subpath of $\tilde{q}$, we have $\ell(q) \leq \ell(\tilde{q})$. In addition, since $p$ is $(\lambda,c)$-quasi-geodesic we have $\ell(\tilde{q}) \leq \lambda d_{X \cup \mathcal{H}}(\tilde{q}_{-}, \tilde{q}_{+}) + c = \lambda d_{X \cup \mathcal{H}}(q_{-}, q_{+}) + c$ (since $q, \tilde{q}$ have the same endpoints). Therefore, $\ell(q) \leq \lambda d_{X \cup \mathcal{H}}(q_{-}, q_{+}) + c$, so $\hat{p}$ is $(\lambda, c)$ quasi-geodesic.
		
		\vs 
		
		Lastly, suppose that $p$ does not backtrack. If $\hat{p}$ backtracks, then there exists a pair of $H_{\lambda}$ components of $\hat{p}$ which are connected. But then replacing these components with the syllables in $p$, we obtain a pair of connected components of $p$, a contradiction. 
		
	\end{proof}

	Our goal is to prove the following theorem: 
	
	\vs 
	
	\textbf{Theorem 2.35: } For any $\lambda \geq 1, c \geq 0, k \geq 0$, $\exists \varepsilon \geq 0$ depending only on $\lambda, c$ and $k$ such that for any k-similar $(\lambda, c)$ quasi geodesics without backtracking $p,q$, the following hold: 
	
	\begin{enumerate}[label = (\alph*)]
		\item The Hausdorff distance (using the non-relative metric $d_X$) between the sets of phase vertices of $p,q$ is bounded above by $\varepsilon$. 
		\item If $s$ is an isolated $H_{\lambda}$ component of $p$ then $d_X(s_{-}, s_{+}) \leq \varepsilon$. 
		\item If $s,t$ are connected components of $p,q$, respectively, then $d_X(s_{-}, t_{-}), d_X(s_{+}, t_{+}) \leq \varepsilon$. 
	\end{enumerate} 

\begin{figure} [H]
	\centering
	\includegraphics[width=0.7\linewidth]{"../Desktop/Honours Project/IMG_1466"}
	\caption{An illustration of the statements of Theorem 2.35}
	\label{fig:img1466}
\end{figure}

	We begin with the proof of part (a) of the main theorem: 
	
	\vs
	
	By Lemma 2.34 we can assume that $p,q$ are locally minimal by contracting all parabolic components of $p,q$ to single parabolic letters. Applying Lemma 1.30, to the pair $p,q$ we get a constant $K_0 = K(\delta, \lambda, c, k) + 1/2$ such that for every vertex $a$ on $p$, there exists a vertex $b$ on $q$ such that $d_{X \cup \mathcal{H}}(a,b) \leq K_0$. Set $K = K(\delta, \lambda, c, K_0) + 1/2$. Note that we can increase $K_0, K$ as necessary to acquire $K \geq K_0 \geq k$. We select vertices $u_1, u_2$ on $p$ according to the following scheme: 
	
	\begin{enumerate}[label = (\roman*)]
		\item If $d_{X \cup \mathcal{H}}(p_-, u) \leq 2K$ (resp. $d_{X \cup \mathcal{H}}(p_+, u) \leq 2K$) then put $u_1 = p_-$ (resp. $u_2 = p_+$). 
		\item If $d_{X \cup \mathcal{H}}(p_-, u) > 2K$ (resp. $d_{X \cup \mathcal{H}}(p_+, u)> 2K$), then define $u_1$ to be the point on $[p_-, u]$ such that $d_{X \cup \mathcal{H}}(u_1, u) = 2K$ (resp. $u_2$ on $[u, p_+]$ such that $d_{X \cup \mathcal{H}}(u_2, u) = 2K$), where by $[p_-, u]$ and $[u, p_+]$ we mean the segments along $p$. 
	\end{enumerate}

	Applying Lemma 1.30, there exist points $v_1, v_2$ on $q$ such that $d_{X \cup \mathcal{H}}(u_i, v_i) \leq K_0$ for each $i = 1,2$. We may assume that if $u_1 = p_-$ (resp. $u_2 = p_+$) then $v_1 = q_-$ (resp. $v_2 = q_+$) because if $d_{X \cup \mathcal{H}}(p_{\pm}, q_{\pm}) \leq d_X(p_{\pm}, q_{\pm}) \leq k \leq K_0$. We now set some notation.
	
	\vs 
	
	Define $p_1 = [p_-, u_1]$ and $p_2 = [u_1, u_2]$ along $p$ and $q_0 = [v_1, v_2]$ along $q$. We next define geodesics $o_i$ between $u_i$ and $v_i$ for each $i$ as follows: 
	
	\begin{enumerate} [label = (\roman*)]
		\item If $u_1 = p_-$ (resp. $u_2 = p_+$) then $o_1$ (resp. $o_2$) is a geodesic in $\Gamma(G; X)$ (i.e. the label of $o_1$ (resp. $o_2$) consist only of letters from $X$).
		\item If $u_1 \neq p_-$ (resp. $u_2 \neq p_+$) then $o_1$ (resp. $o_2$) is a geodesic in $\Gamma(G; X \sqcup \mathcal{H})$. 
	\end{enumerate}

	We observe that in each case, we have $\ell(o_i) \leq K_0$ for each $i$ (indeed, in case (i) where $o_i$ is a geodesic in $X$, then we have $\ell(o_i) = d_X((o_i)_-, (o_i)_+) \leq k \leq K_0$ and in case (ii) where $o_i$ is a geodesic in $X \cup \mathcal{H}$ then we have $\ell(o_i) = d_{X \cup \mathcal{H}}((o_i)_-, (o_i)_+) \leq d_X((o_i)_-, (o_i)_+) \leq K_0$).
	
	\vs 
	
	Let $V$ be the set of all vertices $z$ on $q_0$ which are at minimum distance from $u$ (i.e. $d_{X \cup \mathcal{H}}(z, u) = \min \{d_{X \cup \mathcal{H}}(u, v) : v \text{ is a vertex on } V\}$). Note that for all $z \in V$, we have $d_{X \cup \mathcal{H}}(u, z) \leq K(\delta, \lambda, c, K_0) + 1/2 = K$ (indeed, as shown above, the distance between the endpoints of $p_1$ and $q_0$ is bounded above by $K_0$, so applying Lemma 1.30, there exists a point $b$ on $q_0$ such that $d_{X \cup \mathcal{H}} (u, b) \leq K(\delta, \lambda, c, K_0)$, but $b$ is a distance of at most 1/2 from a vertex $v$ on $q_0$, so $u$ is a distance of at most $K(\delta, \lambda, c, K_0) + 1/2$ from $v$, and so $d_{X \cup \mathcal{H}}(u,z) \leq d_{X \cup \mathcal{H}}(u,v) \leq K(\delta, \lambda, c, K_0) + 1/2$). 
	
	\vs 
	
	For any $z \in V$, we define $O(z)$ to be the set of all geodesics $o$ in $\Gamma(G; X \sqcup \mathcal{H})$ with $o_- = u, o_+ = z$. For each $z \in V$ and $o \in O(z)$, we have a picture as below: 
	
\begin{figure} [H]
	\centering
	\includegraphics[width=0.7\linewidth]{"../Desktop/Honours Project/IMG_1470"}
	\caption{Setup of the proof of Theorem 2.35 (a)}
	\label{fig:img1470}
\end{figure}
	
	Our next goal will be to bound the $X$-distance between $u$ and any $z \in V$. We will do this by bounding above the $X$-length of each component of an $o \in O(z))$ by showing that every component of $o$ is isolated in at least one the cycles $c_1, c_2$, as shown in figure 55. We accomplish this through a series of lemmas, whose proof we sketch using pictures. Before we begin with the lemmas, we will need one more definition. 
	
	\vs 
	
	\textbf{Definition 2.36: } For $z \in V$ and $o \in O(z)$, an \textit{ending component} of $o$ is a component of $o$ that contains the vertex $z$ as an endpoint. If $o$ does not have a component which contains $z$ as an endpoint then it is called a \textit{non-ending component} of $o$.
	
	\vs 
	
	We will deal with ending and non-ending components separately, showing that in each case, the component is isolated in at least one of the cycles $c_1, c_2$. We begin by making a general observation about components.
	
	\vs 
	
	\textbf{Lemma 2.37: } No $H_i$ component of $o_j$ ($j=1,2$) is connected to an $H_i$ component of $o$. 
	
	\begin{proof}
		
		If we had a connection between a component of $o$ and a component of $o_j$, then using the connection between these components, we would deduce that $d_{X \cup \mathcal{H}} < 2K$, so $o_1$ would be a geodesic in $\Gamma(G; X)$, so it would not have any components at all, a contradiction.
		
\begin{figure} [H]
	\centering
	\includegraphics[width=0.7\linewidth]{"../Desktop/Honours Project/IMG_1471"}
	\caption{The contradiction in the proof of Lemma 2.37}
	\label{fig:img1471}
\end{figure}
		
	\end{proof}

	\textbf{Lemma 2.38: } If $s$ is a non-ending component of $o$ and if $s$ is not isolated in $c_j$ then $s$ must be connected to a component of $p_j$. 
	
	\begin{proof}
		
		WLOG, we take $j = 1$. By the above lemma, we know that $s$ cannot be connected to a component of $o_1$. Also $s$ cannot be connected to another component of $o$ because $o$ is a geodesic and hence does not backtrack. Lastly, $s$ cannot be connected to any component of $q_1$, as similar to above, this would yield a path shorter than $o$ between $u$ and a vertex on $q_0$, a contradiction.
		
	\end{proof}

	The above lemma gives us the following corollary: 
	
	\vs 
	
	\textbf{Corollary 2.39: } For every $z \in V$ and $o \in O(z)$, every non-ending component of $o$ is isolated in at least one of the cycles $c_j$. 
	
	\begin{proof}
		
		If a component $s$ of $o$ were not isolated in either cycle, then by the previous lemma, $s$ would be connected to a component of $p_1$ and to a component of $p_2$. This would then yield a connection between two components of $p$, contradicting that $p$ does not backtrack. 
		
	\end{proof}

	We now prove a similar result for ending components. 
	
	\vs 
	
	\textbf{Lemma 2.40: } Let $j \in \{1,2\}$. Suppose that for all $z \in V$, every $o \in O(z)$ has an ending component which is not isolated in $c_j$. Then for every $z \in V$, every $o \in O(z)$ is connected to a component of $p_j$. 
	
	\begin{proof}
		
		Let $s$ be an ending component of $o \in O(z)$ which is not isolated in $c_1$ (taking $j = 1$, WLOG). We note that $s$ cannot be connected to another component of $o$ because $o$ is a geodesic and hence does not backtrack and $s$ cannot be connected to any component of $o_1$ as argued previously. Therefore, $s$ is connected to a component of $q_1$ or a component of $p_1$. We apply induction on $d_{X \cup \mathcal{H}} (z, v_1)$ to show that there exists a component of $p_1$ connected to $s$.
		
		\vs 
		
		For the base case, $d_{X \cup \mathcal{H}}(z, v_1) = 0$, the segment $q_1$ is trivial so $s$ cannot be connected to $q_1$, so it must be connected to a component of $p_1$. 
		
\begin{figure} [H]
	\centering
	\includegraphics[width=0.7\linewidth]{"../Desktop/Honours Project/IMG_1473"}
	\caption{The base case in the proof of Lemma 2.40}
	\label{fig:img1473}
\end{figure}
		
		\vs 
		
		If $d_{X \cup \mathcal{H}}(z, v_1) \geq 1$, suppose $s$ is connected to a component $t$ of $q_1$. As shown in  figure 58 below, we can form a geodesic $o'$ from $u$ to $t_-$ which has the same length as $o$, so $t_- \in V$ and $o' \in O(t_-)$. But $d_{X \cup \mathcal{H}}(v_1, t_-) < d_{X \cup \mathcal{H}}(v_1, z)$, so by induction we have a component of $p_1$ connected to $c$ (see figure 58). This component of $p_1$ is then connected to $s$ because $c$ is connected to $s$.
		
\begin{figure} [H]
	\centering
	\includegraphics[width=0.6\linewidth]{"../Desktop/Honours Project/IMG_1474"}
	\caption{The case $d_{X \cup \mathcal{H}} \geq 1$ in the proof of Lemma 2.40}
	\label{fig:img1474}
\end{figure}
		
	\end{proof}

	From the above lemma, we obtain the following corollary: 
	
	\vs 
	
	\textbf{Corollary 2.41: } There is a $z \in V$ and $o \in O(z)$ such that either $o$ does not have an ending component or $o$ does have an ending component and the ending component of $o$ is isolated in at least one of the cycles $c_1,c_2$.
	
	\begin{proof}
		
		Suppose that no such $z$ existed. Then for every $z \in V$ and every $o \in O(z)$, $o$ has an ending component which is not isolated in either cycle. By the previous lemma, for every $z \in V$, every $o \in O(z)$ has a component connected to a component of both $p_1$ and $p_2$. However, this violates the fact that $p$ does not backtrack. 
		
	\end{proof}

	Returning to the proof of part (a) of Theorem 2.35, we have now shown that all components of every $o \in O(z)$ are isolated in at least one of $c_1, c_2$. We will use this fact to now bound the $X$ - length of these components. 
	
	\vs 
	
	By Corollaries 2.39 and 2.40, there exists a $z \in V$ and an $o \in O(z)$ such that every component $s$ of $o$ is isolated in $c_j$ for some $j$. By Lemma 2.31, we have $d_X(s_-, s_+) \leq ML \ell(c_j)$. 
	
	\vs 
	
	Bounding above the lengths of the subpaths comprising $c_j$, we have $\ell(p_j) \leq \lambda d_{X \cup \mathcal{H}}(u_1, u) + c \leq 2K\lambda + c$, $\ell(q_j) \leq \lambda d_{X \cup \mathcal{H}}(v_1, v_2) + c \leq \lambda (d_{X \cup \mathcal{H}}(u_1, v_1) + d_{X \cup \mathcal{H}}(u_1, u_2) + d_{X \cup \mathcal{H}}(u_2, v_2)) + c \leq 6 K \lambda + c$, $\ell(o), \ell(o_j) \leq K$. Thus, $\ell(c_j) \leq 8K \lambda + 2c + 2K$ for each $j$.  Thus, $d_X(s_-, s_+) \leq ML(8K \lambda + 2c + 2K)$ for every component $s$ of $o$. Hence, $d_X(u,z) \leq d_{X \cup \mathcal{H}}(u,z) \max \{d_X(s_-, s_+): s \text{ is a component of } o\}\leq K ML(8K \lambda + 2c + 2K)$, as required. This proves part (a) of Theorem 2.35. 
	
	\vs
	
	Part (b) of Theorem 2.35 follows quickly from BCP. Indeed, given $k-$similar $(\lambda, c)$-quasigeodesics $p,q$ with no backtracking, connect $p_-$ to $q_-$ and $p_+$ to $q_+$ via geodesics $r,s$ in $\Gamma(G;X)$. Then $\ell(r) = d_X(p_-, q_-) \leq k$ and $\ell(s) = d_X(p_+, q_+) \leq k$. Then by Lemma 1.19 the concatenations $rp$ and $qs$ are $(\lambda, c + (\lambda + 1)k)$-quasi-geodesics without backtracking (because $p,q$ have no backtracking and $r,s$ have no components), so $\exists C = C(\lambda, c, k)$ such that $d_X(s_-, s_+) \leq C$ for any component $s$ isolated in the cycle $(rp)(qs)^{-1}$. 
	
	\vs 
	
	Lastly, to prove (c), suppose we have a pair of connected components $s,t$ of $p,q$, respectively. Denote $c'$ the connector between $s_-, t_-$ and $c''$ the connector between $s_+, t_+$. Let $p_1 = [p_-, s_-]$ along $p$ and let $q_1 = [q_-, t_-]$ along $q$. We note that $c'$ is not connected to any component of $p_1$ or $q_1$ as this would contradict the fact that $p,q$ do not backtrack. By Lemma 1.19, $p_1 c'$ is a $(\lambda, c + \lambda + 1)$ quasi-geodesic (because $c'$ has length 1 in $\Gamma(G; X \cup \mathcal{H}))$. Therefore, $p_1 c', q_1$ is a pair of $k$-similar $(\lambda, c + \lambda + 1)$ quasi geodesics without backtracking and $c'$ is an isolated component, so by (b) we have $d_X(s_-, t_-) \leq C(\lambda, c + \lambda + 1, k)$. A symmetric argument (working on the right hand side of the quasi geodesics $p,q$) yields $d_X(s_+, t_+) \leq C(\lambda, c + \lambda + 1, k)$.
	
\begin{figure} [H]
	\centering
	\includegraphics[width=0.7\linewidth]{"../Desktop/Honours Project/IMG_1475"}
	\caption{An illustration of the proof of Theorem 2.35 (c)}
	\label{fig:img1475}
\end{figure}
	
	$\square$
	
	Note that above, to prove part (b) of Theorem 2.35, we used BCP from the definition of relatively hyperbolic groups in this paper. However, we note that starting from Osin's definition of relatively hyperbolic groups (i.e. hyperbolic relative Cayley graph and linear relative Dehn function, which we have actually used throughout this section), we may prove (b), from which we can deduce the definition of relatively hyperbolic groups of this paper. We note that Lemma 2.31, from which the proof of Theorem 2.35 (a) depended on, holds true when we Osin's definition. Let us now prove Theorem 2.35 (b) using Osin's definition and see how this implies the BCP. 
	
	\vs 
	
	\textit{Proof of Theorem 2.35 (b) using Osin's definition: } 
	
	\vs 
	
	We will show that there exists a constant $C \geq 0$ such that for every component $s$ of $p$ which is not connected to any component of $q$, we have $d_X(s_-, s_+) \leq C$. Let $C = LM(1 + \lambda(2 \epsilon + 1) + c + 2 \epsilon)$, where $\varepsilon = \varepsilon(\lambda, \epsilon, k)$ is the constant from Theorem 2.35 (a). Let $s$ be a component of $p$ which is not connected to any component of $q$. By Theorem 2.35 (a), we have that the $X$-Hausdorff distance between the set of phase vertices is at most $\varepsilon$ (and we assume that all vertices of $p,q$ are phase, as usual), so there exist vertices $w_1, w_2$ on $q$ such that $d_X(s_-, w_1) \leq \varepsilon$ and $d_X(s_+, w_2) \leq \varepsilon$. 
	
	Since $q$ is a $(\lambda, c)$ quasi-geodesic, we may estimate: 
	
	\begin{align*}
	\ell([w_1, w_2]_{q}) &\leq \lambda d_{X \cup \mathcal{H}}(w_1, w_2) + c \\
	&\leq \lambda (d_{X \cup \mathcal{H}}(w_1, s_-) + d_{X \cup \mathcal{H}}(s_-, s_+) + d_{X \cup \mathcal{H}}(s_+, w_2)) + c \\
	&\leq \lambda(2 \varepsilon + 1) + c
	\end{align*}
	
	Connecting $s_-$ to $w_1$ via a geodesic $r_1$ in $\Gamma(G;X)$ and $s_+$ to $w_2$ via a geodesic $r_2$ in $\Gamma(G;X)$ (i.e. geodesic with edges labeled only by elements of $X$). Note that $\ell(r_1) = d_X(s_-, w_1) \leq \varepsilon$ and $\ell(r_2) = d_X(s_+, w_2) \leq \varepsilon$. We note that $s$ is isolated in the cycle $\omega = sr_2 [w_1, w_2]_q^{-1} r_1^{-1}$ (see figure 60 below), as it is not connected to any component of $q$ and it cannot be connected to any component of $r_1$ or $r_2$ because they do not have any components (being geodesics in $\Gamma(G;X)$). By Lemma 2.31 we then have $d_X(s_-, s_+) \leq ML \ell(\omega) = ML (\ell(s) + \ell(r_1) + \ell(r_2) + \ell([w_1, w_2]_q)) \leq ML(1 + 2 \varepsilon + \lambda(2 \varepsilon + 1) + c) = C$, as required.  $\square$
	
	
\begin{figure} [H]
	\centering
	\includegraphics[width=0.7\linewidth]{"../Desktop/Honours Project/IMG_1529"}
	\caption{The proof of Theorem 2.35 (b) using Osin's definition of relative hyperbolicity}
	\label{fig:img1529}
\end{figure}
	
	The BCP now follows immediately from Theorem 2.35 (b). Indeed, consider a $(\lambda, c)$ quasi-geodesic bigon $\alpha \beta^{-1}$ where $\alpha, \beta$ do not backtrack and let $s$ be an isolated component in this bigon. Note then that $\alpha$ and $\beta$ are 0-similar $(\lambda, c)$ quasi-geodesics without backtracking, so by Theorem 2.35 (b), we have $d_X(s_-, s_+) \leq C(\lambda, c, 0)$, where $C(\lambda, c, 0)$ is the constant from Theorem 2.35 (b). Note that $C(\lambda, c, 0)$ does not depend on the quasi-geodesics $\alpha, \beta$, only depending on the quasi-geodesic constants $\lambda, c$ as well as the group presentation. Therefore, $C(\lambda, c, 0)$ is the desired constant in the BCP. From this, we conclude that Osin's definition implies Farb's definition for finitely generated groups. Earlier in this section, we also showed that Farb's definition implies Osin's definition. Thus, we conclude the following: 
	
	\vs 
	
	\textbf{Theorem 2.42 (Equivalent definitions of relative hyperbolicity)}: Let $G$ be a finitely generated group and $(H_{\lambda})_{\lambda \in \Lambda}$ a finite collection of subgroups of $G$, with $\mathcal{H} = \sqcup_{\lambda \in \Lambda} H_{\lambda}$. The following are equivalent: 
	
	\begin{enumerate} [label = (\alph*)]
		\item $G$ is hyperbolic relative to $(H_{\lambda})_{\lambda \in \Lambda}$ in the sense of Farb (that is, for some (equivalently, any) finite generating set $X$, the coned-off Cayley graph $\hat{\Gamma}$ is hyperbolic and the relative Cayley graph $\Gamma(G; X \sqcup \mathcal{H})$ satisfies the BCP). 
		\item $G$ is hyperbolic relative to $(H_{\lambda})_{\lambda \in \Lambda}$ in the sense of Osin (that is, there is a finite relative presentation $\langle X, H_{\lambda} \vert R \rangle$ of $G$, such that the relative Cayley graph $\Gamma(G; X \sqcup \mathcal{H})$ is hyperbolic and the relative Dehn function for the presentation $\langle X, H_{\lambda} \vert R \rangle$ is linear).
	\end{enumerate}
	
	\vs
	
	As an application of this result and hearkening back to Example 2.5, we shall show that the free product of two hyperbolic groups $G,H$ is hyperbolic if and only if $G,H$ are hyperbolic, and that for a hyperbolic group $H$, $\mathbb{Z}^2 * H$ is hyperbolic relative to $\mathbb{Z}^2$ using relative Dehn functions. 
	
	\vs 
	
	Suppose that $G,H$ are hyperbolic, given by finite (non-relative) presentations: $G = \langle X_1 \vert R_1 \rangle$ and $H = \langle X_2 \vert R_2 \rangle$. Then by the first example in Example 2.5, each of $G,H$ are hyperbolic relative to the trivial subgroup. Thus, the finite relative presentations $\langle X_1, \{1\} \vert R_1 \rangle, \langle X_2, \{1\} \vert R_2 \rangle$ have linear relative Dehn functions $f_1, f_2$, respectively. We claim that the finite relative presentation $\langle X_1 \cup X_2, \{1\} \vert R_1 \cup R_2 \rangle$ for $G * H$ has linear relative Dehn function $f = f_1 + f_2$. Indeed, let $w$ be a word in $(X_1 \cup X_2 \cup \{1\})^*$ representing 1 in $G * H$ of length at most $n$. We can decompose $w$ as an alternating product $w = g_1 g_2 ... g_m$ where the $g_i$ alternate between words over $X_1, X_2$ and $1$. Note that by the normal form theorem for free products each $g_i$ is 1 in $G$ or $H$. We denote the length of each $g_i$ (wrt $X_1$ or $X_2$, depending on which alphabet $g_i$ belongs to) by $n_i$. Note that $\sum n_i \leq n$. Then, $\exists k_i \in \mathbb{N}$ such that each $g_i$ can be written as the sum of at most $k_i$ products of conjugates of relators from $R_1$ or $R_2$ (depending on whether $g_i$ is a word over $X_1$ or $X_2$) and $k_i \leq f_{1,2}(k_i) \leq f(k_i)$. Plugging in these expressions of $g_i$ as products of conjugates of relators from $R_1$ or $R_2$, we obtain that $w$ can be written as a product of $\sum k_i$ conjugates of elements of $R_1 \cup R_2$, and $\sum k_i \leq \sum f(n_i) = f (\sum n_i) \leq f(n)$, as required. 
	
	\vs 
	
	Conversely, suppose that one of $G,H$ is not hyperbolic. WLOG, suppose $G$ is not hyperbolic. Then $G$ does not have a linear relative Dehn function, so either its relative Dehn function is undefined or it is super-linear (i.e. larger than $id_{\mathbb{N}}$). Then there exists an $n \in \mathbb{N}$ and words $w$ over $X_1$ of length at most $n$ representing 1 in $G$ such that $k$ (the number of terms in the expression of $w$ as a product of relators from $R_1$) cannot be bounded above by a linear function of $n$. But then such $w$ are also words over $X_1 \cup X_2 \cup \{1\}$ of length at most $n$ representing 1 in $G * H$ for which $k$ cannot be bounded above by a linear function of $n$. Therefore, $G * H$ cannot have a linear relative Dehn function, so it is not hyperbolic. 
	
	\vs 
	
	Now we show that $\mathbb{Z}^2 * H$ is hyperbolic relative to $H$ if $H$ is hyperbolic. Let $H = \langle X \vert R \rangle$ be a finite presentation of $H$ for which $H$ has linear relative Dehn function $f$. A finite relative presentation for $\mathbb{Z}^2 * H$ is $\langle X, \mathbb{Z}^2 \vert R \rangle$. Consider a word $w$ in $(X \cup \mathbb{Z}^2)^*$ representing 1 in $H * \mathbb{Z}^2$. Then $w$ is an alternating product of elements of $X$ and $\mathbb{Z}^2$, as above. By the normal form theorem for free products, we have each word is 1 in its respective free factor. For letters over $X$, we can bound the total number of terms in their expression as a product of conjugates of relators from $R$ by $f(n)$. For the words over $\mathbb{Z}^2$, since we have access to all of $\mathbb{Z}^2$ as generators in $F(X) * \mathbb{Z}^2$, they simply vanish as 1 and are not expressed as products of relators. Thus, we have $w$ expressed as a product of $\sum k_i \leq \sum f(n_i) = f (\sum n_i) \leq f(n)$ conjugates of relators from $R$, as required. 
	
	\vs 
	
	\underline{Symmetric Geodesics}
	
	\vs 
	
	The following concept will be useful in the study of the conjugacy problem: 
	
	\vs 
	
	\textbf{Definition 2.43: } Two paths $p,q$ in $\Gamma(G; X \cup \mathcal{H})$ are called \textit{symmetric} if $\phi(p) = \phi(q)$ as words in $(X \cup \mathcal{H})^*$. Given symmetric geodesics $p,q$, we define elements $g_1 = (p_{-})^{-1} q_{-}$ and $g_2 = (p_{+})^{-1}q_{+}$. We call $g_1, g_2$ \textit{characteristic elements} of $p,q$.
	
\begin{figure} [H]
	\centering
	\includegraphics[width=0.7\linewidth]{"../Desktop/Honours Project/IMG_1476"}
	\caption{Symmetric Geodesics}
	\label{fig:img1476}
\end{figure}
	
	\vs 
	
	\textbf{Remark 2.44: } Elements $g_1, g_2 \in G$ are conjugate if and only if $g_1, g_2$ are characteristic elements of a pair of symmetric geodesics $p,q$. Indeed, if $g_2 = t^{-1} g_1 t$, then taking a geodesic path $p$ in $\Gamma(G; X \cup \mathcal{H})$ to be a geodesic from 1 to $t$ and letting $q$ be a geodesic from $g_1$ to $g_1 t$. Then both $p,q$ have label $\phi(p) = \phi(q) = t$ and we have $(p_{-})^{-1} q_{-} = g_1$ and $(p_{+})^{-1} q_{+} = t^{-1} g_1 t = g_2$. Therefore, $g_1, g_2$ are characteristic elements of $p,q$. Conversely, if we have a pair of symmetric geodesics $p,q$ with $\pi(\phi(p)) = \pi(\phi(q)) = t$ and if $g_1, g_2$ are characteristic elements of $p,q$, then we have $p_{+} = p_{-} t$ and $q_{+} = q_{-} t$, so $g_2 = (p_{+})^{-1}q_{+} = (p_{-} t)^{-1} q_{-} t = t^{-1} (p_{-})^{-1} q_{-} t = t^{-1} g_1 t$, so $g_2, g_1$ are conjugate.  
	
	\vs 
	
	We can think of symmetric paths as being parallel in some sense. Given this interpretation, we can think of pairs of points being synchronous along the symmetric paths. 
	
	\vs 
	
	\textbf{Definition 2.45: } Given symmetric paths $p,q$, vertices $a$ on $p$ and $b$ on $q$ are called \textit{synchronous} if $\ell([p_{-}, a]_p) = \ell([q_{-}, b]_q)$, where $[p_{-}, a]_p$ and $[q_{-}, b]_q$ are subpaths along $p,q$, respectively. Similarly, components $s$ of $p$ and $t$ of $q$ are called \textit{synchronous components} if $\ell([p_{-}, s_{-}]_p) = \ell([q_{-}, t_{-}]_q)$. 
	
\begin{figure} [H]
	\centering
	\includegraphics[width=0.7\linewidth]{"../Desktop/Honours Project/IMG_1477"}
	\caption{Synchronous points $u,v$}
	\label{fig:img1477}
\end{figure}
	
	\vs 
	
	We will prove several results about symmetric geodesics that will be useful in our later discussion on the conjugacy problem. 
	
	\vs 
	
	The first lemma is a fact about general hyperbolic metric spaces that we will apply to $\Gamma(G; X \sqcup \mathcal{H})$. 
	
	\vs 
	
	\textbf{Lemma 2.46: } Let $(X,d)$ be a $\delta$-hyperbolic metric space. For all $k \geq 0$, there exists a constant $E = E(k)$ such that the following holds: for any pair of $k$-similar geodesics $p,q$ in $X$, and for any pair of synchronous points $u$ on $p$ and $v$ on $q$, we have $d(u,v) \leq E$. 
	
\begin{figure} [H]
	\centering
	\includegraphics[width=0.7\linewidth]{"../Desktop/Honours Project/IMG_1483"}
	\caption{The statement of Lemma 2.46}
	\label{fig:img1483}
\end{figure}
	
	\begin{proof}
		
		Let $p,q$ be $k$-similar geodesics (whose endpoints we denote by $p_{-}, p_{+}, q_{-}, q_{+}$) and let $u,v$ be synchronous points on $p$, $q$, respectively. Without loss of generality, we suppose that $d(p_{-}, u) = d(q_{-}, v)$. By Lemma 1.24, we have that $d_{\text{Haus}}(p,q) \leq 2\delta + k$. Therefore, there exists a point $z$ on $q$ such that $d(u,z) \leq 2\delta + k$. We consider two cases on where the point $z$ can lie. 
		
		\vs 
		
		\underline{Case 1:} $z$ is on $[q_{-}, v]$ (where $[q_{-}, v]$ denotes the segment along $q$). Let $t = d(v,z)$. As $d(p_{-}, u) = d(q_{-}, v)$, there exists a point $z'$ on $[p_{-}, u]$ such that $d(z', u) = t$. Our goal is to bound $t$ from above by a constant depending only on $\delta, k$. Let $s = d(z, q_{-})$. We then have $s + t = d(u, p_{-}) \leq 2 \delta + k + s + k \implies t \leq 2(\delta + k)$, where we have used the triangle inequality to bound $d(u, p_{-})$ from above, following the path shown in figure 64 below. Therefore, using the triangle inequality once more, we obtain $d(u,v) \leq d(u,z) + d(z,v) \leq 2\delta + k + t \leq 2\delta + k + 2(\delta + k) = 4\delta + 3k$. 
		
		
\begin{figure} [H]
	\centering
	\includegraphics[width=0.7\linewidth]{"../Desktop/Honours Project/IMG_1500"}
	\caption{Case 1 of the proof of Lemma 2.46}
	\label{fig:img1500}
\end{figure}
		
		\vs 
		
		\underline{Case 2: } $z$ is on $[v, q_{+}]$. We consider three subcases. 
		
		\begin{itemize}
			\item First, suppose that $d(z, v) > d(u, p_{+})$. Let $t = d(z,v)$. Let $w$ be the point on $[v, q_{+}]$ such that $d(v, w) = d(u, p_{+})$. Note that $z$ is on $[w, q_{+}]$ because $d(z,v) = t > d(u, p_{+}) = d(v,w)$. We also note that $d(w, q_{+}) = \ell(q) - \ell(p) \leq 2k$. By the triangle inequality, we have $d(v,w) = d(u, p_{+}) \leq d(u, z) + d(z, q_{+}) + d(q_{+}, p_{+}) \leq 2 \delta + k + 2k + k = 2 \delta + 4k$. Then $t \leq d(v, q_{+}) = d(v,w) + d(w, q_{+}) \leq 2 \delta + 4k + 2k = 2 \delta + 6k$. Therefore, $d(u,v) \leq d(u,z) + d(z,v) = d(u,z) + t \leq 2 \delta + k + 2 \delta + 6k = 4 \delta + 7k$.
			
\begin{figure}[H]
	\centering
	\includegraphics[width=0.7\linewidth]{"../Desktop/Honours Project/IMG_1501"}
	\caption{The first subcase of case 2 in Lemma 2.46}
	\label{fig:img1501}
\end{figure}
			\item Next, suppose that $d(z, v) \leq d(u, p_{+})$ but $d(u, p_{+}) < d(v, q_{+})$. Let $w$ be the point on $[v, q_{+}]$ such that $d(v, w) = d(u, p_{+})$. Note thta $z$ is on $[v, q_{+}]$. As in the above case, we have that $d(w, q_{+}) \leq 2k$, so $d(w, p_{+}) \leq 3k$. Let $s = d(z, w)$. Then $d(u, p_{+}) = t + s \leq 2\delta + k + s + 3k \implies t \leq 2 \delta + 4k$. Hence, $d(u,v) \leq d(u,z) + d(z,v) \leq 2 \delta + k + 4 \delta + 2k = 6 \delta + 3k$. 
			
\begin{figure} [H]
	\centering
	\includegraphics[width=0.7\linewidth]{"../Desktop/Honours Project/IMG_1502"}
	\caption{The second subcase of case 2 in Lemma 2.46}
	\label{fig:img1502}
\end{figure}

			\item Lastly, suppose $d(u, p_{+}) \geq d(v, q_{+})$. We let $z'$ be a point on $[u, p_{+}]$ such that $d(u, z') = d(v,z) =:t$ and let $w'$ be a point on $[u, p_{+}]$ such that $d(u, w') = d(v, q_{+})$ and let $s = d(u, w')$. We note that $d(w', p_{+}) = \ell(p) - \ell(q) \leq 2k$, so $d(w', q_{+}) \leq 3k$, by the triangle inequality. Thus, $t + s = d(u, w') \leq d(u, z) + d(z, q_{+}) + d(q_{+}, w') \leq 2\delta + k + s + 3k \implies t \leq 2 \delta + 4k$. Thus, $d(u,v) \leq d(u,z) + d(z, v) \leq 2 \delta + k + 2 \delta + 4k = 4 \delta + 6k$.  
			
\begin{figure} [H]
	\centering
	\includegraphics[width=0.7\linewidth]{"../Desktop/Honours Project/IMG_1503"}
	\caption{The third subcase of case 2 in Lemma 2.46}
	\label{fig:img1503}
\end{figure}
		\end{itemize}
	
		Thus, we set $E = \max \{4 \delta + 3k, 4 \delta + 7k, 6 \delta + 3k, 4 \delta + 6k\} = \max \{4 \delta + 7k, 6 \delta + 3k\}$. 
		
	\end{proof}

	\vs 
	
	Next, we establish a lemma that tells us of the nature of connected components of minimal symmetric geodesics as well as a distinctness of connectors for distinct pairs of synchronous vertices. 
	
	\vs 
	
	\textbf{Lemma 2.47: } Let $(p,q)$ be a minimal pair of symmetric geodesics in $\Gamma(G; X \sqcup \mathcal{H})$. Then the following hold: 
	
	\begin{enumerate}[label = (\roman*)]
		\item Suppose two $H_i$ components $a,b$ of $p,q$ are connected. Then $a,b$ are synchronous. 
		\item Suppose $u_1, v_1, u_2, v_2$ are two distinct pairs of synchronous vertices with $u_1, u_2$ on $p$ and $v_1, v_2$ on $q$. Then $u_1^{-1}v_1 \neq u_2^{-1}v_2$.
	\end{enumerate}

	\begin{proof}
		
		We prove (i) first, proceeding by contradiction. Suppose that $a,b$ are connected components, but not synchronous. Then we may write $p = p_1 a p_2$ and $q = q_1 b q_2$ where $\ell(p_1) \neq \ell(q_1)$. We assume, without loss of generality, that $\ell(q_1) < \ell(p_1)$. We have a configuration as below: 
		
\begin{figure} [H]
	\centering
	\includegraphics[width=0.7\linewidth]{"../Desktop/Honours Project/IMG_1484"}
	\caption{Arrangement in (i) of Lemma 2.47}
	\label{fig:img1484}
\end{figure}
		
		\vs
		
		Since the cycles $c_1, c_2$ are closed loops, their labels $\phi(c_1), \phi(c_2)$ represent 1 in $G$. Therefore, by the van Kampen theorem, there exist van Kampen diagrams $\Delta_1, \Delta_2$ over the relative presentation $\langle \mathcal{H}_{\lambda}, \lambda \in \Lambda \vert R \rangle$ such that $\phi(\Delta_i) = \phi(c_i)$ for each $i$. 
		
		\vs 
		
\begin{figure} [H]
	\centering
	\includegraphics[width=0.7\linewidth]{"../Desktop/Honours Project/IMG_1485"}
	\caption{van Kampen diagrams for cycles $c_1, c_2$ in Lemma 2.47}
	\label{fig:img1485}
\end{figure}
		
		We notice that $\Delta_1, \Delta_2$ share the common edge $e$, so we may identify $\Delta_1, \Delta_2$ along $e$ to get a new diagram $\Delta$ with $\phi(\Delta) = r_1^{-1} p r_2 q^{-1}$. 
		
\begin{figure} [H]
	\centering
	\includegraphics[width=0.7\linewidth]{"../Desktop/Honours Project/IMG_1486"}
	\caption{Identifying the common edges of van Kampen diagrams for $c_1, c_2$}
	\label{fig:img1486}
\end{figure}
		
		Next, we identify $p,q$ in $\Delta$ to get another diagram $\Psi$
		
		
\begin{figure} [H]
	\centering
	\includegraphics[width=0.7\linewidth]{"../Desktop/Honours Project/IMG_1487"}
	\caption{Identifying $p,q$ to get a new van Kampen diagram $\Psi$}
	\label{fig:img1487}
\end{figure}
		
		\vs 
		
		Next we cut $\Psi$ along the path $q_1 e p_2$ to obtain a new diagram $\Sigma$. The cutting along the specified path makes the cut path part of the boundary of $\Sigma$, and so we have: $\phi(\partial \Sigma) = \phi(r_1)^{-1} U \phi(r_2) U^{-1}$ where $U = \phi(q_1 e p_2)$. 
		
		\vs 
		
		Letting $g_1 = \pi(\phi(r_1))$ and $g_2 = \pi(\phi(r_2))$, we note that $g_2 = \pi(U)^{-1} g_1 \pi(U)$ because $1 = \pi(\phi(\Sigma)) = \pi(\phi(r_1)^{-1} U \phi(r_2) U^{-1}) = \pi(\phi(r_1))^{-1} \pi(U) \pi(\phi(r_2)) = g_1^{-1} \pi(U)^{-1} g_2 \pi(U)$, so $g_1 = \pi(U)^{-1} g_2 \pi(U)$. 
		
		\vs 
		
		By Remark 2.44, the equality $g_1 = \pi(U)^{-1} g_2 \pi(U)$ gives us that $g_1, g_2$ are characteristic elements for a symmetric pair of geodesics $p', q'$ such that $\phi(p') = \phi(q') = U$. However, $g_1, g_2$ are also characteristic elements for $p,q$ and $\ell(p') = \vert \vert U \vert \vert = \ell(q_1) + \ell(e) + \ell(p_2) < \ell(p_1) + 1 + \ell(p_2) = \ell(p)$, a contradiction to the minimality of $p,q$. 
		
		\vs 
		
		We now prove (ii). We again proceed by contradiction. Suppose that $u_1^{-1} v_1 = u_2^{-1} v_2 =: w$ for distinct pairs of synchronous vertices $(u_1, v_1), (u_2, v_2)$. Write $p = t_1 t_2 t_3$ and $q = t_1' t_2' t_3'$. Write $f_1 = \pi(\phi(t_1)) = \phi(\phi(t_1'))$ and $f_3 = \pi(\phi(t_3)) = \pi(\phi(t_3'))$. 
		
\begin{figure} [H]
	\centering
	\includegraphics[width=0.7\linewidth]{"../Desktop/Honours Project/IMG_1488"}
	\caption{The setup of the proof of Lemma 2.47 (ii)}
	\label{fig:img1488}
\end{figure}

		\vs 
		
		Let $g_1, g_2$ be characteristic elements for $p,q$. Note that $g_2  = f_3^{-1} w f_3$ and $g_1 = f_1 w f_1^{-1}$, so $w = f_1^{-1} g_1 f_1$. Then $g_2 = (f_1 f_3)^{-1} g_1 (f_1 f_3)$. However, $\vert f_1 f_3 \vert_{X \cup \mathcal{H}} \leq \vert f_1 \vert_{X \cup \mathcal{H}} + \vert f_3 \vert_{X \cup \mathcal{H}} < \ell(p)$ (as $\ell(t_2) > 0$, since $u_1 \neq u_2$). Therefore, $g_1, g_2$ are characteristic elements for a pair of symmetric geodesics with length shorter than $p$, contradicting the minimality of the pair $(p,q)$.   
		
	\end{proof}
	
	The next lemma we prove tells us that synchronous vertices far away from the endpoints of minimal symmetric geodesics are close. 
	
	\vs 
	
	\textbf{Lemma 2.48: } Let $(p,q)$ be a minimal pair of symmetric geodesics in $\Gamma(G; X \sqcup \mathcal{H})$ such that
	\vskip1pt$\max\{d_{X \cup \mathcal{H}}(p_{-}, q_{-}), d_{X \cup \mathcal{H}}(p_{+}, q_{+})\} \leq k$. Let $u,v$ be synchronous vertices on $p,q$ such that \vskip1pt$\min \{d_{X \cup \mathcal{H}}(p_{-}, u), d_{X \cup \mathcal{H}}(u, p_{+})\} \geq 2 E$, where $E = E(k)$ is the constant from Lemma 2.46. Then $d_X(u,v) \leq 6 ML E^2 k$. 
	
\begin{figure} [H]
	\centering
	\includegraphics[width=0.6\linewidth]{"../Desktop/Honours Project/IMG_1489"}
	\caption{The statement of Lemma 2.48}
	\label{fig:img1489}
\end{figure}
	
	
	\begin{proof}
		
		Because $\min \{d_{X \cup \mathcal{H}}(p_{-}, u), d_{X \cup \mathcal{H}}(u, p_{+})\} \geq 2 E$, there exist distinct points $u_1, u_2$ on $p$ such that $d_{X \cup \mathcal{H}}(u_i, u) = 2 E$ for each $i = 1,2$. Take $u_1$ to be the point to left of $u$ and $u_2$ to be the point to the right of $u$. Define points $v_1, v_2$ on $q$ analogously. 
		
		\vs 
		
		We let $o_i$ denote connecting geodesics from $u_i$ to $v_i$ for each $i = 1,2$ and similarly we connect $u,v$ with a geodesic $o$. Define cycles $c_1, c_2$ as shown in figure 74 below. 
		
\begin{figure} [H]
	\centering
	\includegraphics[width=0.7\linewidth]{"../Desktop/Honours Project/IMG_1490"}
	\caption{The segments $o_i$ and cycles $c_i$ in the proof of Lemma 2.48}
	\label{fig:img1490}
\end{figure}
		
		We note that every component of $o$ is isolated in at least one of the cycles $c_1$ or $c_2$. Indeed, first we note that there can be no component of $o$ connected to any $o_i$ because if there was a component $s$ of $o$ connected to a component $t$ of $o_1$ (WLOG), we would have $d_{X \cup \mathcal{H}}(u,u_1) \leq d_{X \cup \mathcal{H}}(u, s_{-}) + d_{X \cup \mathcal{H}} (s_{-}, t_{-}) + d_{X \cup \mathcal{H}} (t_{-}, u_1) \leq d_{X \cup \mathcal{H}} (u,v) - 1 + 1 + \ell(o_1) - 1$ (we have $d_{X \cup \mathcal{H}} (u, s_{-}) \leq d_{X \cup \mathcal{H}} (u,v)$ and $d_{X \cup \mathcal{H}} (t_{-}, u_1) \leq \ell(o_1) - 1$ because $d_{X \cup \mathcal{H}} (s_{-}, s_{+}) = d_{X \cup t_{-}, t_{+}} = 1$). However, we have $d_{X \cup \mathcal{H}}(u,v), \ell(o_1) \leq E$, so we conclude with $d_{X \cup \mathcal{H}} (u,u_1) \leq 2E -1 < 2E$, a contradiction to the definition of the $u_i$. Therefore, components of $o$ can only possibly be connected to components of $p$ or $q$. Thus, if a component $s$ of $o$ is isolated neither in $c_1$ nor in $c_2$, this means that $s$ is connected either to a component $r_1$ of $p_1$ and a component $r_2$ of $q_2$, or to a component $r_1'$ of $p_2$ and a component $r_2'$ of $q_1$ (see figures 76 and 77 below).
		
		
\begin{figure} [H]
	\centering
	\includegraphics[width=0.7\linewidth]{"../Desktop/Honours Project/IMG_1491"}
	\caption{One of the possible cases of components connected to $s$ in the proof of Lemma 2.48}
	\label{fig:img1491}
\end{figure}

\begin{figure} [H]
	\centering
	\includegraphics[width=0.7\linewidth]{"../Desktop/Honours Project/IMG_1492"}
	\caption{The other possible case of components connected to $s$ in the proof of Lemma 2.48}
	\label{fig:img1492}
\end{figure}

		In either case, we obtain a connection between components of $p,q$ which are not synchronous (as one of the components is to the left of $o$ while the other to the right) and according to Lemma 2.47 (i), this is impossible. We therefore conclude that every component of $o$ is isolated in at least one of the cycles $c_1$ or $c_2$. 
		
		\vs 
		
		By Lemma 2.31, every component of $o$ has $X$-length bounded above $ML \max_i \ell(c_i)$. By Lemma 2.46, we know that $\ell(o) \leq E$ and $\ell(o_i) \leq E$ for each $i=1,2$. Therefore, for each $i$, we have $\ell(c_i) = d_{X \cup \mathcal{H}}(u_i, u) + \ell(o_i) + d_{X \cup \mathcal{H}} (v_i, v) + \ell(o) \leq 2E + E + 2E + E = 6E$. 
		
		\vs 
		
		Finally, we conclude that $d_X(u,v) \leq \ell(o) ML \max_i \ell(c_i) \leq E \cdot ML6E = 6MLE^2$. If $k \geq 1$, then we obtain $d_X(u,v) \leq 6 ML E^2 k$, and if $k = 0$, then as $(p,q)$ is a minimal pair of symmetric 0-similar geodesics, $p,q$ are both trivial paths (length 0), hence $d_X(u,v) = 0 = 6ML E^2 \cdot 0$. Therefore, in either case we see that $d_X (u,v) \leq 6 ML E^2 k$, as required. 
		
	\end{proof}

	As a corollary of the above results, we show that for any pair of conjugate elements, we can bound the length of any conjugating element between these elements in terms of a bound on the relative length of these elements. This will prove to be of considerable use in our discussion of the conjugacy problem. 
	
	\vs 
	
	\textbf{Corollary 2.49: } For any $k \geq 0$, there exists a constant $\rho = \rho(k)$ such that for any pair of conjugate elements $f,g \in G$ such that $\max \{\vert f \vert_{X \cup \mathcal{H}}, \vert g \vert_{X \cup \mathcal{H}}\} \leq k$, there exists $t \in G$ such that $f^t = g$ and $\vert t \vert_{X \cup \mathcal{H}} \leq \rho$. 
	
	\begin{proof}
		
		Let $E = E(k)$ be the constant given by Lemma 2.46. Let $p,q$ be a minimal pair of symmetric geodesics with characteristic elements $f,g$. Set $t = \pi(\phi(p))$. Note then that $f^t = g$. 
		
		\vs 
		
		If $\ell(p) \leq 4E$, then we are done, as then we can set $\rho = 4E$ to get $\vert t \vert_{X \cup \mathcal{H}} \leq \rho$. Therefore, we assume $\ell(p) > 4E$. In this case, there exists a subsegment $p_0$ of $p$ such that $d_{X \cup \mathcal{H}}(p_{-}, (p_0)_{-}) = d_{X \cup \mathcal{H}}((p_0)_{+}, p_{+}) = 2E$. Thus, for every vertex $u$ on $p_0$ we have $\min\{d_{X \cup \mathcal{H}}(p_{-}, u), d_{X \cup \mathcal{H}}(u, p_{+})\} \geq 2E$. Projecting each vertex $u$ on $p_0$ to its synchronous vertex $v$ on $q$, by Lemma 2.48, we have $d_X(u,v) \leq 6ML E^2 k$. Let $n = \ell(p_0)$. Then we have $n+1$ vertices $u_1,...,u_{n+1}$ on $p_0$, with their corresponding synchronous vertices $v_1,...,v_{n+1}$ on $q$. We have $\vert u_i^{-1} v_i \vert_X = d_X (u_i,v_i) \leq 6ML E^2 k$ and by Lemma 2.47 (ii), we have $u_i^{-1} v_i \neq u_j^{-1} v_j$ for $u_i \neq u_j$. Therefore, the map $\phi: \{u_1,...,u_{n+1}\} \rightarrow B_X(6MLE^2 k, 1)$ given by $\phi(u_i) = u_i^{-1} v_i$ is an injection and we conclude that $n + 1 \leq \vert B_X(6MLE^2 k, 1) \vert$. Note that $B_X(6MLE^2 k, 1)$ is finite because the generating set $X$ is finite. 
		
\begin{figure} [H]
	\centering
	\includegraphics[width=0.7\linewidth]{"../Desktop/Honours Project/IMG_1493"}
	\caption{An illustration of the proof of Corollary 2.49}
	\label{fig:img1493}
\end{figure}
		
		\vs 
		
		Therefore, we have $\vert t \vert_{X \cup \mathcal{H}} = \ell(p) = \ell(p_0) + 4E \leq \vert B_X(6MLE^2 k, 1) \vert + 4E - 1$. Thus, we set $\rho(k) = \vert B_X(6MLE^2 k, 1) \vert + 4E - 1$. 
		
	\end{proof}
	
	\subsection{Algebraic Properties of Relatively Hyperbolic Groups}
	
	As in the above sections, we assume here that we have a finitely generated group $G$ hyperbolic relative to the collection of subgroups $\{H_1,...,H_m\}$ with finite relative presentation $\langle X, H_1,...,H_m \vert R \rangle$. We present here some interesting algebraic properties of relatively hyperbolic groups, omitting the proof of a few results. Many of these properties will not have any applications to the algorithmic problems that we will discuss in the next section (though they do have important applications to algorithmic problems that we will not discuss in this paper, such as the order problem and the root problem), but we still present these results (and some proofs) because they are interesting in their own right. 
	
	\vs
	
	\underline{Algebraic Properties of elements of finite order}
	
	\vs 
	
	Let $G$ be a group hyperbolic relative to the collection of subgroups $\{H_{\lambda}\}_{\lambda \in \Lambda}$. We distinguish two types of elements: 
	
	\vs 
	
	\textbf{Definition 2.50: } An element $g \in G$ is called \textit{parabolic} if it is conjugate to an element of $H_{\lambda}$ for some $\lambda$. Otherwise, $g$ is called \textit{hyperbolic}. 
	
	\vs 
	
	It turns out that hyperbolic elements in relatively hyperbolic groups have only finitely many conjugacy classes. This is useful in solving the \textit{order} problem (see Osin [16]). 
	
	\vs 
	
	\textbf{Theorem 2.51: } Let $G$ be a group hyperbolic relative to subgroups $\{H_{\lambda}\}_{\lambda \in \Lambda}$. Then there are only finitely many conjugacy classes of hyperbolic elements of finite order. 
	
	\vs 
	
	We will prove this theorem by sketching the proof of a couple of lemmas. 
	
	\vs 
	
	The first lemma we will examine tells us that elements of finite order in $G$ are conjugate to elements of bounded length. 
	
	\vs 
	
	\textbf{Lemma 2.52: } Let $g \in G$ have finite order and let $\delta$ be a hyperbolicity constant of $\Gamma(G; X \sqcup \mathcal{H})$. Then $g$ is conjugate to $h \in G$ with $\vert h \vert_{X \cup \mathcal{H}} \leq 8 \delta + 1$. 
	
	\begin{proof}
		
		We may assume $g \neq 1$, as the result is clearly true if $g = 1$. We proceed by contradiction, assuming that no such element $h$ exists. Let $g_0$ be a shortest element in the conjugacy class $[g]$. Then $\vert g_0 \vert_{X \cup \mathcal{H}} > 8 \delta + 1$ because $g$ is not conjugate to any element of length at most $8 \delta + 1$. We fix a word $U$ in the free monoid $(X \cup \mathcal{H})^*$ of minimal length such that $\pi(U) = g_0$ (so $U$ satisfies $\vert \vert U \Vert \vert = \vert g \vert_{X \cup \mathcal{H}}$). For each $n \in \mathbb{N}$, we define a path $p_n$ with $(p_n)_- = 1$ and $\phi(p_n) = U^n$. Then each $p_n$ is a $8 \delta + 1$-local geodesic. Indeed, suppose $q$ is a subpath of $p_n$ with $\ell(q) \leq 8 \delta + 1$. Then because $\vert \vert U \vert \vert > 8 \delta + 1$, we have that $\phi(q)$ is a subword of a cyclic permutation $U'$ of $U$ (see picture below). Note that $\vert \vert U' \vert \vert = \vert \vert U \vert \vert $ and $\pi(U')$ is conjugate to $\pi(U)$, being cyclic permutations of each other in $G$. If $q$ were not a geodesic, then we could replace $q$ with a shorter path $q'$ with the same endpoints. This would yield a word $U''$ with $\pi(U'') = \pi(U') = \pi(U) = g_0$ but $\vert \vert U'' \vert \vert < \vert \vert U' \vert \vert = \vert \vert U \vert \vert$, contradicting the fact that $U$ is the shortest word over $X \cup \mathcal{H}$ representing $g_0$. Therefore, $q$ is a geodesic, as required. 
		
		\vs 
		
		As each $p_n$ is an $8 \delta +1$-local geodesic, by Lemma 1.31 (ii), $p_n$ is a $(\lambda, c)$-quasi geodesic for $(\lambda, c)$, with $\lambda \leq 3$, $c = 2 \delta$. Therefore, we have $\ell(p_n) \leq \lambda d_{X \cup \mathcal{H}}((p_n)_-, (p_n)_+) + c = \lambda d_{X \cup \mathcal{H}}(1, g_0^n) + c = \lambda \vert g_0^n \vert_{X \cup \mathcal{H}} + c$. Note that $\ell(p_n) = n \vert \vert U \vert \vert = n \vert g_0 \vert_{X \cup \mathcal{H}}$, so we obtain: $n \vert g_0 \vert_{X \cup \mathcal{H}} \leq \lambda \vert g_0^n \vert_{X \cup \mathcal{H}} + c$ for each $n \in \mathbb{N}$. But if $m \in \mathbb{N}$ is the order of $g_0$, then for each $k \in \mathbb{N}$ we have $\vert g_0^{km} \vert_{X \cup \mathcal{H}} = 0$, so that $km \vert g_0 \vert_{X \cup \mathcal{H}} \leq c$ for all $k \in \mathbb{N}$. As $g_0 \neq 1$ (since $g \neq 1$, so $1 \notin [g]$), we have $\vert g_0 \vert_{X \cup \mathcal{H}} > 0$. Therefore, $\lim_{k \rightarrow \infty} km \vert g_0 \vert_{X \cup \mathcal{H}} = \infty$, a contradiction to above. 
		
\begin{figure} [H]
	\centering
	\includegraphics[width=0.7\linewidth]{"../Desktop/Honours Project/IMG_1531"}
	\caption{Proof of Lemma 2.52: The path $p_n$}
	\label{fig:img1531}
\end{figure}
		
	\end{proof}
	
	The next lemma gives us a bound on the $X$-length of minimal $X \cup \mathcal{H}$-length elements in the conjugacy class of hyperbolic elements. 
	
	\vs 
	
	\textbf{Lemma 2.53: } There exists $B \geq 0$ such that for any finite order hyperbolic $g \in G$ with $\vert g \vert_{X \cup \mathcal{H}}$ being minimal among all elements in $[g]$, we have $\vert g \vert_X \leq B \vert g \vert_{X \cup \mathcal{H}}^2$. 
	
	\begin{proof}
		
		We sketch a proof with the aid of diagrams. 
		
		\vs 
		
		We let $U \in (X \cup \mathcal{H})^*$ be a word of minimal length such that $\pi(U) = g$. Denote $n$ to be the order of $g$ and let $\ell = \vert g \vert_{X \cup \mathcal{H}}$. Let $p$ be any cycle in $\Gamma(G; X \sqcup \mathcal{H})$ such that $\phi(p) = U^n$. 
		
		\vs 
		
		We construct a subpath of $p_0$ of $p$ as follows: 
		
		\begin{enumerate}[label = (\roman*)]
			\item If $p$ has no backtracking, set $p = p_0$
			\item If $p$ backtracks, then we may decompose $p$ as $p = a s_1 b s_2 c$ where $s_1, s_2$ are adjacent connected components of $p$ (meaning that there are no components in between $s_1$ and $s_2$ which are connected to each other or connected to $s_1$ or $s_2$). Therefore, in this decomposition, $b$ does not backtrack and has no components connected to $s_1$ or $s_2$. We set $p_0 = b$. 
		\end{enumerate}
		
\begin{figure} [H]
	\centering
	\includegraphics[width=0.7\linewidth]{"../Desktop/Honours Project/IMG_1532"}
	\caption{Proof of Lemma 2.53: Case (ii) in the construction of $p_0$ and the path $e$}
	\label{fig:img1532}
\end{figure}
		
		\vs
		
		Note that $p_0$ is a subword of a cyclic permutation of $U^n$, so we can write $\phi(p_0) = U_0^k V$ where $U_0$ is a cyclic permutation of $U$ and $\vert \vert V \vert \vert < n$. We note that $k > 0$ since if $k = 0$, then $\phi(p_0) = V$, contradicting the minimality of $\vert g \vert_{X \cup \mathcal{H}}$ in $[g]$. 
		
		\vs
		
		We then connect the endpoints of $p_0$ with a path $e$ which has $\ell(e) \leq 1$ (in case (i), $e$ is trivial and in case (ii) $e$ is a connector between endpoints of the components $s_1, s_2$). We denote $c = p_0 e^{-1}$ the resulting cycle. We note that $c$ has no backtracking since in case (i), $c = p_0$ and in case (ii), $p_0$ has no backtracking and no component of $p_0$ is connected to $s_1$ or $s_2$. 
		
		\vs 
		
		We now estimate the $X$-length of the syllables of $U$. If $W$ is an $H_j$ syllable of $U$, then as there are $k$ copies of cyclic permutations of $U$ in $\phi(p_0)$, there are at least $k$ copies of $W$ in $\phi(c)$. 
		
		\vs 
		
		Therefore, by Lemma 2.31, we have $k \vert \pi(W) \vert_X \leq ML \ell (c)$. Estimating $\ell(c)$, we have $\ell(c) = \ell(p_0) + \ell(e) \leq k \ell + \vert \vert V \vert \vert + 1 \leq k \ell + \ell - 1 + 1 = \ell (k+1)$. Thus, $k \vert \pi(W) \vert_X \leq ML \ell (k+1)$, so $\vert \pi(W) \vert_X \leq ML \ell \frac{k + 1}{k} \leq 2ML \ell$ ($\frac{k+1}{k} \leq 2$ because $k \geq 1$). Thus, having estimated the $X$-length of all syllables of $U$, we have $\vert \pi(U) \vert_X \leq \vert \pi(U) \vert_{X \cup \mathcal{H}} 2ML \ell = 2 ML \ell^2$. Thus, we set $B = 2ML$.
		
	\end{proof}

	Using the above lemmas, we now prove Theorem 2.51: 
	
	\vs 
	
	From each conjugacy class of a finite order hyperbolic element $g$, choose a $g_0 \in [g]$ of minimal length. By Lemma 2.53 we have $\vert g_0 \vert_X \leq B \vert g_0 \vert_{X \cup \mathcal{H}}^2$ for some $B$ independent of $g_0$ or $g$. By Lemma 2.52 we have $\vert g_0 \vert_{X \cup \mathcal{H}} \leq 8 \delta + 1$, so that $\vert g_0 \vert_X \leq B(8 \delta + 1)^2$. Therefore, we obtain a map from the set of all conjugacy classes of finite order hyperbolic elements into $B_X(1, B(8 \delta + 1)^2)$, given by $[g_0] \mapsto g_0$, where $g_0$ is a minimal length element in its conjugacy class. This map is injective since conjugacy classes are pairwise disjoint. In addition, $B_X(1, B(8 \delta + 1)^2)$ is finite because $X$ is finite. Therefore, the set of conjugacy classes of finite order hyperbolic elements is finite. $\square$
	
	\vs 
	
	\underline{Translation numbers}
	
	\vs 
	
	We next briefly discuss an important algebraic and geometric concept called the \textit{translation number}. 
	
	\vs 
	
	\textbf{Definition 2.54: } Given a group $G$ generated by a finite set $X$ relative to the collection of subgroups $\{H_{\lambda}\}_{\lambda \in \Lambda}$, we define the translation number of $x \in G$ by: $\tau^{\text{rel}}(x) = \lim_{n \rightarrow \infty} \frac{\vert x^n \vert_{X \cup \mathcal{H}}}{n}$. 
	
	\vs 
	
	Note that the limit in the above definition always exists. To show this, we prove a more general statement (known as Fekete's lemma). Let $(a_n)_{n=1}^{\infty}$ be a subadditive sequence, that is, $a_{n+m} \leq a_n + a_m$ for all $n,m \in \mathbb{N}$. Then $\lim_{n \rightarrow \infty} \frac{a_n}{n} = \inf_{n \in \mathbb{N}} \frac{a_n}{n}$ (the proof below is due to [12]). 
	
	\vs 
	
	Indeed, let $L = \inf_{n \in \mathbb{N}} \frac{a_n}{n}$ and let $\epsilon > 0$. Then $\exists n \in \mathbb{N}$ such that $\frac{a_n}{n} < L + \epsilon/2 \implies a_n < n(L + \epsilon/2)$. For any $m \geq n$, writing $m = qn + r$ for $q \in \mathbb{N}$ and $0 \leq r < n$, we have by subadditivity $a_m \leq q a_n + a_r$. Letting $b = \max \{a_1,...,a_n\}$, we have $\frac{a_m}{m} \leq \frac{qa_n + a_r}{m} \leq \frac{qa_n + b}{m}< \frac{q n(L + \epsilon/2) + b}{m}$. Since $\frac{q n(L + \epsilon/2) + b}{m} \rightarrow L + \epsilon/2$ as $m \rightarrow \infty$, we have $M \in \mathbb{N}$ such that for all $m' \geq M$, $\vert \frac{q n(L + \epsilon/2) + b}{m'} - (L+\epsilon/2) \vert < \epsilon/2 \implies \frac{q n(L + \epsilon/2) + b}{m'} < L + \epsilon$. Thus, for $m \geq \max \{n, M\}$, we have $\frac{a_m}{m} < L + \epsilon$. But also for any $m \in \mathbb{N}$, since $L = \inf_{n \in \mathbb{N}} \frac{a_n}{n}$ we have $\frac{a_m}{m} \geq L > L - \epsilon$. Thus, for any $m \geq \max \{n, M\}$, we have $L - \epsilon < \frac{a_m}{m} < L + \epsilon$, so $\lim_{n \rightarrow \infty} \frac{a_n}{n} = L$.
	
	\vs 
	
	Note that this proves the translation number limit exists and converges to $\inf_{n \in \mathbb{N}} \frac{\vert x^n \vert_{X \cup \mathcal{H}}}{n}$ because the function $n \mapsto \vert x^n \vert_{X \cup \mathcal{H}}$ is subadditive, by Proposition 1.33 (b) and simple induction.
	
	\vs 
	
	Some useful properties of translation numbers are outlined in the following lemma: 
	
	\vs 
	
	\textbf{Lemma 2.55: } For $g \in G$, we have the following: 
	
	\begin{enumerate} [label = (\roman*)]
		\item $\tau^{\text{rel}}(g) \leq \vert g \vert_{X \cup \mathcal{H}}$
		\item For any $n \in \mathbb{Z}$, $\tau^{\text{rel}}(g^n) = \vert n \vert \tau^{\text{rel}}(g)$.
	\end{enumerate}

	\begin{proof}
		
	(i): By Proposition 1.33 (b) and simple induction, for each $n \in \mathbb{N}$, we have $\vert g^n \vert_{X \cup \mathcal{H}} \leq n \vert g \vert_{X \cup \mathcal{H}}$. Therefore, $\tau^{\text{rel}}(g) \leq \lim_{n \rightarrow \infty} \frac{n \vert g \vert_{X \cup \mathcal{H}}}{n} = \vert g \vert_{X \cup \mathcal{H}}$. 
	
	\vs 
	
	(ii): If $n \geq 0$, we have $\tau^{\text{rel}}(g^n) = \lim_{m \rightarrow \infty} \frac{\vert (g^{n})^m \vert_{X \cup \mathcal{H}}}{m} = \lim_{m \rightarrow \infty} \frac{\vert g^{nm} \vert_{X \cup \mathcal{H}}}{m} = \lim_{m \rightarrow \infty} \frac{\vert g^{m} \vert_{X \cup \mathcal{H}}}{m/n} = n \lim_{m \rightarrow \infty} \frac{\vert g^{m} \vert_{X \cup \mathcal{H}}}{m} = n \tau^{\text{rel}}(g)$.
	
	\vs 
	
	If $n < 0$, then $n = -\vert n \vert$. Then $\tau^{\text{rel}}(g^n) = \tau^{\text{rel}}(g^{-\vert n \vert}) = \tau^{\text{rel}}((g^{-1})^{\vert n \vert}) = \vert n \vert \tau^{\text{rel}}(g^{-1})$ (by the $n \geq 0$ case) $= \vert n \vert \tau^{\text{rel}}(g)$ ($\tau^{\text{rel}}(g^{-1}) = \tau^{\text{rel}}(g)$ because $\vert g^{-1} \vert_{X \cup \mathcal{H}} = \vert g \vert_{X \cup \mathcal{H}}$, as shown in Proposition 1.33 (a)). 
		
	\end{proof}
	
	\vs 
	
	An interesting fact about translation numbers in relatively hyperbolic groups is that there is a uniform, positive lower bound on the translation length of infinite order hyperbolic elements. 
	
	\vs 
	
	\textbf{Theorem 2.56: } There exists $d > 0$ such that for all hyperbolic, infinite order $g \in G$ we have $\tau^{\text{rel}}(g) > d$. 
	
	\vs 
	
	Though interesting in its own right, we will not make use of this theorem in our paper and so we do not present its proof here. We refer the interested reader to Osin's paper [16] for the proof of this theorem and its applications. 
	
	\vs 
	
	\underline{Commutative Transitivity}
	
	\vs 
	
	Commutative transitivity is an interesting property of certain groups which has to do with the following relation on non-trivial group elements $a \sim b \iff [a,b] = 1$ (i.e. $a$ commutes with $b$). A \textit{commutative transitive (CT)} group is one in which the relation $\sim$ is a transitive relation (therefore making $\sim$ an equivalence relation on the set of non-trivial elements, since reflexivity and symmetry of $\sim$ always hold). Note that a group is CT if and only if the centralizer of every non-trivial element is abelian (for this reason, CT groups are also sometimes referred to as CA, for "Centralizer Abelian"). CT groups have many nice structural properties, and their theory is explored in papers such as [8] and [17]. Examples of CT groups include abelian groups and free groups. Interestingly, finitely generated, torsion-free relatively hyperbolic groups are also CT, as proved by Zhang in [19] by generalizing a theorem of Baumslag concerning the class of free groups to the class of finitely generated, toral (i.e. torsion-free) relatively hyperbolic groups. The proof of this result departs too greatly from the focus of this thesis, so we shall omit it, though this result will come in great use to us when we discuss the solution to the generalized conjugacy problem in the next section. 
		
		
		\newpage
	\subsection{Algorithmic Problems}
	
	We examine several algorithmic problems in relatively hyperbolic groups and construct algorithms which solve these problems. 
	
	\vs 
	
	\underline{Word Problem}: 
	
	\vs 
	
	We consider a recursively presented group $G = \langle X \vert R \rangle$ (i.e. $X,R$ are recursive sets, meaning that there exists a recursive algorithm which enumerates their elements). The \textit{word problem} asks if, given an arbitrary word $w = x_1,...,x_n$ in $G$, $x_i \in X^{\pm}$, whether or not $w = 1$ in $G$. It was shown by Novikov in [14] and Boone in [2] that for general finitely presented groups the word problem is unsolvable (that is, there exist finitely presented groups for which no algorithm can solve the word problem). However, for a finitely presented group $G$ hyperbolic relative to a collection of subgroups $\{H_1,...,H_m\}$ with each subgroup $H_i$ having solvable word problem, it turns out that the word problem is solvable in $G$ as well. We first need to prove a couple of simple lemmas that will be employed in our solution of the word problem. 
	
	\vs 
	
	\textbf{Lemma 2.57: } Let $F = F(S)$ be the free group on $S$ ($\vert S \vert < \infty$). Then $F$ has solvable word problem. 
	
	\begin{proof}
		
		Let $w = s_1 s_2 ... s_n, s_i \in S^{\pm}$ be a word in $F$. Then $w = 1$ if and only if $w$ is the empty word over $S^{\pm}$ (i.e. $n = 0$ in the above expression), as otherwise $w=1$ would induce a non-trivial relation among the free generators of $F$, contradicting the fact that $F$ is a free group having no non-trivial relations. Therefore, to determine whether $w=1$ in $F$, one simply has to freely reduce $w$ and check if it reduces to the empty word. Therefore, the word problem is solvable. 
		
	\end{proof}
	
	\vs 
	
	\textbf{Lemma 2.58: } Let $H_1,...,H_n$ be finitely presented groups with solvable word problem. Then the free product $H_1 * ... * H_n$ has solvable word problem. 
	
	\begin{proof}
		
	We show this for the case $n = 2$ and apply induction for general $n$. 
	
	\vs 
	
	When $n=2$, any non-trivial word $w$ in $G$ can be decomposed as $w = g_1 h_1 ... g_m h_m$ for some $m \geq 0$, where $g_i \in H_1$ ($g_1$ possibly empty) and $h_i \in H_2$ ($h_m$ possibly empty). If $m = 0$ then $w = 1$ as it is the empty word in this case, so we may assume that $m > 0$. We check whether some $g_i = 1$ for $i = 2,...,m$ in $H_1$ or $h_i = 1$ for some $i = 1,...,m-1$ in $H_2$ using the solution to the word problem in $H_1, H_2$. If one of these $g_i, h_i$ is 1, then we remove it and repeat the process. If we eliminate all of the $g_i, h_i$, then $w = 1$. Otherwise, if we find each such $g_i, h_i \neq 1$, then we conclude by the normal form theorem that $w \neq 1$. Therefore, the word problem is solvable in $H_1 * H_2$. 
	
	\vs 
	
	For general $n$, we apply induction and the solution for $n = 2$ to deduce that the word problem is solvable in $H_1 * ... * H_n$.
	
	\end{proof}
	
	\vs
	
	\textbf{Theorem 2.59: } Let $G$ be a finitely generated group hyperbolic relative to subgroups $\{H_1,...,H_m\}$ with a solution to the word problem being known in each subgroup $H_i$. Then the word problem is solvable in $G$. 
	
	\begin{proof}
		
		We make use of van Kampen diagrams and Dehn functions. Choose a finite generating set $X$ of $G$. We may also assume that $X$ is symmetrized (i.e. $X = X^{-1}$), since if $X$ is not symmetrized, we need only append finitely many generators and finitely many relators to the relative presentation to obtain a relative presentation $\langle X, H_1,...,H_m \vert R \rangle$ with symmetrized generating set $X$. We may also assume that the relative presentation is reduced (if it is not reduced, we make it reduced by contracting each $H_i$ syllable to a single letter from $H_i$ and this operation does not affect the presentation because the new relators are still the same elements in $F := F(X) * (*_{i=1}^m \tilde{H_i})$, $\tilde{H_i} \cong H_i$). For each $i$, we let $\Omega_i$, $\Omega$ be the sets introduced in the section on combinatorial properties of relatively hyperbolic groups. By Proposition 2.22, we have that $H_i = \langle \Omega_i \rangle$ for all $i$ and each $\Omega_i$ is finite. Let $\varphi_i : H_i \rightarrow \tilde{H_i}$ be a group isomorphism and let $\tilde{\Omega} = \cup_{i=1}^m \varphi_i(\Omega_i)$. Note that $\tilde{\Omega}$ is finite, being the finite union of finite sets $\varphi_i(\Omega_i)$. Now being the free product of $F(X)$ and the $\tilde{H_i}$, and since each $\tilde{H_i}$ is generated by $\varphi_i(\Omega_i)$, we have that $F = \langle X \cup (\cup_{i=1}^m \varphi_i(\Omega_i)) \rangle = \langle X \cup \tilde{\Omega} \rangle$. We denote $Z = X \cup \tilde{\Omega}$. Since $X$ and $\tilde{\Omega}$ are both finite, $Z$ is finite and so $F$ is finitely generated. 
		
		\vs 
		
		Now let $n \in \mathbb{N}$ and let $w$ be a word in $X^*$ with $\vert \vert w \vert \vert \leq n$. By the van Kampen lemma, $w = 1$ in $G$ if and only if there exists a van Kampen diagram $\Delta$ over $\langle X, H_1,...,H_m \vert R \rangle$ with $\phi(\partial \Delta) = w$. Given such a diagram $\Delta$, we claim that every edge of $\Delta$ is labeled by an element of $Z$. Indeed, if $e$ is an external edge then $e$ is part of $\partial \Delta$ which is labeled by $w$, a word in $X$, hence $e$ is a letter from $Z$, and if $e$ is an internal edge, then by Lemma 2.19 $e$ is an edge of the boundary of some $R$-cell, so $\phi(e) \in Z$ because $\phi(e)$ is either a letter from $X$ in which case $\phi(e) \in Z$ or $\phi(e)$ is a letter from some $\tilde{H_\lambda}$, in which case $\phi(e) \in \tilde{\Omega_\lambda} \subseteq \tilde{\Omega} \subseteq Z$. 
		
		\vs
		
		Next, we have $w = 1$ in $G$ if and only if $w = \Pi_{i=1}^k f_i^{-1} r_i f_i$ for $f_i \in F$, $r_i \in R$ and $k \leq \delta^{\text{rel}}(n)$. Note that each $f_i$ labels a path in $\Delta$ without self-intersection, therefore, $ \vert f_i \vert_Z \leq \vert \vert f_i \vert \vert \leq \vert E(\Delta) \vert \leq M \mathcal{N}_R(\Delta) + \ell(\partial \Delta) \text{ (by Corollary 2.20)} = M \delta^{\text{rel}}(n) + \vert \vert w \vert \vert \leq M \delta^{\text{rel}}(n) + n$. Hence, because $Z$ is finite, there are only finitely many possibilities for the elements $f_i$. 
		
		\vs 
		
		Therefore, in order to solve the word problem in $G$ we just need to check finitely many (bounded above by a recursive function of $n$ (a recursive function is a function from $\mathbb{N}$ to $\mathbb{N}$ whose values can be recursively enumerated)) equalities $w = \Pi_{i=1}^k f_i^{-1} r_i f_i$ in $F$. We can check each equality because $F$ is a free product of groups with solvable word problem, therefore $F$ has solvable word problem by Lemma 2.58. We conclude therefore that $G$ has solvable word problem. 
		
	\end{proof}

	\underline{Farb's solution to the word problem}
	
	\vs

	There is another approach to solving the word problem in relatively hyperbolic groups which is an algorithm that relies on the geometry of the relative Cayley graph rather than on van Kampen diagrams and Dehn functions. We describe this algorithm, given in Farb's paper [8]: 
	
	\vs
	
	In this subsection, we fix a finite generating set $X$ of $G$ and consider a finite collection of finitely generated subgroups $\{H_1,...,H_m\}$ of $G$ such that $G$ is hyperbolic relative to this collection. We denote $\Gamma =  \Gamma(G;X)$ and the coned-off Cayley graph relative to the collection $\{H_1,...,H_m\}$ by $\hat{\Gamma}$. We fix a hyperbolicity constant $\delta$ for $\hat{\Gamma}$. We assume that the word problem is solvable in each subgroup $H_i$. 
	
	\vs
	
	Farb's solution to the word problem involves employing a curve shortening algorithm which shortens closed loops in $\Gamma$. We begin by describing this algorithm. 
	
	\vs 
	
	\textit{The Curve Shortening Algorithm: } 
	
	\vs 
	
	We denote $C = C(1,2)$ to be the constant from the bounded coset penetration property and we let $k = 4\delta$. Let $p$ be a loop in $\Gamma$ based at the identity with label $w$ (so that $w = 1$ in $G$). 
	
	\vs 
	
	For each subgroup $H_i$, we let $\mathcal{Y}_i$ be a set of words in $F(X)$ representing the generators in $Y_i$. We define the following balls in the free groups $F(\mathcal{Y}_i)$ and $F(X)$: 
	
	\vs 
	
	$\mathcal{L}_1^i = \{z \in F(\mathcal{Y}_i): \vert z \vert_{\mathcal{Y}_i} \leq C\}$ and $\mathcal{L}_2^i = \{z \in F(X): \vert z \vert_X \leq (8 \delta-1)c\}$. We use these sets in a process called \textit{component reduction}. 
	
	\vs 
	
	\textit{Component Reduction: } For each $H_i$ component $h_i$ of $\phi(p)$, if $\pi(h_i) = \pi(\alpha)$ for some $\alpha \in \mathcal{L}_1^i$ (we can check this equality because the word problem is solvable in $H_i$), then we replace $h_i$ by $\alpha$ in $\phi(p)$ (where we write $\alpha$ as a word in $X$ in this replacement).  
	
	\vs 
	
	We consider the following cases. First, if $p$ is a $k$-local geodesic, then by Lemma 1.31 (i), we have that $p \subseteq N_{3 \delta}([p_-, p_+]) = N_{3 \delta}(1)$. It follows that $\ell(p) \leq 3 \delta$. Indeed, if $\ell(p) \geq 3 \delta + 1$ then $p$ would have a subpath $q$ of length $3 \delta + 1 \leq 4 \delta = k$ starting at the identity, hence $q$ would be a geodesic and we would have $d_{X \cup \mathcal{H}}(1, q_+) = 3 \delta + 1 > 3 \delta$, hence $p$ would leave $N_{3 \delta}(1)$, a contradiction. Thus, $\ell(p) \leq 3 \delta < k$, so $p$ is a geodesic and hence if $\pi(w) = 1$ then $p$ is the trivial path at 1, so that $w$ is the empty word. 
	
	\vs 
	
	If $p$ is not a $k$-local geodesic, then we show that we can reduce $p$ to a $k$-local geodesic as follows. As $p$ is not a $k$-local geodesic, there exists a maximal subpath $q$ of length at most $k$ which is not a geodesic (but any proper subpath of $q$ is a geodesic, by maximality). We note that $q$ is a (1,2)-quasi-geodesic (because $q$ is the concatenation of a geodesic and an edge of length 1, hence is a (1,2) quasi geodesic by Lemma 1.19) without backtracking (because if we had two connected components of $q$, then these components would have to be the first and last edges of $q$, otherwise they would be contained in a proper subpath of $q$ which is a geodesic, and we know geodesics do not backtrack, but then $q$ would be a path of length 1, hence a geodesic, a contradiction). We connect the endpoints of $q$ with a geodesic $\alpha$ having minimal possible $X$-length among all geodesics joining the endpoints of $q$. We claim that there are no connected components of $q$ and $\alpha$. 
	
	\vs
	
	Indeed, if we had a pair of connected components $s$ of $\alpha$ and $t$ of $q$ (as shown below in figures 81 and 82), then we must have $\ell([\alpha_-, s_-]) = \ell([q_-, t_-]_q)$ because if $\ell([\alpha_-, s_-]) < \ell([q_-, t_-]_q)$ then traveling along $[\alpha_-, s_-]$ and then along a connector to $t_+$ (see figure 81 below), we would obtain $d(\alpha_-, t_+) \leq \ell([\alpha_-, s_-]) + 1$. Now either $t$ is the last edge of $q$ or not. If $t$ is not the last edge of $q$, then $[q_-, t_+]_q$ is a proper subpath of $q$ and hence is a geodesic, but traveling along connectors as shown below, we find $d(q_-, t_+) \leq \ell([\alpha_-, s_-]) + 1 < \ell([q_-, t_-]_q) + 1 = \ell([q_-, t_+]_q)$, contradicting that $[q_-, t_+]_q$ is a geodesic. On the other hand, if $t$ is the last edge of $q$, then $s$ must also be the last edge of $\alpha$, otherwise connecting $t_+$ to $s_-$, we obtain a path of length 1 connecting $t_+$ to $s_-$, which is shorter than the path along $\alpha$ from $s_-$ to $t_+$, which has length at least 2 if $s$ is not the last edge of $\alpha$. Thus, as $s$ and $t$ are the last edges of their respective paths and are both $H_i$ components, in the loop $q\alpha^{-1}$ we combine $s$ and $t$ into a single component. Therefore, in the loop $q\alpha^{-1}$ we may assume that we do not have occurances of adjacent components $q, \alpha$ meeting at an endpoint. Conversely, if $\ell([q_-, t_-]_q) < \ell([\alpha_-, s_-])$, then traveling along $[q_-, t_-]_q$ then along a conenctor, we would obtain a path connecting the endpoints of $[\alpha_-, s_+]$ which is shorter than $\ell([\alpha_-, s_+])$, contradicting the fact that $[\alpha_-, s_+]$ is a geodesic. But now since $\ell([\alpha_-, s_-]) = \ell([q_-, t_-]_q)$ and $\ell(\alpha) < \ell(q)$, we must have $\ell([s_-, \alpha_+]) < \ell([t_-, q_+]_q)$. However, as noted above, we may assume that neither $s$ nor $t$ are the last components of $\alpha, q$, so $[t_-, q_+]_q$ is a geodesic, and so connecting $s_+$ to $t_-$, we obtain a shorter path than $[t_-, q_+]_q$ traveling along $\alpha$ to $s_+$ and then along the connector of length 1 to $t_-$, contradicting the fact that $[t_-, q_+]_q$ is a geodesic. We conclude therefore (after compressing possible endpoint components to a single component of the loop $q \alpha^{-1}$) that $\alpha, q$ have no connected components. Therefore, $\alpha, q$ form (1,2) quasi-geodesics with no backtracking sides and in which all components are isolated. By the BCP, we then have for each component $s$ of the cycle $\alpha q^{-1}$ that $d_X(s_-, s_+) \leq C$. Therefore, we have $\vert \pi(\phi(\alpha q^{-1})) \vert_X \leq C \vert \pi(\phi(\alpha q^{-1})) \vert_{X \cup \mathcal{H}} \leq C(k + k - 1) = C(8 \delta - 1)$. Therefore, $\pi(\phi(\alpha q^{-1})) \in \mathcal{L}_2^i$. Thus, replacing $q$ with $\alpha$ gives us a new path $p'$ with label $w'$ such that $\pi(w') = \pi(w)$. We perform this procedure for all such maximal non-geodesic subpaths. We continue this coset reduction procedure until we are left with a $k$-local geodesic, and then revert to the above case. 
	
	\vs 
	
	In this way, starting with a word $w$ in $G$, we take a path $p$ in $\hat{\Gamma}$ with $\phi(p) = w$ and apply the reduction procedure above. We see that at the end of the procedure, the path $p$ shortens to a $k$-local geodesic $p'$ with label $w'$ having $\pi(w') = \pi(w)$. As seen above, if $\pi(w') = 1$ then $w'$ is the empty word and also it is clear that if $w'$ is the empty word then $\pi(w') = 1$. Therefore, our original word $w$ reduces to the empty word if and only if $w' = 1$ in $G$ if and only if $w = 1$ in $G$, which solves the solves the word problem in $G$. 
	
	
\begin{figure} [H]
	\centering
	\includegraphics[width=0.7\linewidth]{"../Desktop/Honours Project/IMG_1525"}
	\caption{One possibility for connected components in Farb's solution to the word problem}
	\label{fig:img1525}
\end{figure}

\begin{figure} [H]
	\centering
	\includegraphics[width=0.7\linewidth]{"../Desktop/Honours Project/IMG_1526"}
	\caption{The other possibility for connected components in Farb's solution to the word problem}
	\label{fig:img1526}
\end{figure}


	\vs 

	\underline{Membership Problem}: 
	
	\vs 
	
	We next analyze a related problem to the word problem: the membership problem. The membership problem is the following: given a subgroup $H$ of a group $G$ and an element $g \in G$, decide whether $g \in H$. It turns out that the membership problem is solvable for each parabolic subgroup $H_i$ in a relatively hyperbolic group provided the word problem is solvable for every $H_i$. 
	
	\vs
	
	\textbf{Theorem 2.60: } Let $G$ be a group hyperbolic relative to finitely presented subgroups $\{H_1,...,H_m\}$. Suppose that the word problem is solvable in each $H_i$. Then the membership problem is solvable in each $H_i$. 
	
	\vs 
	
	To prove this, we first must introduce the concept of \textit{distortion function} of a subgroup. 
	
	\vs
	
	\textbf{Definition 2.61: } Let $G$ be a group generated by some finite set $X$ and let $H$ be a subgroup of $G$ generated by some finite set $Y$. Then the \textit{distortion function} of $H$ in $G$ with respect to generating sets $X,Y$ is the function $\Delta_H^G : \mathbb{N} \rightarrow \mathbb{N} $ defined by $\Delta_H^G (n) = \max \{ \vert x \vert_Y : x \in H, \vert x \vert_X \leq n\}$. The distortion function measures how much the word length distorts when we use the word length with respect to the subgroup generating set. If $\Delta_H^G \sim n$ then we say that the subgroup $H$ is \textit{undistorted} in $G$. 
	
	\vs 
	
	The following lemmas will be key to solving the membership problem. 
	
	\vs 
	
	\textbf{Lemma 2.62: } Let $G$ be a finitely generated group which is finitely presented relative to the collection of finitely generated subgroups $\{H_1,...,H_m\}$ and suppose that the relative Dehn function $\delta^{\text{rel}}$ is well-defined.  Then for any $i$, we have $\Delta_{H_i}^G \preccurlyeq \delta^{\text{rel}}$. 
	
	\begin{proof}
		
		Let $X$ be a finite generating set for $G$. Let $n \in \mathbb{N}$ and let $h \in H_i$ such that $\vert h \vert_X \leq n$. Let $w$ be a shortest over $X$ representing $h$. 
		
		\vs 
		
		In the relative Cayley graph $\Gamma(G; X \cup \mathcal{H})$, we form a cycle $c = pq^{-1}$ where $\phi(p) = w$ and $\phi(q) = h$. Then we see that $q$ is an isolated $H_i$ component of $c$ (because $p$ is labeled by a word over $X$, so there are no  $H_i$ components of $c$ other than $q$). Then applying Lemma 2.21, we obtain: $\vert h \vert_{\Omega_i} = \vert \pi(\phi(q)) \vert_{\Omega_i} \leq M \text{Area}^{\text{rel}}(c) \text{ (by Lemma 2.21)} \leq M \delta^{\text{rel}}( \vert \vert w \vert \vert + 1) = M \delta^{\text{rel}}( \vert h \vert_X + 1) \leq M \delta^{\text{rel}}(n + 1)$. Therefore, $\Delta_{H_i}^G (n) \leq M \delta^{\text{rel}}(n + 1)$ for all $n \in \mathbb{N}$. So,  $\Delta_{H_i}^G \preccurlyeq \delta^{\text{rel}}$.
		
	\end{proof}

	\vs 
	
	\textbf{Lemma 2.63: } Suppose $G$ is a finitely generated group with solvable word problem and $H$ is a finitely generated subgroup. Then the membership problem for $H$ is solvable if $\Delta_H^G$ is bounded above by a recursive function. [9]
	
	\begin{proof}
		
		Let $X$ be a finite generating set for $G$. Let $Y = \{y_1,...,y_m\}$ be a set of elements in $F(X)$ such that $\mathcal{Y} := \bar{Y} = \{\bar{y_1},...,\bar{y_m}\}$ generates $H$ and let $f(n)$ be a recursive function which bounds $\Delta_H^G(n)$ from above. Let $g \in G$ and let $n = \vert g \vert_X$. Then we note that $g \in H$ if and only if $g$ can be expressed as the product of at most $f(n)$ elements of $\bar{Y}$ (indeed, if $g \in H$, then as $\vert g \vert_X = n$, we have $\vert g \vert_{\mathcal{Y}} \leq \Delta_H^G(n) \leq f(n)$ so that $g$ can be expressed as a product of at most $f(n)$ elements and if $g$ can be written as the product of at most $f(n)$ elements of $\mathcal{Y}$, then clearly $g \in H$). Therefore, because $f$ is recursive, there exists an algorithm which can create a list of all products of elements of $\mathcal{Y}$ of length at most $f(n)$ and we apply the solution to the word problem in $G$ to check whether $g$ is equal to any of these elements (note that this list is finite because $\mathcal{Y}$ is finite, so there are only finitely many checks that need to be done). By what we remarked above, $g \in H$ if and only if $g$ is equal to one of these elements. Therefore, we can determine whether $g \in H$, and so the membership problem for $H$ is solvable.
		
	\end{proof}

	We note that the converse to the above lemma is true and its proof can be found in [9], though we will not need it. 
	
	\vs
	
	We are now ready to solve the membership problem: 
	
	\vs 
	
	\begin{proof}
		
		Since the word problem is solvable in each subgroup $H_i$, the word problem is solvable in $G$ by Theorem 2.59. Furthermore, by Lemma 2.62 for each $i$, $\Delta_{H_i}^G$ is bounded above by the recursive function $M \delta^{\text{rel}}$. By Lemma 2.63, we conclude that the membership problem is solvable for $H_i$.
		
	\end{proof}

	\underline{Parabolicity problems}
	
	\vs 
	
	We next turn to the study of \textit{parabolicity problems}. Given a group $G$ finitely presented relative to finitely generated subgroups $\{H_1,...,H_m\}$ and an element $g \in G$, the \textit{general} parabolicity problem asks whether $g$ is a parabolic element, that is, if $g$ is conjugate to an element of some $H_i$. The \textit{special} parabolicity problem asks whether $g \in G$ is conjugate to an element of a particular subgroup $H_i$. Under the assumption of solvability of certain algorithmic problems in the parabolic subgroups $H_i$, the general and special parabolicity problems are solvable in $G$. 
	
	\vs 
	
	\textbf{Theorem 2.64: } Let $G$ be a group hyperbolic relative to recursively presented subgroups $\{H_1,...,H_m\}$. Then the following hold: 
	
	\begin{enumerate}[label = (\alph*)]
		\item If the word problem is solvable in each $H_i$ then the general parabolicity problem is solvable in $G$. Moreover, there is an algorithm which finds for a given $g \in G$ a conjugating element $t \in G$ and $j \in \{1,...,m\}$ such that $g^t \in H_j$, if such $t,j$ exist. 
		\item If the conjugacy problem is solvable in each $H_i$, then the special parabolicity problem is solvable in $G$ for each $H_j$ (the conjugacy problem asks, given $g,h \in G$, whether $g,h$ are conjugate). Moreover, there is an algorithm which finds, for a given $g \in G$ and $j \in \{1,...,m\}$, a conjugating element $t \in G$ such that $g^t \in H_j$, if such $t$ exists. 
	\end{enumerate}

	We must prove a couple of auxiliary lemmas to prove the above theorem.
	
	\vs 
	
	The first lemma we will prove tells us where to look for conjugating elements of parabolic elements. 
	
	\vs 
	
	\textbf{Lemma 2.65: } Let $G$ be a finitely generated group hyperbolic relative to recursively presented subgroups $\{H_1,...,H_m\}$. Then there exists a recursive function $\sigma: \mathbb{N} \rightarrow \mathbb{N}$ such that the following holds. Given a parabolic $g \in G$ such that $\vert g \vert_X \leq k$, there exists $t \in G$ with $\vert t \vert_X \leq \sigma(k)$ and a $j \in \{1,...,m\}$ such that $g^t \in H_j$.
	
	\begin{proof}
		
		Let $\mathcal{P}$ denote the set of all symmetric geodesics which have $(g,h)$ as characteristic elements, where $h \in \cup_{i=1}^m H_i$. Note that $\mathcal{P}$ is non-empty because $g$ is parabolic, so $g$ is conjugate to some $h \in \cup_{i=1}^m H_i$ and we know that conjugate elements yield a pair of symmetric geodesics having these conjugate elements as the characteristic elements (cf. Remark 2.44). From $\mathcal{P}$ we choose a set of symmetric geodesics $(p,q)$ of minimal length in $\Gamma(G; X \sqcup \mathcal{H})$. Let $r$ be an edge labeled by $h$ connecting $p_+$ to $q_+$ and let $t = \pi(\phi(p)) = \pi(\phi(q))$. By Lemma 2.49 we have that $\vert t \vert_{X \cup \mathcal{H}} \leq \rho(k)$ for some recursive function $\rho: \mathbb{N} \rightarrow \mathbb{N}$. Our goal will be to bound the $X$-length of all of the components of $p$ to get a bound for the $X$-length of $t$. 
		
		\vs 
		
		First, we observe that $p,q$ do not have any connected synchronous components. Indeed, if there existed a pair of synchronous connected $H_i$ components $a,b$ of $p,q$, respectively, then we could write $p = p_1 a p_2$ and $q = q_1 b q_2$. But note that because $a,b$ are synchronous, we have $\phi(p_1) = \phi(q_1)$ so that $p_1, q_1$ are symmetric geodesics. In addition, $p_1, q_1$ have characteristic elements $(g,h)$ for some $h \in \cup_{i=1}^m H_i$ because $a,b$ are connected $H_i$ components, so $a_{-}, b_{-}$ are connected by $h \in H_i$. Therefore, $(p_1,q_1) \in \mathcal{P}$. However, since $a,b$ have relative length 1, we have $\ell(p_1) < \ell(p)$ and $\ell(q_1) < \ell(q)$. Therefore, we have a contradiction to the minimality of $(p,q)$. Hence, $p,q$ can have no connected synchronous components. By Lemma 2.47 (i), since we have just shown that we cannot have connected synchronous components, it follows that we cannot have connected components of $p,q$ at all. 
		
\begin{figure} [H]
	\centering
	\includegraphics[width=0.7\linewidth]{"../Desktop/Honours Project/IMG_1537"}
	\caption{Connected components of $p,q$ in the proof of Lemma 2.65}
	\label{fig:img1537}
\end{figure}
		
		\vs 
		
		Next, we observe that no component of $p$ can be connected to $r$. Indeed, suppose that we did have some $H_i$ component $c$ of $p$ connected to $r$. Decomposing $p$ as $s_1 c s_2$ we would have that $\pi(\phi(c s_2)) \in H_i$, because $(c s_2)_{-}^{-1}(c s_2)_{+} = h_1 h^{-1} \in H_i$ (refer to figure 83). Let $s_1'$ be the subpath of $q$ with the label as $s_1$. Let $t_0 = \pi(\phi(s_1))$ and $t_1 = \pi(\phi(c p_2))$. Then we have $g^t_0 = t_0^{-1} g t_0 = (t t_1^{-1})^{-1} g (t t_1^{-1})= t_1  (t^{-1} g t) t_1^{-1} = t_1 h t_1^{-1} \in H_i$ because $t_1, h \in H_i$. Therefore, $(s_1,s_1')$ are symmetric geodesics with characteristic elements $(g, g^t_0)$ and $g^t_0 \in H_i$. Hence, $(s_1, s_1') \in \mathcal{P}$. But $\ell(s_1) < \ell(p)$ and $\ell(s_1') < \ell(q)$, because $\ell(c) = 1$. Thus, we again contradict the minimality of $(p,q)$. We conclude therefore $r$ is not connected to any component of $p$. Similarly, $r$ is not connected to any component of $q$. Note that by Lemma 1.19, since $q,r$ is a geodesic and $r$ is a path of length 1, $qr$ is a (1,2)-quasi-geodesic and also $p$ is a (1,2)-quasi-geodesic because it is a geodesic. Note also that $p, qr$ are $k$-similar because $d_X(p_{-}, (qr)_{-}) = \vert g \vert_X \leq k$ and $d_X(p_{+}, (qr)_{+}) = 0$. Lastly, we also note that $p,qr$ have no backtracking, as $p$ does not have any backtracking because it is a geodesic and $qr$ does not have backtracking because $q$ is a geodesic and no components of $q$ are connected to $r$. Therefore, $p, qr$ is a pair of $k$-similar (1,2)-quasi-geodesics without backtracking, so the BCP propert applies. By the BCP property, since every component of $p$ is not connected to any component of $qr$, we have for any component $s$ of $p$ that $d_(s_{-}, s_{+}) \leq \varepsilon(1,2,k)$. By replacing every component of $p$ by a word over $X$, we therefore obtain that $\vert t \vert_X = d_X(p_{-}, p_{+}) \leq d_{X \cup \mathcal{H}}(p_{-}, p_{+}) \varepsilon(1,2,k) = \vert t \vert_{X \cup \mathcal{H}} \varepsilon(1,2,k) \leq \rho(k) \varepsilon(1,2,k)$. Therefore, we set $\sigma(k) = \rho(k) \varepsilon(1,2,k)$ for all $k \in \mathbb{N}$.
		
\begin{figure} [H]
	\centering
	\includegraphics[width=0.7\linewidth]{"../Desktop/Honours Project/IMG_1538"}
	\caption{The case when a component of $p$ is connected to $r$ in Lemma 2.65}
	\label{fig:img1538}
\end{figure}
		
	\end{proof}

	The next lemma gives us an analagous result for specific parabolic subgroups. 
	
	\vs 
	
	\textbf{Lemma 2.66: } Let $G$ be a finitely generated group hyperbolic relative to recursively presented subgroups $\{H_1,...,H_m\}$ where each $H_i$ has solvable conjugacy problem. Then there exists a recursive function $\theta: \mathbb{N} \rightarrow \mathbb{N}$ such that given any $g \in G$ with $\vert g \vert_X \leq k$ and $g$ conjugate to an element of $H_i$ for some $i$, then there exists $t \in G$ such that $\vert t \vert_X \leq \theta(k)$ and $g^t \in H_i$. 
	
	\begin{proof}
		
		We begin our setup in much the same way as the above Lemma. Let $\mathcal{Q}$ denote the set of all symmetric geodesics which have characteristic elements $(g,h)$, where $h \in H_i$ is conjugate to $g$ (here $i$ is fixed). Let $(p,q) \in \mathcal{Q}$ be a pair of symmetric geodesics with minimal length. Let $t = \pi(\phi(p)) = \pi(\phi(q))$. By Lemma 2.49, we have $\vert t \vert_{X \cup \mathcal{H}} \leq \rho(k)$ for some recursive function $\rho : \mathbb{N} \rightarrow \mathbb{N}$. 
		
		\vs 
		
		Following the same arguments in the proof of the above lemma, we note that $p, qr$ is a pair of $k$-similar (1,2)-quasi-geodesics without backtracking and that no component of $p$ is connected to $r$. However, unlike in the proof of the above Lemma, since we consider a fixed subgroup $H_i$, $p,q$ we may have connected components. If $p,q$ have no connected components, then we conclude the proof by simply repeating the arguments in the proof of the above lemma. Therefore, we assume that $p,q$ have connected components. We decompose $p = p_1 a_1 p_2 a_2 ... p_n a_n p_{n+1}$ and $q = q_1 b_1 q_2 b_2 ... q_n b_n q_{n+1}$, where for each $j$, $a_j, b_j$ are connected components while none of the $p_j, q_j$ have any connected components (see figure 84). We note that each $v_j$ is conjugate to $w_j$ in $H_{i_j}$ for some subgroup $i_j$. Also, by Theorem 2.35 (c) we have that $d_X((v_j)_{\pm}, (w_j)_{\pm}) \leq \varepsilon(1,2,k)$. Since the conjugacy problem is solvable in each $H_i$, there exists a recursive function $\tau: \mathbb{N} \rightarrow \mathbb{N}$ such that for any pair of conjugate elements $u,v \in H_i$ for any $i$, $\exists s \in H_i$ such that $v^s = w$ and $\vert s \vert_X \leq \tau(\max \{\vert v \vert_X, \vert w \vert_X\})$. Therefore, for each $j$, $\exists s_j \in H_{i_j}$ such that $v_j^{s_j} = w_j$ and $\vert s_j \vert \leq \tau(\varepsilon(1,2,k))$. 
		
\begin{figure} [H]
	\centering
	\includegraphics[width=0.7\linewidth]{"../Desktop/Honours Project/IMG_1539"}
	\caption{Connected components in the proof of Lemma 2.66. The $v_i, w_i$ are component connectors}
	\label{fig:img1539}
\end{figure}
		
		\vs 
		
		Now we let $z = \pi(\phi(p_1)) s_1 ... \pi(\phi(p_n)) s_n \pi(\phi(p_{n+1}))$. We note that $g^z = h$ because $g^{\pi(\phi(p_1))} = v_1$, $v_j^{s_j} = w_j$ for all $j$ and $w_n^{\pi(\phi(p_{n+1}))} = h$. 
		
		\vs 
		
		Finding an upper bound for $\vert z \vert_X$: we know that no components of $p_j, q_j$ are connected for $j =1,...,n$ and also no components of $p_{n+1}r, q_{n+1}$ are connected. In addition, by $p_j, q_j$ are $\ell = \max \{\varepsilon(1,2,k), k\}$-similar and so are $p_{n+1}r, q_{n+1}$. Also, none of the $p_j, q_j$ have backtracking since they are geodesics and also $p_{n+1}r$ does not have backtracking since $r$ does not connect to any component of $p_{n+1}$. Therefore, by Theorem 2.35 (b), for any component $a$ of $p_{n+1}$ we have $\vert a \vert_X \leq \varepsilon(1,2,\ell)$. Thus, we have $\vert t \vert_X \leq \rho(k) \max \{\varepsilon(1,2,\ell), \tau(\varepsilon(1,2,k))\}$, as the $\varepsilon(1,2,\ell)$ bounds components of $p_j$ while $\tau(\varepsilon(1,2,k))$ bounds the $X$-length of the components $a_j$, as $\varepsilon(1,2,\ell)$ is an upperbound on the $X$-length of the components and we may assume $\tau(n) \geq n$ for all $n$. Thus, we set $\theta(k) = \rho(k) \max \{\varepsilon(1,2,\ell), \tau(\varepsilon(1,2,k))$
		
	\end{proof}

	We are now in a position to prove our main theorem. 
	
	\begin{proof}
		
		We begin with (a). Since the word problem is solvable in each $H_i$, the membership problem is solvable for each $H_i$ by Theorem 2.60. Given $g \in G$, by Lemma 2.65, $g$ is parabolic iff $g^t \in H_j$ for some $j \in \{1,...,m\}$ and $\vert t \vert_X \leq \sigma(\vert g \vert_X)$. Since $X$ is finite, note that there are only finitely many $t \in G$ satisfying $\vert t \vert_X \leq \sigma(\vert g \vert_X)$ for any $g \in G$. So for each $j$, we need only check for finitely many $t$ such that $g^t \in H_j$ (and we can make such a check because the membership problem is solvable in each $H_j$). Hence, the general parabolicity problem is solvable and the algorithm that we have outlined above will find the conjugating element $t$. 
		
		\vs 
		
		For (b), we apply similar reasoning. Since the conjugacy problem is solvable in each $H_i, $, the word problem is also solvable in each $H_j$ (because a word is equal to 1 iff it is conjugate to 1), hence the membership is solvable in $H_i$. By Lemma 2.66, $g^t \in H_i$ iff $\vert t \vert_X \leq \theta(\vert g \vert_X)$. Therefore, we need only check (using the solution to the membership problem) for finitely many $t \in G$ whether $g^t \in H_i$. Hence, the special parabolicity problem is solvable and the above algorithm will return a conjugating element.
		
	\end{proof}
	
	\vs 
	
	\underline{Problems for hyperbolic elements}: 
	
	\vs 
	
	\underline{Conjugacy Problem}
	
	\vs 
	
	We outline a solution of the conjugacy problem given by Osin in [16] for a particularly simple case: the conjugacy problem for hyperbolic elements, which asks whether two hyperbolic elements are conjugate in $G$. The case for parabolic elements requires more work and we shall not present the solution in this paper. We will, however, discuss the conjugacy problem for parabolic elements in a more general framework, when we discuss the \textit{generalized conjugacy problem} in the next section. The solution to the conjugacy problem for parabolic elements is treated in Bumagin's paper [5]. 
	
	\vs 
	
	The solution to the conjugacy problem for hyperbolic elements will consist of proving the following: 
	
	\vs 
	
	\textbf{Theorem 2.67: } Let $G$ be a finitely generated group hyperbolic relative to recursively presented subgroups $\{H_1,...,H_m\}$ where each $H_i$ has solvable word problem. Then the conjugacy problem for hyperbolic elements is solvable. 
	
	\begin{proof}
			By Theorem 2.59 (the solution to the word problem), to solve the conjugacy problem for hyperbolic elements in $G$, it suffices to show that there exists a recursive function $\alpha: \mathbb{N} \rightarrow \mathbb{N}$ such that whenever $f,g$ are conjugate hyperbolic elements such that $\max\{ \vert f \vert_X, \vert g \vert_X \} \leq k$ we can find an element $t \in G$ such that $\vert t \vert_X \leq \alpha(k)$ and $f^t = g$. Indeed, if this condition holds then given hyperbolic $f,g \in G$ with $\max\{ \vert f \vert_X, \vert g \vert_X \} \leq k$, we use our solution to the word problem to check whether $f^t = g$ for each $t \in G$ with $\vert t \vert_X \leq \alpha(k)$. If we find such a $t$, then we conclude that $f,g$ are conjugate and if no such $t$ conjugates $f$ to $g$, then by the above condition, we conclude that $f,g$ are not conjugate. Note that because $X$ is finite, there are only finitely many elements $t$ such that $\vert t \vert_X \leq \alpha(k)$, (the number of such elements is bounded above by a recursive function of $k$ due to the presence of the recursive function $\alpha$, so we can in fact enumerate all such elements $t$ by an algorithm) therefore the number of equalities $f^t = g$ that we need to check is bounded above by a recursive function, so each equality can be checked by an algorithm. We now proceed to prove the above stated condition. 
		
		\vs 
		
		Suppose that $f,g$ are conjugate hyperbolic elements. Then by Remark 2.44, $f,g$ are characteristic elements for a pair of minimal length symmetric geodesics $(p,q)$ (see figure 85 below). We note that $p,q$ do not have any connected synchronous components as if there were connected synchronous components of $p,q$ then $f,g$ would be conjugate to the connectors of those components, hence $f,g$ would be parabolic elements, a contradiction. By Lemma 2.47 (i), this means that $f,g$ do not have any connected components at all, hence all components of $f$ and $g$ are isolated. By Theorem 2.35, as $p,q$ are $k$-similar geodesics, there exists some constant $C(1,0,k)$ such that $\vert s \vert_X \leq C(1,0,k)$ for every component $s$ of $f$ and $g$. In addition, by Corollary 2.49, we can bound the relative length of the conjugate element $t = \pi(\phi(p)) = \pi(\phi(q))$ as: $\vert t \vert_{X \cup \mathcal{H}} \leq \rho(k)$ for some constant $\rho(k)$. Therefore, we have the upper bound $C(1,0,k)$ on the $X$-length of each edge of $p,q$ and we have the upper bound $\rho(k)$ on the number of edges of $p,q$, so we conclude that $\vert t \vert_X \leq C(1,0,k)\rho(k)$. Therefore, we set $\alpha(k) = C(1,0,k)\rho(k)$ to be our recursive function. 
		
		
\begin{figure} [H]
	\centering
	\includegraphics[width=0.7\linewidth]{"../Desktop/Honours Project/IMG_1482"}
	\caption{Characteristic elements $f,g$ for the pair of symmetric geodesics $p,q$ in the proof of Theorem 2.67}
	\label{fig:img1482}
\end{figure}
		
	\end{proof}
	
	\underline{Generalized Conjugacy Problem}

	\vs 
	
	In this section, we outline some work done by the author on constructing an algorithm to solve the \textit{generalized conjugacy problem} (also known as the \textit{conjugacy problem for finite lists} and the \textit{simultaneous conjugacy problem}) in relatively hyperbolic groups, which remains an open problem. 
	
	\vs 
	
	Let $G$ be a finitely generated group hyperbolic relative to finitely generated subgroups $\{H_1,...,H_m\}$ where for each $i$ the conjugacy problem is solvable in $H_i$. The generalized conjugacy problem asks, given two finite lists $\{g_1,...,g_n\}$ and $\{h_1,...,h_n\}$ of elements of $G$, whether there exists $t \in G$ such that $g_i^t = h_i^t$ for all $i = 1,...,n$. It was shown by Bridson and Howie in [4] that there is an algorithm to solve the generalized conjugacy problem for lists of infinite order elements in hyperbolic groups. The case of lists containing torsion (finite order) elements proves to be more challenging and is not solved in [4]. The author has attempted to generalize these algorithms to relatively hyperbolic groups. 
	
	\vs 
	
	We begin by outlining the algorithm of Bridson and Howie to solve the generalized conjugacy problem for finite lists of torsion-free elements and we discuss the generalization to relatively hyperbolic groups. We first introduce some terminology. 
	
	\vs 
	
	\textbf{Definition 2.68: } Let $a \in G$ have infinite order. We define $a^{\infty} = \overline{\{a^n \cdot 1: n \geq 0\}} \cap \partial \Gamma(G; X)$ (where $1$ is the vertex of the identity in the Cayley graph $\Gamma(G; X)$), where the closure is taken in the natural topology on $\overline{\Gamma(G; X)} = \Gamma(G; X) \cup \partial \Gamma(G; X)$ introduced in section 1. Similarly, define $a^{-\infty} = \overline{\{a^n \cdot 1: n \leq 0\}} \cap \partial \Gamma(G; X)$. The \textit{limit set} of $a$ on $\partial \Gamma(G; X)$ is the set $\{a^{-\infty}, a^{\infty}\}$. 
	
	\vs 
	
	As with the solution to the ordinary conjugacy problem (i.e. conjugacy for single elements), the solution to the generalized conjugacy problem consists of finding an upper bound for the length of conjugating elements between two conjugate lists in terms of the length of the elements in the list. The following theorem is key to Bridson and Howie's solution to the generalized conjugacy problem in torsion-free hyperbolic groups.
	
	\vs 
	
	\textbf{Theorem 2.69: } Let $G$ be a torsion-free hyperbolic group. Let $a_1, a_2, b_1, b_2, x \in G$ be non-trivial elements such that $a_i^x = b_i$ for all $i = 1,2$ and suppose that $a_2$ does not fix the limit set $\{a_1^{\pm \infty}\}$ of $a_1$ on $\partial G$. Then $\vert x \vert \leq \alpha \mu + \beta$, where $\alpha, \beta$ are constants independent of $x$ (but may depend on $a_i, b_i$) and $\mu = \max \{\vert a_1 \vert, \vert a_2 \vert, \vert b_1 \vert, \vert b_2 \vert \}$. 
	
	\vs 
	
	To prove this theorem, we will first have to introduce some additional terminology. 
	
	\vs 
	
	We fix a linear ordering on the finite generating set $X$ of $G$, which induces a lexicographic order on words in the free monoid $X^*$. This then induces a lexicographic order on finite length geodesics in $\Gamma(G; X)$, via the lexicographic order on their labels. As introduced by Delzant in [7], a \textit{special geodesic} is an oriented geodesic that is minimal among all geodesics with the same start and end points (minimal with respect to the above lexicographic order). Similarly, an infinite length geodesic is called \textit{special} if all of its finite subsegments are special geodesics. We list some important properties of special geodesics that we will make use of in proving the above theorem. Their proof depends purely on working with the hyperbolic geometry of $\Gamma(G; X)$ and we shall omit the proof to preserve conciseness of this paper. Details can be found in Bridson and Haefliger's book [3] and in Delzant's paper [7]. 
	
	\vs 
	
	\textbf{Proposition 2.70: } Let $a \in G \setminus \{1\}$. Then: 
	
	\begin{enumerate}[label = (\roman*)]
		\item There exists an infinite special geodesic $\gamma$ joining $a^{-\infty}$ to $a^{\infty}$ such that $\exists B > 0$ such that $d(a^n, \gamma) \leq B$ for all $n \in \mathbb{Z}$. 
		\item The number of special geodesics joining $a^{-\infty}$ to $a^{\infty}$ is bounded above by a constant $R$ which only depends on the hyperbolicity constant $\delta$ of $\Gamma(G; X)$ and on $k := \vert X \vert$. 
		\item $a$ permutes the above finite set of special geodesics and $a^{R!}$ acts on each special geodesic as a translation in the direction of $a^{\infty}$ (i.e. $a^{R!}$ moves each point on any special geodesic $\gamma$ joining $a^{-\infty}$ to $a^{\infty}$ a distance $\ell_{\gamma}$ along $\gamma$ towards $a^{- \infty}$). 
		\item There is a constant $K$ depending only on $\delta$ and $k$ such that for any special geodesic $\gamma$ from $a^{-\infty}$ to $a^{\infty}$, $C(a) \subseteq N_{(1 + \vert a \vert)K}(\gamma)$ (where $C(a)$ denotes the centralizer of $a$ in $G$). 
	\end{enumerate}

	We will also require the following Lemma, which says that geodesic lines which diverge slow enough must be within a bounded Hausdorff distance of each other.
	
	\vs 
	
	\textbf{Lemma 2.71: } Let $\ell$ be a positive integer. Suppose that $\sigma_1, \sigma_2$ are bi-infinite geodesics in $\Gamma(G; X)$ ($G$ is a hyperbolic group, $X$ is a finite generating set), and $a_1, a_2 \in G$ are such that $a_i$ acts by translation of length $\ell$ along $\sigma_i$ ($i=1,2$). If $(d(p_1, p_2) + d(a_1^n(p_1), a_2^n(p_2)) + 4 \delta)/\ell + (2K+1)^{6 (\delta + 1)} < \vert n \vert$ for some $p_1 \in \sigma_1, p_2 \in \sigma_2$ and some $n \in \mathbb{Z}$, then $d_{\text{Haus}}(\sigma_1, \sigma_2) \leq 2 \delta$.  
	
	\begin{proof}
		
		For illustrations of the geometry described below, refer to figures 86 and 87. 
		
		\vs
		
		First, note that we may assume $n > 0$ (replacing $a_i$ with $a_i^{-1}$ if necessary). Choose $p_0$ to be the vertex on $\sigma_1$ between $p_1$ and $a_1^n (p_1)$ such that $d(p_1, p_0) = d(p_1, p_2) + 2 \delta + 1$ (note that such a point $p_0$ exists because $d(p_1, a_1^n (p_1)) = \ell n > \ell (2K + 1)^{6 (\delta + 1)} + d(p_1, p_2) + d(a_1^n(p_1), a_2^n(p_2)) + 4 \delta > d(p_1, p_2) + 2 \delta$). 
		
		\vs 
		
		We note that for any $j = 0,..., (2K+1)^{6 \delta}$, we have $a_1^j (p_0) \in [p_1, a_1^n(p_1)]_{\sigma_1}$ (the segment on $\sigma_1$ between $p_1$ and $a_1^n(p_1)$). Indeed, $d(p_1, a_1^j (p_0)) = d(p_1, p_0) + d(p_0, a_1^j(p_0)) = d(p_1, p_2) + 2\delta + j \ell \leq d(p_1, p_2) + 2\delta + (2K + 1)^{6 \delta}\ell \leq d(p_1, p_2) + d(a_1^n(p_1), a_2^n(p_2)) + 4\delta + (2K + 1)^{6 \delta}\ell < \ell n = d(p_1, a_1^n (p_1))$. Therefore $a_1^j(p_0) \in [p_1, a_1^n(p_1)]_{\sigma_1}$. 
		
		\vs 
		
		In addition, for any $j = 0,..., (2K+1)^{6 \delta}$, $d(p_1, a_1^j(p_0)) > d(p_1, p_2) + 2 \delta$ and $d(a_1^j (p_0), a_1^n(p_1)) > d(a_1^n(p_1), a_2^n(p_2)) + 2 \delta$. Indeed, $d(p_1, a_1^j (p_0)) = d(p_1, p_0) + j \ell \geq d(p_1, p_0) > d(p_1, p_2) + 2 \delta$ and $d(a_1^j (p_0), a_1^n (p_1)) = d(p_1, a_1^n(p_1)) - d(p_1, a_1^j (p_0)) = n \ell - (d(p_1, p_0) + j \ell) = n \ell - (d(p_1, p_2) + 2 \delta + 1 + j \ell) \geq n\ell - (2K+1)^{6 \delta} \ell - d(p_1, p_2) - 2 \delta - 1 > (2K + 1)^{6 (\delta+1)} + d(p_1, p_2) + d(a_1^n(p_1), a_2^n (p_2)) + 4 \delta - (2K+1)^{6 \delta} \ell - d(p_1, p_2) - 2 \delta  - 1 > d(a_1^n (p_1), a_2^n (p_2)) + 2 \delta$, as required.
		
		\vs 
		
		Given any point $x$ on $[p_1, a_1^n (p_1)]_{\sigma_1}$ such that $d(x, p_1) > d(p_1, p_2) + 2 \delta$ and $d(x, a_1^n (p_1)) > d(a_1^n (p_1), a_2^n (p_2)) + 2 \delta$, using the $2\delta$-slimness of the geodesic quadrilateral with vertices $p_1, a_1^n (p_1), a_2^n (p_2), p_2$, there exists a point $q(x)$ on $\sigma_2$ such that $d(x, q(x)) \leq 2 \delta$ (indeed, note that no such point can exist on $[p_1, p_2]$ or $[a_1^n (p_1), a_2^n (p_2)]$ because for any point $y$ on $[p_1, p_2]$, we have $d(x,y) \geq d(x, p_1) - d(p_1, y) \geq d(x, p_1) - d(p_1, p_2) > d(p_1, p_2) + 2\delta - d(p_1, p_2) = 2\delta$ and similarly for any $y$ on $[a_1^n (p_1), a_2^n (p_2)]$, we have $d(x, y) \geq d(x, a_1^n (p_1)) - d(a_1^n (p_1), y) \geq d(x, a_1^n (p_1)) - d(a_1^n (p_1), a_2^n (p_2)) > d(a_1^n (p_1), a_2 ^n (p_2)) + 2\delta - d(a_1 ^n (p_1), a_2^n (p_2)) = 2 \delta$). Denote $q_0 = q(p_0)$. 
		
		\vs 
		
		Next, for $x$ between $p_0$ and $a_1^n (p_1)$ such that $d(x, p_0) > 6 \delta$, we define $q'(x)$ to be the point between $q_0$ and $a_2^n (p_2)$ synchronous to $x$ , i.e. such that $d(p_0, x) = d(q_0, q'(x))$. We note that $q(x)$ cannot lie between $p_2$ and $q_0$ because then $q(x)$ would have to be $2\delta$ close to some point on $[p_1,p_0] \cup [p_1, p_2] \cup [p_0, q_0]$, but for any $y \in [p_1,p_0] \cup [p_1, p_2] \cup [p_0, q_0]$, we have $d(y, x) > 4 \delta$ (indeed, if $y \in [p_1, p_0]$, we have $d(y,x) \geq d(p_0, x) \geq 6 \delta > 4 \delta$, if $y \in [p_1, p_2]$, then by the triangle inequality, $d(y,x) \geq d(p_1, x) - d(p_1, y) \geq d(p_1, x) - d(p_1, p_2) = d(p_0, p_1) + d(p_0, x) - d(p_1, p_2) > d(p_1, p_2) + 2 \delta + 6 \delta - d(p_1, p_2) = 8 \delta$, and if $y \in [p_0, q_0]$, then by the triangle inequality, $d(y,x) \geq d(p_0, x) - d(y, p_0) \geq d(p_0, x) - d(p_0, q_0) > 6 \delta - 2 \delta = 4 \delta$). Therefore, if $q(x) \in [p_2, q_0]_{\sigma_1}$, then for some $y \in [p_1,p_0] \cup [p_1, p_2] \cup [p_0, q_0]$, we have $d(y, q(x)) \leq 2 \delta$ and $d(y, x) > 4 \delta$, so $d(x, q(x)) \geq d(x, y) - d(y, q(x)) > 4 \delta - 2 \delta = 2 \delta$, a contradiction. Therefore, $q(x)$ lies in $[q_0, a_2^n (p_2)]_{\sigma_2}$. As $q'(x)$ also lies in this interval, we have $d(q(x), q'(x)) = \vert d(q_0, q(x)) - d(q_0, q'(x)) \vert = \vert d(q_0, q(x)) - d(p_0, x) \vert \leq d(p_0, q_0) + d(x, q(x)) \leq 4 \delta$, where the second last line follows from the triangle inequality. Thus, we obtain $d(x, q'(x)) \leq d(x, q(x)) + d(q(x), q'(x)) \leq 2 \delta + 4\delta = 6 \delta$. 
		
		\vs 
		
		Next, we note that because $a_2$ translates a distance $\ell$ along $\sigma_2$, we have that for $x = a_1^j (p_0)$, $q'(x) = a_2^j (q_0)$. We also have that for $6\delta < j \leq (2K+1)^{6\delta}$, $d(p_0, a_1^j (p_0)) = j \ell > 6 \delta$, so by above, $= d(a_1^j (p_0), a_2^j (q_0)) = d(a_1^j (p_0), q'(a_1^j (p_0))) \leq 6 \delta$. Thus, we have a map $\{6 \delta + 1, ..., (2K+1)^{6\delta}\} \rightarrow B(6 \delta, 1)$ given by $j \mapsto (a_1^j (p_0))^{-1}(a_2^j(q_0))$. We note that $\vert B(6 \delta, 1) \vert < 1 + (2k)^{6 \delta} < (2K+1)^{6 \delta} - (6 \delta + 1)$, so the above map cannot be injective and therefore there exist $s < t \in \{6 \delta + 1, ..., (2K+1)^{6\delta}\}$ such that $(a_1^s (p_0))^{-1}(a_2^s(q_0)) = (a_1^t (p_0))^{-1}(a_2^t(q_0)) \implies a_1^{t-s} = a_2^{t-s}$. From this we show that $d_{\text{Haus}}(\sigma_1, \sigma_2) \leq 2 \delta$, as follows: 
		
		\vs 
		
		Let $y \in \sigma_1$.  Choose an $m \in \mathbb{Z}$ such that $a_1^{m(t-s)} (y) \in [a_1^s (p_0), a_1^t (p_0)]_{\sigma_1}$ (such an $m$ indeed exists because $\sigma_1 = \cup_{k \in \mathbb{Z}} [a_1^{(k-1)(t-s)}(y), a_1^{k(t-s)}(y)]$, so $\exists m \in \mathbb{Z}$ such that $a_1^s(p_0) \in [a_1^{(m-1)(t-s)}(y), a_1^{m(t-s)}(y)]$. This implies $a_1^{m(t-s)}(y) \in [a_1^s(p_0), a_1^t (p_0)]_{\sigma_1}$ as $a_1^{m(t-s)}(y)$ is to the right of $a_1^s(p_0)$ (where the order on $\sigma_1$ is induced from the order on $\mathbb{R}$) and $d(a_1^s (p_0), a_1^{m(t-s)}(y)) \leq \ell (t - s) = \ell([a_1^s(p_0), a_1^t (p_0)]_{\sigma_1})$. We then have: $d(y, \sigma_2) \leq d(y, a_2^{m(s-t)}(q(a_1^{m(t-s)}(y)))) = d(a_2^{m(t-s)}(y), q(a_1^{m(t-s)})(y)) = d(a_1^{m(t-s)}(y), q(a_1^{m(t-s)})(y)) \leq 2 \delta$. Hence, $\sigma_1 \subseteq N_{2 \delta} (\sigma_2)$ and in an analogous manner we deduce $\sigma_2 \subseteq N_{2 \delta}(\sigma_1)$, hence $d_{\text{Haus}}(\sigma_1, \sigma_2) \leq 2 \delta$. 
		
		
		
\begin{figure} [H]
	\centering
	\includegraphics[width=0.7\linewidth]{"../Desktop/Honours Project/IMG_1523"}
	\caption{The setup in the proof of Lemma 2.71}
	\label{fig:img1523}
\end{figure}

\begin{figure} [H]
	\centering
	\includegraphics[width=0.7\linewidth]{"../Desktop/Honours Project/IMG_1524"}
	\caption{The points $q_0$ and $q(x)$ in the proof of Lemma 2.71}
	\label{fig:img1524}
\end{figure}
		
	\end{proof}

	
	\begin{proof}
		
		We now begin to describe the proof of Theorem 2.69. Refer to figures 88 and 89 for visual aids to the proof below. 
		
		\vs 
		
		We begin by applying Proposition 2.70 (i) to obtain a special geodesic $\sigma_1$ joining $a^{-\infty}$ to $a^{\infty}$ and an integer $t$ with $0 < t \leq R!$ such that $a_1^t$ acts as translation of length $\ell$ along $\sigma_1$. By Proposition 2.70 (iv), we also have $C(a_1) \subseteq N_{(1 + \vert a_1 \vert)K}(\sigma_1)$. 
		
		\vs 
		
		We note also that $a_2 a_1^t a_2^{-1}$ acts by translation of length $\ell$ along $\sigma_2 : = a_2 (\sigma_1)$ (indeed, for any $p$ on $\sigma_2$, we have $p = a_2 (q)$ for some $q$ on $\sigma_1$, so that $d(p, a_2 a_1^t a_2^{-1}(p) )= d(a_2(q), a_2 a_1^t a_2^{-1}(a_2(q))) = d(a_2(q), a_2 a_1^t (q)) = d(q, a_1^t (q)) = \ell$). In addition, $a_2C(a_1)a_2^{-1} = C(a_2 a_1 a_2^{-1}) \subseteq N_{(1 + \vert a_2 a_1 a_2^{-1})K}(\sigma_2)$. 
		
		\vs 
		
		The idea of the proof is to choose points $p_1$ on $\sigma_1$, $p_2$ on $\sigma_2$ which are at a minimum distance to $x$ among all points on $\sigma_1$, $\sigma_2$. We bound above the distances $d(x, p_1)$ and $d(x, p_2)$ using the solution to the ordinary conjugacy problem to bound above the length of a conjugating element between specific elements. Then we project the identity 1 onto each special geodesic and choose $n,m \in \mathbb{Z}$ such that the projection of 1 is in the segment $[(a_1^t)^n p_1, (a_1^t)^{n+1} p_1]$ of $\sigma_1$ and such that the projection of 1 onto $\sigma_2$ is on the segment $[(a_2 a_1^t a_2^{-1})^m p_2, (a_2 a_1^t a_2^{-1})^{m+1} p_2]$. We then use Lemma 2.71 to estimate $\vert n \vert$, noting that because $\sigma_1, \sigma_2$ do not have the same endpoints on $\partial G$, they are not within a bounded Hausdorff distance, which yields the appropriate bound on $\vert n \vert$. We detail these steps below.
		
		\vs 
		
		First, since $a_1$ is conjugate to $b_1$, there exists an element $c_1$ such that $a_1^{c_1} = b_1$ and $\vert c \vert \leq N_1 (\vert a_1 \vert + \vert b_1 \vert) + M_2$, where the constants $N_1, M_1$ do not depend on $a_1$ or $b_1$ (this upper bound which is linear in $\vert a_1 \vert, \vert b_1 \vert$ does not follow from the solution to the conjugacy problem presented in this paper (our upper bound was exponential in the lengths of the conjugate words), but it is shown in Bumagin's paper [5] (Theorem 1.1) as well as Bridson and Howie's paper [3] (Proposition 2.3) that such a linear upper bound on the length of a conjugating element indeed exists). Similarly, we have $a_2^{-1} a_1 a_2$ is conjugate to $b_2^{-1} b_1 b_2$, because $b_2 = t^{-1} a_2 t$ and $b_1 = t^{-1} a_1 t$, thus $a_2^{-1} a_1 a_2 = (t^{-1} b_2^{-1} t) (t^{-1} b_1 t) (t^{-1} b_2 t) = t^{-1} b_2 b_1 b_2 t$. Therefore, there exists a conjugating element $c_2$ from $a_2^{-1} a_1 a_2$ to $b_2^{-1} b_1 b_2$ such that $\vert c_2 \vert \leq N_2 (\vert a_2^{-1} a_1 a_2 \vert + \vert b_2^{-1} b_1 b_2 \vert) + M_2$ for constants $N_2, M_2$ independent of $a_1,a_2,b_1,b_2$.  
				
		\vs 
		
		Next, we let $p_1, p_2$ be the closest points on $\sigma_1, \sigma_2$ to $x$. Because $x$ and $c_1$ both conjugate $a_1$ to $b_1$, we have $xc _1^{-1} \in C(a_1)$ (see Remark 2.72). But by Proposition 2.70, we have that $C(a_1) \subseteq N_{(1 + \vert a_1 \vert)K}(\sigma_1)$, so letting $q$ denote the closest point on $\sigma_1$ to $xc_1^{-1}$, we have $d(x, p_1) \leq d(x, q) \leq d(x, xc_1^{-1}) + d(xc_1^{-1}, q) \leq \vert c_1 \vert + (1 + \vert a_1 \vert)K \leq N_1 (\vert a_1 \vert + \vert b_1 \vert) + M_1 + (1 + \vert a_1 \vert)K$. Similarly, we obtain $d(x, p_2) \leq N_2 (\vert a_2^{-1} a_1 a_2 \vert + \vert b_2^{-1} b_1 b_2 \vert) + M_2 + (1 + \vert a_2 ^{-1} a_1 a_2 \vert)K$. Thus, by the triangle inequality, we have $d(p_1, p_2) \leq d(x, p_1) + d(x, p_2) \leq N_1 (\vert a_1 \vert + \vert b_1 \vert) + M_1 + (1 + \vert a_1 \vert)K + N_2 (\vert a_2^{-1} a_1 a_2 \vert + \vert b_2^{-1} b_1 b_2 \vert) + M_2 + (1 + \vert a_2 ^{-1} a_1 a_2 \vert)K \leq N_1 (\vert a_1 \vert + \vert b_1 \vert) + M_1 + (1 + \vert a_1 \vert)K + N_2 (\vert a_1 \vert + \vert b_1 \vert + 2 \vert a_2 \vert + 2 \vert b_2 \vert) + M_2 + (1 + \vert a_1 \vert + 2 \vert a_2 \vert)K \leq 2(N_1 + N_2) (\vert a_1 \vert + \vert a_2 \vert + \vert b_1 \vert + \vert b_2 \vert) + M_1 + M_2 + 2K(1 + \vert a_1 \vert + \vert a_2 \vert) \leq 2(N_1 + N_2 + K) (\vert a_1 \vert + \vert a_2 \vert + \vert b_1 \vert + \vert b_2 \vert) + M_1 + M_2$. 
		
		\vs 
		
		Now since $a_1^t$ acts by translation of length $\ell$ on $\sigma_1$, projecting 1 to a closest point $q$ to 1 on $\sigma_1$, we have that there exists $n \in \mathbb{Z}$ such that $q \in [a_1^{tn} (p_1), a_1^{t(n+1)} (p_1)]$. Then, noting that $1 \in C(a_1)$, we have $d(1, a_1^{tn}(p_1)) \leq d(1, q) + d(q, a_1^{tn}(p_1)) \leq d(1, q) + d(a_1^{tn} (p_1), a_1^{t(n+1)} (p_1)) \leq (1 + \vert a_1 \vert)K + d(p_1, a_1^t (p_1)) = (1 + \vert a_1 \vert)K + \ell \leq (1 + \vert a_1 \vert)K + t \vert a_1 \vert$ (recall that $\ell \leq t \vert a_1 \vert$, by Lemma 2.55) $ \leq (1 + \vert a_1 \vert)K + R \vert a_1 \vert \leq (R + K) (\vert a_1 \vert + 1)$.
		
		\vs 
		
		Similarly, by projecting 1 onto $\sigma_2$, we have that there exists $m \in \mathbb{Z}$ such that $d(1, a_2^{-1} a_1^{tm} a_2 (p_2)) \leq (R + K) (\vert a_2^{-1} a_1 a_2 \vert + 1) \leq (R+K) (\vert a_1 \vert + 2 \vert a_2 \vert + 1)$.
		
		\vs 
		
		By the triangle inequality, we then have: $d(a_1^{tn}(p_1), a_2^{-1}a_1 a_2 (p_2)) \leq d(1, a_1^{tn}(p_1)) + d(1, a_2^{-1}a_1 a_2 (p_2)) \leq (R + K) (\vert a_1 \vert + 1) + (R+K) (\vert a_1 \vert + 2 \vert a_2 \vert + 1) \leq 2(R+K)(\vert a_1 \vert + \vert a_2 \vert + 1) \leq 2(R+K)(\vert a_1 \vert + \vert a_2 \vert + \vert b_1 \vert + \vert b_2 \vert)$. 
		
		\vs
		
		We now seek to remove the presence of the integer $m$ from our inequalities, and then conclude by bounding $\vert n \vert$. 
		
		\vs 
		
		Because $a_1^t$ and $a_2^{-1} a_1^t a_2$ act by translations of length $\ell$ on their respective geodesics, we have:
		
		\begin{align*}
		\vert n - m \vert \ell &= \vert d(p_1, a_1^{tn}(p_1)) - d(p_2, a_2^{-1} a_1^{tm} a_2 (p_2)) \vert \\
		&\leq d(p_1, p_2) + d(a_1^{tn}(p_1), a_2^{-1} a_1^{tm} a_2 (p_2)) \hspace{0.25cm} \text{by the triangle inequality} \\
		&\leq 2(N_1 + N_2 + K) (\vert a_1 \vert + \vert a_2 \vert + \vert b_1 \vert + \vert b_2 \vert) + M_1 + M_2 + 2(R+K)(\vert a_1 \vert + \vert a_2 \vert + 1) \\
		&\leq 2(N_1 + N_2 + 2K + R) (\vert a_1 \vert + \vert a_2 \vert + \vert b_1 \vert + \vert b_2 \vert) + M_1 + M_2 \\
		&\leq 8(N_1 + N_2 + 2K + R)\mu + M_1 + M_2. 
		\end{align*}
		
		\vs 
		
		We then have by the triangle inequality:  $d(a_1^{tn}(p_1), a_2^{-1} a_1^{tn} a_2 (p_2)) \leq d(a_1^{tn}(p_1), a_2^{-1} a_1^{tm} a_2 (p_2)) + \vert n - m \vert \ell \leq 2(R + K) (\vert a_1 \vert + \vert a_2 \vert + 1) + 8(N_1 + N_2 + 2K + R)\mu + M_1 + M_2 \leq 8 (N_1 + N_2 + 6K + 5R) \mu + M_1 + M_2$. 
		
		\vs 
		
		Now since $\sigma_1, \sigma_2$ have different endpoints on $\partial G$, they are not within a bounded Hausdorff distance of each other and so in particular they are not contained in the $2 \delta$ neighbourhood of each other. Hence, by Lemma 2.71, we obtain a bound on $\vert n \vert$: $\vert n \vert \leq (2K + 1)^{6 \delta} + 1/\ell (d(p_1, p_2) + d(a_1^{tn}(p_1), a_2^{-1} a_1^{tn}a_2(p_2)) + 4 \delta) \leq (2K + 1)^{6 \delta} + 1/\ell (2(N_1 + N_2 + K) (\vert a_1 \vert + \vert a_2 \vert + \vert b_1 \vert + \vert b_2 \vert) + M_1 + M_2 + 8 (N_1 + N_2 + 6K + 5R) \mu + M_1 + M_2 + 4 \delta) \leq (2K + 1)^{6 \delta} + 1/\ell (8(2N_1 + 2N_2 + 6K + 5R)\mu + 2(M_1 + M_2) + 4 \delta)$. 
		
		\vs 
		
		Finally, we have: 
		
		\begin{align*}
		\vert x \vert &= d(1,x) \\
		&\leq d(1, a_1^{tn}(p_1)) + d(a_1^{tn}(p_1), p_1) + d(p_1, x) \\
		&= d(1, a_1^{tn}(p_1)) + \ell \vert n \vert + d(p_1, x) \\
		&\leq (R + K)(\vert a_1 \vert + 1) + \ell (2K + 1)^{6 \delta} + 8(2N_1 + 2N_2 + 6K + 5R)\mu \\
		&+ 2(M_1 + M_2) + 4 \delta + N_1 (\vert a_1 \vert + \vert b_1 \vert) + M_1 + (1 + \vert a_1 \vert)K \\
		&\leq (2(R + K) + 2K + 8(2N_1 + 2N_2 + 6K + 5R))\mu + 3M_1 + 2M_2 + 4\delta + \ell(2K + 1)^{6 \delta}. 
		\end{align*}
		
		It remains to estimate the translation length $\ell$. We have $\ell = \tau(a_1^t) = t \tau(a_1) \leq R \vert a_1 \vert$ (cf. Lemma 2.55) $\leq R \mu$. Thus, we obtain: $\vert x \vert \leq (2(R + K) + 2K + 8(2N_1 + 2N_2 + 6K + 5R) + R(2K+1)^{6 \delta})\mu + 3M_1 + 2M_2 + 4 \delta$. Thus, we set $\alpha = 2(R + K) + 2K + 8(2N_1 + 2N_2 + 6K + 5R) + R(2K+1)^{6 \delta}$ and $\beta = 3M_1 + 2M_2 + 4 \delta$. 
		
\begin{figure} [H]
	\centering
	\includegraphics[width=0.7\linewidth]{"../Desktop/Honours Project/IMG_1521"}
	\caption{Using the triangle inequality to bound the length of $x$ in the proof of Theorem 2.69}
	\label{fig:img1521}
\end{figure}

\begin{figure} [H]
	\centering
	\includegraphics[width=0.7\linewidth]{"../Desktop/Honours Project/IMG_1522"}
	\caption{Projecting 1 onto the geodesic $\sigma_1$ in the proof of Theorem 2.69}
	\label{fig:img1522}
\end{figure}

		
		
	\end{proof}
	
	\textbf{Remark 2.72: } Let $G$ be any group and let $g_1,...,g_n$, $h_1,...,h_n$ be two conjugate lists, that is $\exists t \in G$ such that $g_i^t = h_i$ for all $i$. We note that the conjugating elements for the two lists are precisely the elements in the set $(C(g_1) \cap ... \cap C(g_n))t$. Indeed, if $s = rt$ for some $r \in C(g_1) \cap ... \cap C(g_n)$, then for each $i$, we have $g_i^s = s^{-1} g_i s = (rt)^{-1} g_i (rt) = t^{-1} r^{-1} g_i r t = t^{-1} g_i t = h_i$, and conversely, if $s \in G$ is such that $g_i^s = h_i$ for all $i$, then $g_i^s = g_i^t$, so $s^{-1} g_i s = t^{-1} g_i t \implies (st^{-1})^{-1} g_i (st^{-1}) = g_i \implies st^{-1} \in C(g_i) \implies s \in C(g_i)t$. Hence $s \in (C(g_1) \cap ... \cap C(g_n))t$. 
	
	\vs 
	
	We now give Bridson and Howie's algorithm, and describe why it works afterwards: 
	
	\vs 
	
	The algorithm begins by taking as input two finite lists $A = \{a_1,...,a_m\}$ and $B = \{b_1,...,b_m\}$ of elements in a torsion-free hyperbolic group $G$ with finite generating set $X$. Note that in a hyperbolic group, the conjugacy (and conjugacy search) problem (and hence also the word problem) is solvable because recall from Example 2.5 that a hyperbolic group is a group which is hyperbolic relative to the trivial subgroup, which clearly has solvable word problem, so by Theorem 2.67, the conjugacy problem for hyperbolic elements (which are precisely the non-trivial elements in this case) is solvable, hence the conjugacy problem is solvable for all elements, because by Theorem 2.59, the word problem is solvable, so we can determine whether any given element in the group is trivial (i.e. conjugate to the identity).  
	
	\vs 
	
	\underline{Step 1: } First, we check if $a_1 = b_1 = 1$ using the solution to the word problem. If this is true and $m = 1$, then output YES and stop. If this is true and $m > 1$, delete $a_1, b_1$ from their respective lists and repeat this step with the new shorter lists as input. If one of $a_1, b_1 = 1$ but the other is not 1, then output NO and stop. If neither is 1, then proceed to Step 2. 
	
	\vs 
	
	\underline{Step 2: } Apply the solution to the conjugacy problem to determine if there exists $y \in G$ such that $a_1^y = b_1$ and produce such an element if it exists using the solution to the conjugacy search problem. If such a $y$ exists, proceed to Step 3. If such a $y$ does not exist, output NO and stop. 
	
	\vs 
	
	\underline{Step 3: } Using the solution to the word problem, check for which $i$ we have $a_1 a_i = a_i a_1$. If $a_1 a_i = a_i a_1$ for all $i = 2,...,m$, then proceed to Step 4 and if $a_1 a_i \neq a_i a_1$ for some $i$, then proceed to Step 5. 
	
	\vs
	
	\underline{Step 4: } Using the $y$ from Step 2, for each $i = 2,...,m$ check whether $y^{-1} a_i y  = b_i$. If this equality holds for all $i = 2,...,m$, then output YES with conjugating element $y$. If there exists an $i$ such that $y^{-1} a_i y \neq b_i$, then ouput NO and stop. 
	
	\vs
	
	\underline{Step 5: } Let $i$ from Step 3 be such that $a_1 a_i \neq a_i a_1$. Let $r = \vert y \vert + \alpha \mu + \beta$, where $\alpha \mu + \beta$ is the upper bound from Theorem 2.69 on the length of elements conjugating $\{a_1, a_i\}$ to $\{b_1, b_i\}$. List all of the elements in $C(a_1) \cap B(r,1)$. For each $z \in C(a_1) \cap B(r,1)$, check whether $(zy)^{-1} a_i (zy) = b_i$. If this equality fails for all $z \in C(a_1) \cap B(r,1)$, then output NO and stop. Otherwise, if the equality holds for some $z \in C(a_1) \cap B(r,1)$, then proceed to Step 6. 
	
	\vs 
	
	\underline{Step 6: } Let $zy$ be from Step 5. If $(zy)^{-1} a_k (zy) = b_k$ for all $k = 2,...,m$ then output YES along with the conjugating element $zy$ and stop. If $(zy)^{-1} a_k (zy) \neq b_k$ for some $k$, then output NO and stop. 
	
	\vs 
	
	We now describe why this algorithm indeed solves the generalized conjugacy problem for finite lists in torsion free hyperbolic groups.
	
	\vs 
	
	The first non-trivial claim to verify is the negative outcome of Step 4: if $a_1 a_i = a_i a_1$ for all $i = 2,...,m$ and $y$ is such that $a_1^y = b_1$, then if $a_i^y \neq b_i$ for some $i = 2,...,m$, then the lists $\{a_1,...,a_m\}, \{b_1,...,b_m\}$ are not conjugate. Indeed, suppose the lists were conjugate, so that $\exists t \in G$ such that $a_i^t = b_i$ for all $i = 1,...,m$. By Remark 2.72, conjugating elements between the two lists are of the form $rt$, where $r \in C(a_1) \cap ... \cap C(a_m)$. Now torsion-free hyperbolic groups are commutative transitive (CT) (see the section on algebraic properties of relatively hyperbolic groups for the definition of a CT group), as seen in [19]. In a CT group, if $a,b \neq 1$ and $[a,b] = 1$, then $C(a) = C(b)$ (indeed, note that commutativity is an equivalence relation for non-trivial elements in a CT group and the equivalence class of $a$ is $C(a) \setminus \{1\}$, so if $[a,b] = 1$, $C(a) \setminus \{1\} = C(b) \setminus \{1\}$, so $C(a) = C(b)$, as $1 \in C(a), C(b)$). Therefore, as $[a_1, a_i] = 1$ for all $i = 1,...,m$, we have $C(a_i) = C(a_1)$ for all $i$, so that $ C(a_1) \cap ... \cap C(a_m) = C(a_1)$. Therefore, if the two lists are conjugate by some element $t$, then conjugating elements are precisely of the form $rt$, where $r \in C(a_1)$. We know that $a_1^y = b_1$ and $a_1^t = b_1$, so $y \in C(a_1)t$. Therefore, $y$ must be a conjugating element between the lists, a contradiction. Hence, the two lists are not conjugate. 
	
	\vs 
	
	Next, we explain why the negative outcome of Step 5 indeed tells us that the lists are not conjugate. Since $a_1^y = b_1$, conjugating elements between $a_1$ and $b_1$ are precisely those of the form $zy$, where $z \in C(a_1)$. By Theorem 2.69, there exists a conjugating element $zy$ with $\vert zy \vert \leq \alpha \mu + \beta$. Note that the latter inequality implies $\vert z \vert \leq \alpha \mu + \beta + \vert y \vert = r$ (because $\vert z \vert = \vert zyy^{-1} \vert \leq \vert zy \vert + \vert y^{-1} \vert  =  \vert zy \vert + \vert y \vert \implies \vert z \vert - \vert y \vert \leq \vert zy \vert $). So, if the two lists $\{a_1, a_i\}$ and $\{b_1, b_i\}$ are conjugate, then there exists $z \in C(a_1) \cap B(r,1)$ such that $a_i^{zy} = b_i$ for the given $i$ in Step 5. Therefore, the negative outcome implies the lists $\{a_1, a_i\}$ and $\{b_1, b_i\}$ are not conjugate, so that the full lists $\{a_1,...,a_m\}$ and $\{b_1,...,b_m\}$ are not conjugate. 
	
	\vs 
	
	Lastly, we explain why the negative outcome of Step 6 implies that the lists are not conjugate. Because torsion-free hyperbolic groups are CT, commutativity is an equivalence relation for non-trivial elements and, as mentioned above, the equivalence class of $a \neq 1$ is $C(a) \setminus \{1\}$. Therefore, since $[a, a_i] \neq 1$, we have $(C(a) \setminus \{1\}) \cap (C(a_i) \setminus \{1\}) = \emptyset$, so that $C(a) \cap C(a_i) = \{1\}$. Since $a_1^{zy} = b_1$ and $a_i^{zy} = b_i$ (from Step 5), the conjugating elements between the lists $\{a_1, a_i\}$ and $\{b_1, b_i\}$ are precisely those in $(C(a_1) \cap C(a_i))zy = \{1\}zy$. Therefore, $zy$ is the unique conjugating element between $\{a_1, a_i\}$ and $\{b_1, b_i\}$. Thus, $zy$ is the only possible candidate which can conjugate $\{a_1,...,a_m\}$ to $\{b_1,...,b_m\}$. So, if $a_k^{zy} \neq b_k$ for some $k$, then there can be no element which conjugates $\{a_1,...,a_m\}$ to $\{b_1,...,b_m\}$. 
	
	\newpage
	
	\section{Conclusion: Future and ongoing work}
	
	In this paper, we have outlined the fundamentals of the theory of relatively hyperbolic groups. We have introduced relatively hyperbolic groups as objects aimed at generalizing the familiar class of hyperbolic groups and we have discussed the rich geometric and algebraic structure of relatively hyperbolic groups, seeing how such structure permits a host of algorithmic problems, many of which fail to be solvable in many well-known classes of groups, to be solvable in relatively hyperbolic groups. 
	
	\vs 
	
	Having seen an algorithm to solve the generalized conjugacy problem in torsion-free hyperbolic groups, we describe future and ongoing work to generalize this algorithm to finite lists of hyperbolic elements in torsion-free groups hyperbolic relative to a finite set of subgroups each of which having solvable word problem, as well as the challenges faced in doing so. 
	
	\vs 
	
	The main challenge in generalizing to the relatively hyperbolic case arises in generalizing Theorem 2.69, on the bound of conjugating elements between the lists. Many of the techniques used in proving Theorem 2.69 rely on hyperbolic geometry of the Cayley graph of $G$ and hence are generalizable to the relative Cayley graph, which provides a bound on the relative length of conjugating elements. However, in order to ensure finiteness of balls for the algorithm, we must establish an upper bound on the non-relative length of conjugating elements, which requires additional work. For example, the proof of Theorem 2.69 relies on elements $a_i$ acting by translations on geodesics in the Cayley graph. We may obtain that these elements act by translations on geodesics in the relative Cayley graph using the same techniques as presented in Bridson and Howie's paper, but it is not clear if the elements $a_i$ can act by translations on special geodesics in the non-relative Cayley graph, which would be required to obtain a bound on the $X$-length of a conjugating element, if we are to follow the structure of the proof of Theorem 2.69. Proposition 2.70 appears to not pose any issues in generalizing to the relatively hyperbolic case, though some details remain to be worked out. The author has found, however, that Lemma 2.71 generalizes completely to the relative hyperbolic case. Thus, the main problem in generalizing Theorem 2.69 lies in issues of relative vs non-relative translation lengths. The author continues to explore ways to generalize the proof of Theorem 2.69, as well as alternate techniques to generalize Theorem 2.69. Beyond this simplified case of solving the generalized conjugacy problem for lists of hyperbolic elements in torsion-free relatively hyperbolic groups, there remain the cases of torsion groups as well as lists containing parabolic elements, on which the author has not made much progress. 
	
	\vs 
	
	In the algorithm itself, the first four steps generalize without any difficulties (note that finitely generated torsion-free relatively hyperbolic groups are CT, as mentioned in the section on algebraic propertoes of relatively hyperbolic groups). Step 5 generalizes completely once we establish appropriate bounds on the non-relative length of conjugating elements. Step 6 also poses no difficulty in the generalization as it relies only on the CT property. Therefore, we see that the main hurdle in generalizing the algorithm arises in finding an upper bound on the non-relative length of conjugating elements between the two lists. To derive the appropriate bounds on the non-relative length of conjugating elements requires venturing beyond purely hyperbolic geometry of the relative Cayley graph and tying the non-relative Cayley graph into the picture, as we have done throughout this paper. The author believes the generalization of the bound on conjugating elements to be plausible and has made some progress in this direction, but no conclusive results have yet been obtained. 
	
	\vs
	
	Besides the algorithmic problems presented in this paper, there are other algorithmic problems which also admit solutions in relatively hyperbolic groups (provided the parabolic subgroups admit solutions to appropriate algorithmic problems, most frequently the word problem) that we did not discuss. Namely, relatively hyperbolic groups admit solutions to the \textit{order problem} (which asks to compute the order of a given element $g$ in the group), the \textit{root problem} (which, given some positive integer $n > 1$ and an element $g$ in the group, asks to determine if an $n$-th of $g$ exists in the group (that is, if there is an element $f$ in the group such that $f^n = g$) and find an $n$-th root if it exists), and the \textit{power conjugacy problem}, which asks whether given two elements $g,h$ in the group, if there exist integers $\ell, k$ such that $g^{\ell} \sim h^k$. The solutions to these problems for hyperbolic elements is described in Osin's paper [16] (many of the algebraic properties of hyperbolic elements crucial to the solution of the above problems, such as having only finitely many conjugacy classes, fail for parabolic elements in general, so the order, root and power conjugacy problems are only considered for hyperbolic elements in Osin's paper). 
	
	\vs 
	
	We conclude by noting that relatively hyperbolic groups are not the widest class of groups which generalize hyperbolic groups. Going beyond relatively hyperbolic groups, there exist the more general \textit{groups with hyperbolically embedded subgroups} described in [6], from which arise \textit{acylindrically hyperbolic groups} described by Osin in [15]. Beyond even these structures, the still more general \textit{hierarchically hyperbolic spaces and groups} were introduced in 2017 by Sisto in [17] and discussed further by Behrstock, Hagen and Sisto in [1], offer a very rich framework generalizing the structure of many classes of groups, including mapping class groups and relatively hyperbolic groups. Such structures offer exciting avenues for further study of generalizations of hyperbolicity and related algorithmic problems in a similar spirit to those discussed in this thesis. 
	
	
	\newpage
	
	\section{Appendix: Review of elementary group theory}
	
	\underline{Basic definitions and examples}
	
	\vs
	
	\textbf{Definition A.1}: A \textit{group} is a set $G$ equipped with a binary operation $\cdot$ (that is, a map $\cdot: G \times G \rightarrow G$) which satisfies the following axioms: 
	
	\begin{itemize}
		\item (G1) $\cdot$ is associative. That is, for any $g,h,k \in G$, $g\cdot(h \cdot k) = (g \cdot h) \cdot k$.
		\item (G2): That is, there exists $e \in G$ such that for all $g \in G$, $g\cdot e = e \cdot g = g$.
		\item (G3): For any $g \in G$, there exists $h \in G$ such that $g \cdot h = h \cdot g = e$. We denote the element $h$ by $g^{-1}$.
	\end{itemize}
	
	In addition, a group is called \textit{abelian} if $\cdot$ is commutative, that is, if $g \cdot h = h \cdot g$ for all $g,h \in G$. 
	
	\vs 
	
	The identity element and inverses in any group are unique (this is left to the reader to verify). 
	
	\vs
	
	\textbf{Example A.2 (Examples of Groups)}: 
	
	Groups often arise as symmetries of various mathematical objects as well as in the arithmetic structure of number systems.
	
	\begin{itemize}
		\item $\mathbb{Z}, \mathbb{Q}, \mathbb{R}, \mathbb{C}$ equipped with addition $+$ are all abelian groups. 
		\item If $k$ is any field, then $k^{*} = k \setminus \{0\}$ is a group under the field multiplication, called the group of units of $k$.
		\item When $X$ is any non-empty set, the set $S_X := \{\sigma: X \rightarrow X  : \sigma \text{ is a bijection}\}$ is a group under the operation of composition of maps. $S_X$ is called the \textit{symmetric group} on $X$. When $X = \{1,2,...,n\}$ (where $n \in \mathbb{N} := \{1,2,3,...\}$), we denote $S_X$ by $S_n$.
		\item For $n \in N$, $GL_n(\mathbb{R}) := \{A \in M_n(\mathbb{R})  : \text{det}(A) \neq 0\}$ is a group under matrix multiplication. 
		\item $\mathbb{Z}_n$ under addition of residue classes is an abelian group. This is called the \textit{cylic group of order n}. 
	\end{itemize}
	
	\vs
	
	We will often omit writing the symbol of the operation and we will often denote the group $G$ with operation $\cdot$ simply by the underlying set $G$.
	
	\vs
	
	\textbf{Definition A.3}: A subset $H$ of a group $G$ is called a \textit{subgroup} of $G$ if $H$ is a group under the same operation as in $G$. As associativity is inherited from $G$, this definition is equivalent to the following properties: 
	
	\begin{itemize}
		\item $H$ is closed under the group operation. That is, for all $h,k \in H$, $hk \in H$
		\item The identity element $e$ of $G$ is in $H$.
		\item $H$ is closed under inverses. That is, for any $h \in H$, $h^{-1} \in H$. 
	\end{itemize}
	
	\vs 
	
	If $H$ is a subgroup of $G$, we will denote this by $H \leq G$.
	
	\vs 
	
	\textbf{Example A.4 (Examples of subgroups): } 
	
	\begin{itemize}
		\item $\mathbb{Z} \leq \mathbb{Q} \leq \mathbb{R} \leq \mathbb{C}$ are all subgroups under +.
		\item $SL_n(\mathbb{R}) := \{A \in GL_n(R) : \text{det}(A) = 1\}$ and $O(n) = \{A \in GL_n(\mathbb{R}) : A^T A = A A^T = I\}$ are both subgroups of $GL_n(\mathbb{R})$.
	\end{itemize}
	
	\vs 
	
	\underline{Group homomorphisms and isomorphisms}
	
	\vs
	
	We briefly review the structure preserving maps of group theory. 
	
	\vs
	
	\textbf{Definition A.5: } Let $G,H$ be groups. A \textit{group homomorphism} from $G$ to $H$ is a map $\varphi: G \rightarrow H$ satisfying $\varphi(g_1g_2) = \varphi(g_1)\varphi(g_2)$ for all $g_1, g_2 \in G$. A bijective group homomorphism is called a \textit{group isomorphism}. A group isomorphism from a group $G$ onto itself is called an \textit{automorphism} of $G$.
	
	\vs 
	
	\textbf{Definition A.6: } Let $\varphi: G \rightarrow H$ be a group homomorphism. The \textit{kernel} of $\varphi$, denoted $\ker(\varphi)$, is $\varphi^{-1} (\{e_H\}) = \{g \in G : \varphi(g) = e_H\}$.
	
	\vs 
	
	\textbf{Proposition A.7 (Properties of group homomorphisms): } Let $\varphi: G \rightarrow H$ be a group homomorphism. Then:
	
	\begin{enumerate} [label = (\alph*)]
		\item $\varphi(e_G) = e_H$
		\item $\varphi(g^{-1}) = \varphi(g)^{-1}$
		\item $\varphi(A) \leq H$ for any $A \leq G$
		\item $\varphi^{-1}(B) \leq G$ for any $B \leq H$ (in particular, $\ker(\varphi) \leq G$)
		\item $\ker(\varphi) = \{e_G\} \iff \varphi$ is injective
	\end{enumerate}
	
	\begin{proof}
		
		(a): Let $g \in G$ be any element. Then $\varphi(g) = \varphi(ge_G) = \varphi(g) \varphi(e_G)$. Canceling $\varphi(g)$ on both sides yields $e_H = \varphi(e_G)$. 
		\vs
		(b): We have $\varphi(g) \varphi(g^{-1}) = \varphi(g g^{-1}) = \varphi(e_G) = e_H$. Therefore, $\varphi(g^{-1}) = \varphi(g)^{-1}$. 
		\vs
		(c): Given $h_1, h_2 \in \varphi(A)$, we have $h_1 = \varphi(a_1)$ and $h_2 = \varphi(a_2)$ for some $a_1, a_2 \in A$. Then $h_1 h_2 = \varphi(a_1) \varphi(a_2) = \varphi(a_1 a_2) \in \varphi(A)$, as $a_1a_2 \in A$. Next, $e_H \in \varphi(A)$ by (a), as $e_G \in A$. Lastly, given $h \in \varphi(A)$, $h = \varphi(a)$ for some $a \in A$, so then by (b) $h^{-1} = \varphi(a)^{-1} = \varphi(a^{-1}) \in \varphi(A)$. Therefore, $\varphi(A) \leq H$. 
		\vs
		(d) : Given $g_1, g_2 \in \varphi^{-1}(B)$, we have $\varphi(g_1 g_2) = \varphi(g_1) \varphi(g_2) \in B$ because $\varphi(g_1), \varphi(g_2) \in B$ and $B \leq H$, so $g_1 g_2 \in \varphi^{-1}(B)$. Next, $e_G \in \varphi^{-1}(B)$ as $\varphi(e_G) = e_H \in B$. Lastly, given $g \in \varphi^{-1}(B)$, by (b) we have $\varphi(g^{-1}) = \varphi(g)^{-1} \in B$ as $\varphi(g) \in B$, therefore $g^{-1} \in \varphi^{-1}(B)$. Therefore, $\varphi^{-1}(B) \leq G$. 
		\vs 
		(e): Suppose $\ker(\varphi) = \{e_G\}$ and suppose that $\varphi(g_1) = \varphi(g_2)$. Then $\varphi(g_1 g_2^{-1}) = \{e_H\} \implies g_1 g_2^{-1} = e_G \implies g_1 = g_2$. Thus, $\varphi$ is injective. Conversely, suppose that $\varphi$ is injective. If $g \in \ker(\varphi)$ then $\varphi(g) = e_H = \varphi(e_G) \implies g = e_G$. Therefore, $\ker(\varphi) = \{e_G\}$. 
		
	\end{proof}
	
	\vs
	
	\underline{Operations on subgroups and groups generated by subsets}
	
	\vs 
	
	One of the most basic operations that we can do with subgroups is to intersect them. This will always yield another subgroup. 
	
	\vs 
	
	\textbf{Proposition A.8:} Let $G$ be a group and let $(H_i)_{i \in I}$ be a collection of subgroups of $G$ (where $I$ denotes any index set). Then $\cap_{i \in I} H_i \leq G$. 
	
	\vs 
	
	\begin{proof}
		We verify each of the axioms given in Definition A.3. 
		
		\begin{itemize}
			\item Let $h,k \in \cap_{i \in I} H_i$. Then $h,k \in H_i$ for all $i \in I$. Since $H_i$ is a subgroup of $G$ for all $i \in I$, we have $hk \in H_i$ for all $i \in I$. Therefore, $hk \in \cap_{i \in I} H_i$. 
			\item Since each $H_i$ is a subgroup of $G$ for all $i \in I$, we have $e \in H_i$ for all $i \in I$, so, $e \in \cap_{i \in I} H_i$. 
			\item If $h \in \cap_{i \in I} H_i$, then as $H_i$ is a subgroup for all $i \in I$, we have $h^{-1} \in H_i$ for all $i \in I$, hence, $h^{-1} \in \cap_{i \in I} H_i$. 
		\end{itemize}
		
		Therefore, $\cap_{i \in I} H_i$ is a subgroup of $G$.
		
	\end{proof}
	
	Recall that the union of subgroups is in general not a subgroup, as the following example shows. 
	
	\vs
	
	\textbf{Example A.9: } Consider the following additive subgroups of $\mathbb{Z}$: $2 \mathbb{Z} := \{2n: n \in \mathbb{Z}\}$ and $3 \mathbb{Z} := \{3n: n \in \mathbb{Z}\}$. Then $2,3 \in 2\mathbb{Z} \cup 3\mathbb{Z}$ but 5 = 2 + 3 is not in $2 \mathbb{Z} \cup 3 \mathbb{Z}$ since 5 is neither divisible by 2 nor by 3, so $2 \mathbb{Z} \cup 3 \mathbb{Z}$ fails to be a subgroup of $\mathbb{Z}$. 
	
	\vs 
	
	The fact that the intersection of subgroups is always a subgroup validates the following definition: 
	
	\vs 
	
	\textbf{Definition A.10: } Let $G$ be a group and let $S$ be any subset of $G$. The \textit{subgroup generated by S}, denoted $\langle S \rangle$, is the intersection of all subgroups of $G$ containing $S$. Equivalently (proof omitted), $\langle S \rangle = \{\Pi_{i=1}^n s_i^{\varepsilon_i} : s_i \in S \text{ and } \varepsilon_i = \pm 1, n \in \mathbb{N}\}$
	
	\vs 
	
	We now turn to perhaps the most important operation in non-abelian groups, conjugation. 
	
	\vs 
	
	\textbf{Definition A.11: } Let $G$ be a group and let $g,h \in G$. The \textit{conjugate of h by g} is the element $h^g := ghg^{-1}$. Moreover, we say that two elements $g,h \in G$ are \textit{conjugate} if there exists $t \in G$ such that $g^t = h$. 
	
	\vs 
	
	We may define a relation $\sim$ on $G$ by $g \sim h$ if and only if $g,h$ are conjugate in $G$. Recall that $\sim$ is an equivalence relation, as for any $g \in G$, $g$ is conjugate to itself via $e$ so $g \sim g$, given $g,h \in G$ if $g \sim h$ then $\exists t \in G$ such that $g^t = h$, but then $h^{t^{-1}} = g$, so $h \sim g$, and if $g \sim h$ and $h \sim k$, then $\exists s \in G$ such that $g^s = h$ and $\exists t \in G$ such that $h^t = k$ but then $g^{ts} = (ts)g(ts)^{-1} = t(sgs^{-1})t^{-1} = tht^{-1} = k$, so $g \sim k$. 
	
	\vs 
	
	We denote the $\sim$ equivalence class generated by $g \in G$ by $[g]_G$ and call it the \textit{conjugacy class of g in G}. 
	
	\vs 
	
	\textbf{Proposition A.12: } For any $g \in G$, the map $\tau_g: G \rightarrow G$ given by $\tau_g(h) = ghg^{-1} \forall h \in G$ is an automorphism of $G$. 
	
	\begin{proof}
		
		First, $\tau_g$ is a group homomorphism because for any $h_1, h_2 \in G$, we have $\tau_g(h_1 h_2) = g h_1 h_2 g^{-1} = g h_1 g^{-1} g h_2 g^{-1} = \tau_g(h_1) \tau_g(h_2)$. 
		
		\vs 
		
		Next, $\tau_g$ is bijective because for any $h \in G$ $(\tau_g \circ \tau_{g^{-1}}) (h) = (g g^{-1} h g g^{-1}) = h = g^{-1} g h g^{-1} g = (\tau_{g^{-1}} \circ \tau_{g}) (h)$. Therefore, $\tau_g \circ \tau_{g^{-1}} = \tau_{g^{-1}} \circ \tau_g = \text{id}_G$, so $\tau_g$ is bijective. 
		
	\end{proof}
	
	The automorphisms $\tau_g$ are called \textit{inner automorphisms}. As an immediate consequence of the above proposition, it follows that for any subgroup $H \leq G$ and any $g \in G$ that $\tau_g(H) = gHg^{-1}$ is a subgroup of $G$, called the conjugate subgroup of $H$ by $g$. Subgroups which are fixed set-wise by inner automorphisms have a special place in group theory. 
	
	\vs 
	
	\textbf{Definition A.13: } A subgroup $H$ of a group $G$ is called a \textit{normal subgroup} of $G$ if $gHg^{-1} = H$ for all $g \in G$. If $H$ is normal in $G$ we write $H \triangleleft G$.
	
	\vs 
	
	There are several equivalent characterizations of normality, the proof of which we omit. 
	
	\vs 
	
	\textbf{Proposition A.14: } The following are equivalent for a subgroup $H$ of a group $G$. 
	
	\begin{enumerate}[label = (\alph*)]
		\item $\forall g \in G$, $gHg^{-1} = H$ (i.e. $H \triangleleft G$)
		\item $\forall g \in G$, $gHg^{-1} \subseteq H$
		\item $\forall g \in G$, $gH = Hg$
	\end{enumerate}
	
	\vs 
	
	Recall that given a subgroup $H$ of a group $G$ and an element $g \in G$, the set $gH := \{gh: h \in H\}$ (resp. $Hg := \{hg: h \in H\}$) is called the \textit{left (resp. right) coset} of $H$ generated by $g$. The element $g$ is called a \textit{representative} of the left (resp. right) coset $gH$ (resp. $Hg$). The set of all left cosets is denoted $G/H$. We may define a relation on $G$ by $g_1 \sim g_2 \iff g_2^{-1} g_1 \in H \iff g_1 H = g_2 H$. One can easily verify that this is an equivalence relation and that the $\sim$- equivalence classes are precisely the left cosets of $H$ in $G$. The set $G/H$ inherits an operation from $G$ as follows: $(g_1H)(g_2H) := (g_1 g_2)H$. However, in general this operation will not be well-defined (i.e. independent of coset representatives). For that, we require $H \triangleleft G$. 
	
	\vs
	
	\textbf{Proposition A.15: } Let $G$ be a group and $H \leq G$. Then the operation on $G/H$: $(g_1 H) (g_2 H) = (g_1 g_2) H$ is well-defined if and only if $H \triangleleft G$. 
	
	\begin{proof}
		
		Suppose that the multiplication of left cosets is well-defined, i.e., that if $a_1H = a_2H$ and $b_1 H = b_2 H$ then $(a_1 b_1)H = (a_2 b_2)H$. Let $g \in G$ and let $h \in H$. Note that $ghH = gH$ and $g^{-1}hH = g^{-1}H$. Thus, by the well-definiteness of left coset multiplication, we have: $H = (gg^{-1})H = (ghg^{-1}h)H = (ghg^{-1})H$. Therefore, $ghg^{-1} \in H$. As $h \in H$ was arbitrary, we conclude that $gHg^{-1} \subseteq H$. Hence, $H \triangleleft G$. 
		
		\vs 
		
		Conversely, suppose that $H \triangleleft G$ and suppose that $a_1,a_2,b_1,b_2 \in G$ are such that $a_1 H = a_2 H$, so that $a_2^{-1}a_1 \in H$ and $b_1 H = b_2 H$, so that $b_2^{-1}b_1 \in H$. Then $(a_2 b_2)^{-1} (a_1 b_1) = b_2^{-1} (a_2^{-1} a_1) b_1$. Now $a_2^{-1}a_1 \in H$, so denoting $h = a_2^{-1}a_1$, we have $(a_2 b_2)^{-1} (a_1 b_1) = b_2^{-1} h b_1$. Since $H \triangleleft G$ we have $b_1^{-1}h b_1 = h' \in H$, so $h b_1 = b_1 h'$. Thus, $b_2^{-1} h b_1 = b_2^{-1} b_1 h'$. Since $b_2^{-1}b_1 \in H$ and $h' \in H$ we have that $b_2^{-1} b_1 h' \in H$. Therefore, $(a_2 b_2)^{-1} (a_1 b_1) \in H$, so $(a_1 b_1)H = (a_2 b_2)H$.
		
	\end{proof}
	
	Once the operation on left cosets is well-defined, it is easy to see that it gives a group structure to $G/H$. In this case, we call $G/H$ the \textit{quotient group} of $G$ by $H$. An analogous construction can be done for right cosets. 
	
	\vs
	
	Recall that given two groups $G, H$, there is a natural group structure on their cartesian product $G \times H$, defined by componentwise multiplication: $(g_1, h_1)(g_2, h_2) = (g_1 g_2, h_1, h_2)$ for all $(g_1, h_1),(g_2, h_2) \in G \times H$. It is straightforward to verify that this operation defines a group structure on $G \times H$, which has identity $e = (e_G, e_H)$. This is called the \textit{external direct product of G and H}.
	
	\vs
	
	\textbf{Definition A.17: } The \textit{direct product} of groups $G,H$ is the group $G \times H$ with group operation being componentwise multiplication. 
	
	\vs 
	
	We may also define the notion of \textit{internal product} of subgroups $H,K$ of a group $G$, given by the product of the subsets of $G$: $HK := \{hk: h \in H, k \in K\}$. Recall however that $HK$ is not a subgroup of $G$ in general (for example, consider $G = S_3$, $H = \langle (12) \rangle$ and $K = \langle (23) \rangle$, then $HK = \{\text{id}, (12), (23), (123)\}$ is not a subgroup since $(123)^{-1} = (132) \notin HK$), but is a subgroup provided at least one of $H,K$ is normal in $G$. 
	
	\vs 
	
	\textbf{Proposition A.18: } Let $H,K$ be subgroups of a group $G$ with at least one of $H,K$ normal in $G$. Then $HK \leq G$. 
	
	\begin{proof}
		
		Without loss of generality, suppose that $H \triangleleft G$.
		
		\vs
		
		Let $a,b \in HK$. Then $a = h_1 k_1$ and $b = h_2 k_2$ for some $h_1, h_2 \in H$ and $k_1, k_2 \in K$. Then $ab = h_1 k_1 h_2 k_2$. Since $H \triangleleft G$, $k_1 h_2 k_1^{-1} = h_3 \in H$, so $k_1 h_2 = h_3 k_1$. Therefore, $ab = h_1 h_3 k_1 k_2 \in HK$. 
		
		\vs 
		
		We have $e \in H,K$ so $e = ee \in HK$.
		
		\vs 
		
		Let $a = hk \in HK$. Then $a^{-1} = k^{-1}h^{-1}$. Since $H \triangleleft G$, $k^{-1} h^{-1} k = h_1 \in H \implies k^{-1} h^{-1} = h_1 k^{-1} \in HK$. Therefore, $a \in HK$.
		
	\end{proof}
	
	\underline{Group actions on sets}
	
	\vs 
	
	One of the most important applications of group theory are group actions. Group actions show us precisely how groups can be expressed as symmetries of various mathematical objects. We consider here groups represented as symmetries of the most basic mathematical objects: sets, in which the symmetries are bijections. 
	\vs 
	
	\textbf{Definition A.19: } Let $X$ be any non-empty set and let $G$ be a group. A \textit{group action} of $G$ on $X$ is a map $\cdot: G \times X \rightarrow X$ which satisfies the following properties: 
	
	\begin{enumerate}[label = (\roman*)]
		\item For all $g,h \in G$ and $x \in X$, we have $g \cdot (h \cdot x) = (gh) \cdot x$
		\item For all $x \in X$, we have $e \cdot x = x$
	\end{enumerate}
	
	If we have a group action of $G$ on $X$, we say that $G$ acts on $X$, writing $G \curvearrowright X$, and we call $X$ a $G$-set. 
	
	\vs 
	
	An equivalent way to think about group actions is in terms of group homomorphisms, in the sense of the following theorem: 
	
	\vs 
	
	\textbf{Theorem A.20: } Let $G$ be a group and $X$ a non-empty set. A group action of $G$ on $X$ induces a group homomorphism $\rho: G \rightarrow S_X$. Conversely, every group homomorphism $\rho: G \rightarrow S_X$ induces a group action of $G$ on $X$. 
	
	\begin{proof}
		
		Suppose we have a group action $\cdot$ of $G$ on $X$. Define $\rho: G \rightarrow S_X$ by $\rho(g) = \sigma_g$, where $\sigma_g: X \rightarrow X$ is defined by $\sigma_g(x) = g \cdot x$ for all $x \in X$. Note that $\sigma_g \in S_X$ for all $g \in G$ because given $x \in X$, $(\sigma_g \circ \sigma_{g^{-1}}) (x) = g \cdot (g^{-1} \cdot x) = (gg^{-1}) \cdot x = e \cdot x = x$ and similarly $(\sigma_{g^{-1}} \circ \sigma_g) (x) = g^{-1} \cdot (g \cdot x) = (g^{-1} g) \cdot x = e \cdot x = x$. Therefore, $\sigma_g$ is invertible, so $\sigma_g \in S_X$. 
		
		\vskip5pt
		
		Next, we show that $\rho$ is a group homomorphism. We have $\rho(gh) = \sigma_{gh}$. Now given $x \in X$, $\sigma_{gh}(x) = (gh) \cdot x = g \cdot (h \cdot x) = \sigma_g(\sigma_h (x)) = (\sigma_g \circ \sigma_h) (x)$. Therefore, $\sigma_{gh} = \sigma_g \circ \sigma_h = \rho(g) \circ \rho(h)$. Thus, $\rho$ is a group homomorphism. 
		
		\vs 
		
		Conversely, suppose we have a group homomorphism $\rho: G \rightarrow S_X$. We define an action of $G$ on $X$ by $g \cdot x = \rho(g)(x)$. This is indeed an action because given $g,h \in G$ and $x \in X$, we have $g \cdot (h \cdot x) = \rho(g) (\rho(h) (x)) = (\rho(g) \circ \rho(h)) (x) = \rho(gh) (x) = (gh) \cdot x$, and $e \cdot x = \rho(e) (x) = \text{id}_X (x) = x$, where $\rho(g) \circ \rho(h) = \rho(gh)$ and $\rho(e) = \text{id}_X$ hold because $\rho$ is a group homomorphism. 
	\end{proof}
	
	\vs 
	
	The homomorphism $\rho: G \rightarrow S_X$ in the above theorem is called a \textit{permutation representation} of $G$ on $X$. For the readers familiar with category theory, the permutation representation picture of group actions allows us to define a group action on any mathematical object in a natural way. Given any category $\mathcal{C}$ and an object $A \in \text{Ob} (\mathcal{C})$, one can define the action of $G$ on $A$ as a group homomorphism $\rho: G \rightarrow \text{Iso}(A)$, where $\text{Iso}(A)$ denotes the group of invertible morphisms from $A$ to $A$ (with operation composition of morphisms). In this way, group actions allow us to realize groups as symmetries of arbitrary mathematical objects. 
	
	\vs 
	
	We next discuss some terminology related to group actions.
	
	\vs
	
	\textbf{Definition A.21: } The \textit{orbit} of $x \in X$ under a group $G$ is $G\cdot x = \{g \cdot x: g \in G\}$. The \textit{stabilizer} of $x \in X$ is $\text{Stab}(x) = \{g \in G: g\cdot x = x\}$
	
	\vs
	
	\textbf{Definition A.22: } A group action $G \curvearrowright X$ is called \textit{faithful} if $g \cdot x = x \text{ } \forall x \in X \implies g = e$. This is equivalent to the induced permutation representation $\rho: G \rightarrow S_X$ being injective. 
	
	\vs 
	
	\textbf{Definition A.23: } A group action $G \curvearrowright X$ is called \textit{free} if for all $g \in G$, $g \cdot x = x$ for some $x \in X$ implies $g = e$. This is equivalent to all stabilizers being trivial.
	
	\newpage
	
	\section{Bibliography}
	
	[1] Jason Behrstock, Mark Hagen, and Alessandro Sisto. Hierarchically hyperbolic spaces II: Combination theorems and the distance formula. Pacific J. Math., 299(2):257–338, 2019.
	\vs
	[2] W. W. Boone. The word problem. \textit{Annals of mathematics}, pages 207-265, 1959.  
	\vs
	[3] Bridson and Haefliger book, M. R. Bridson and A. Haefliger, “Metric Spaces of Non-Positive Curvature”, Grund. Math. Wiss. 319, Springer-Verlag, Berlin-Heidelberg-New York, 1999.
	\vs
	[4] M.R. Bridson, J. Howie, \textit{Conjugacy of finite subsets in hyperbolic groups}, preprint. 
	\vs
	[5] I. Bumagin, \textit{Time complexity of the conjugacy problem in relatively hyperbolic groups}, arXiv:1407.4528, 16 Jul 2014. 
	\vs
	[6] F. Dahmani, V. Guirardel, D. Osin. \textit{Hyperbolically embedded subgroups and rotating families in groups acting on hyperbolic spaces}, arXiv:1111.7048, 2 Dec 2014.
	\vs
	[7] T. Delzant, Sous-groupes distinguées et quotients des groups hyperboliques, Duke Math. J. 83 (1996), 661-682.
	\vs
	[8] B. Farb, Relatively hyperbolic groups, GAFA, 8 (1998), 810–840.
	\vs
	[9] B. Farb, The extrinsic geometry of subgroups and the generalized word prob-
	lem, Proc. London Math. Soc. 68 (1994), 3, 577–593.
	\vs
	[10] B. Fine, A. Gaglione, G. Rosenberger, D. Spellman, \textit{On CT and CSA Groups and Related Ideas}, arXiv:1506.02636v1, 8 Jun 2015. 
	\vs
	[11] M. Gromov, Hyperbolic groups, in “Essays on Group Theory” (S.M. Gersten ed.), MSRI Publ. 8, Springer-Verlag, 1987, pp. 75–263.
	\vs
	[12] Frank R. Kschischang, \textit{The Subadditivity Lemma}. University of Toronto. 3 Nov 2009. 
	\vs
	[13] R.C. Lyndon, P.E. Shupp, Combinatorial Group Theory, Springer–Verlag,
	1977.
	\vs
	[14] Petr Sergeevich Novikov. On the algorithmic unsolvability of the word problem in group theory. Trudy Matematicheskogo Instituta imeni VA Steklova, 44:3-143, 1955. 
	\vs
	[15] D. Osin, \textit{Acylindrically hyperbolic groups}, arXiv:1304.1246, 16 Apr, 2015.
	\vs
	[16] D. Osin, \textit{Relatively hyperbolic groups: Intrinsic geometry, algebraic properties, and algorithmic problems}, arXiv:math/0404040, 2 Apr 2004. 
	\vs
	[17] A. Sisto, \textit{What is a hierarchically hyperbolic space?}, arXiv:1707.00053, 30 Jun 2017. 
	\vs
	[18] Y. F. Wu, Groups in which Commutativity is a Transitive Relation, J. of Algebra, 207, 1998,165-181.
	\vs
	[19] M. Zhang, \textit{On Properties of Relatively Hyperbolic Groups}. (PhD Thesis). Carleton University, Ottawa, 2017. 
	
	

	
	
	
	
	

	
	
\end{document}